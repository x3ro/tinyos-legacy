\documentclass[12pt]{article}

\usepackage{graphicx}
\usepackage{graphics}
\usepackage{multicol}
\usepackage{epsfig,amsmath,amsfonts}

\makeatletter                                   % Make '@' accessible. 
\oddsidemargin=0in                              % Left margin minus 1 inch.
\evensidemargin=0in                             % Same for even-numbered pages.
\marginparsep=0pt                               % Space between margin & text

\renewcommand{\baselinestretch}{1.5}              % Double spacing

\textwidth=6.5in                                % Text width (8.5in - margins).
\textheight=9in                                 % Body height (incl. footnotes)

\topmargin=0in                                  % Top margin minus 1 inch.
\headsep=0.0in                                  % Distance from header to body.
\skip\footins=4ex                               % Space above first footnote.
\hbadness=10000                                 % No "underfull hbox" messages.
\makeatother                                    % Make '@' special again.


\begin{document}

\fontfamily{cmss}                               % Make text sans serif.
\fontseries{m}                                  % Medium spacing.
\fontshape{n}                                   % Normal: not bold, etc.
\fontsize{10}{10}                               % 10pt font, 10pt line spacing 

\title{TOSSIM System Description}
\author{Philip Levis}
\maketitle

\fontfamily{cmr}                                % Make text Roman (serif).
\fontseries{m}                                  % Medium spacing.
\fontshape{n}                                   % Normal: not bold, etc.
\fontsize{10}{10}                               % 10pt font, 10pt line spacing

\section*{System Overview}

TOSSIM (TinyOS SIMulator) is a simulation environment for TinyOS. It
simulates a set of motes all running the same TinyOS program. The
number of motes can be set at run time at the command line. TOSSIM
uses a discrete event simulation to handle core clock and radio clock
events. It simulates radio communication with a few simplifications:
e.g.  mote transmissions never cancel one another (they are always in
phase).

Messages can be injected into the simulation from an outside source,
and network traffic can be visualized or stored by an external
program.

Currently, TOSSIM has three network connectivity models: simple
connectivity (all the motes are in one cell), static connectivity (the
network graph is established at startup and never changes), and space
connectivity (the motes move around randomly in a square area, and
potentiometer settings change their transmission range).

Unless one uses an external program to read in data produced by TOSSIM
(e.g. the TOSSIM Java GUI), very little information is provided. Use
of {\tt dbg} settings by the {\tt DBG} environment variable allow for
some information on the state of the simulation (see {\tt
system/include/dbg\_modes.h}), but due to its volume this is only
really useful for post-run analysis.

\section*{Getting Started}

TOSSIM is compiled using {\tt MakefileTOS}, which is in the root
TinyOS directory ({\tt tos/tos} in the CVS tree). {\tt MakefileTOS} is
similar in structure to the other TinyOS makefiles; the desired
application is set using the {\tt DESC} variable, etc. Every mote in
TOSSIM runs the same code image (application).

The TOSSIM executable is named {\tt main}. It has the following usage:

{\tt main [-r <static|simple|space>] [-p usec] num\_nodes}

The {\tt -r} option specifies the radio model to use. None of the
current radio models take into account interference or noise; signals
are perfectly transmitted and received. The {\tt simple} model places
all of the motes in a single radio cell; every mote can hear every
other mote. The {\tt static} model reads in a file at startup ({\tt
cells.txt}) that specifies the which motes can hear each-other. The
format of this file is defined at the end of this document. The {\tt
space} model is a rudimentary model of 2D space. Motes are randomly
placed at startup, then move in a random, Brownian motion-like
fashion. Signal reception is based on distance between motes, with
potentiometer settings mapping to a distance table that has no basis
in reality. The default model is {\tt simple}.

The {\tt -p} option allows simulation text output to be more easily
understood. If it is specified, then the simulation pauses at every
system clock (not radio clock) event for the specified number of
microseconds. Since system clock events are often major determinants
of mote behavior, this allows a user to make the simulation output be
broken up enough to be read, instead of the standard constant stream.

The {\tt num\_nodes} option specifies how many nodes should be
simulated. A compile-time upper bound ({\tt TOSNODES}) is specified in
{\tt TOSSIM.h}. The standard distribution sets this value to be 1000.

\section*{The GUI}

The TOSSIM visualization GUI is included in the default TinyOS release
as a Java application. Currently, the simulation only supports radio
packet messages (incoming and outgoing). It resides in the {\tt
tools/} directory of the default TinyOS release, and can be run by
typing {\tt java TossimGUI} when in that directory. It has a single
optional parameter, {\tt -r}, for specifying a file of radio messages
to introduce to the simulation.

\subsection*{File Format}

TossimGUI allows a file of radio packets to be specified at the
command line. These packets are read in and then sent over port
10576. The file has the following format:

\vspace{0.1in}
$<$time (decimal long long)$>$ $<$mote (decimal short) $<$data0 (hex byte)$>$ ... $<$data37 (hex byte)$>$.
\vspace{0.1in}

Each value is separated from the other by whitespace. All whitespace
is insignificant. A sample file exists at {\tt tools/radio.txt}.

\section*{TOSSIM Internals}

Currently, TOSSIM compiles by using alternative implementations of
some core system components (e.g. {\tt MAIN.c}, {\tt CLOCK.c}) and
incorporating a few extra files (e.g. {\tt event\_queue.c}). Also,
TinyOS macros such as {\tt VAR} are provided alternative definitions
in {\tt tos.h}.

The basic TOSSIM structure definitions exist in {\tt
system/tossim/tossim.h}. These include the global simulation state
(event queue, time, etc.) and individual node state. Hardware settings
that affect multiple components (such as the potentiometer setting)
currently must be modeled as global node state; the {\tt static}
nature of TinyOS frames prevents otherwise. {\tt tossim.h} also
includes a few useful macros, such as {\tt NODE\_NUM} and {\tt
THIS\_NODE}; these should be used whenever possible, so minimal code
modifications will be necessary if the data representation
changes. They will probably soon be transitioned to functions for this
reason.

{\tt MAIN.c} is where the TOSSIM global state is defined. The {\tt
main()} function parses the command line options, initializes the
TOSSIM global state, initializes the external communication
functionality (defined in {\tt external\_comm.c}), then initializes the
state of each of the simulated motes with {\tt MAIN\_SUB\_INIT} and {\tt
MAIN\_SUB\_START}. It then starts executing a small loop that handles
events and schedules tasks. Currently, tasks are uninterruptible and
execute instantaneously (in simulation time). We hope to model tasks
more accurately soon.

TOSSIM catches SIGINT (control-C) to exit cleanly. This is useful when
profiling code.


\subsection*{RFM}

Instead of requiring a certain level of radio simulation accuracy,
TOSSIM uses an extensible radio model. By doing so, the simulation can
be run with any level of accuracy a user wants (and is willing to
write). While simple protocol correctness tests might be satisfied
with a model that assumes perfect signal transmission, protocol
performance tuning tests might want to introduce bit error and
transient interference.

The interface to the radio model is defined in {\tt rfm\_model.h}:

\begin{verbatim}
typedef struct {
  void(*init)();
  void(*transmit)(int, char);
  void(*stop\_transmit)(int);
  char(*hears)(int);
  void* data;
} rfm_model;
\end{verbatim}

The data pointer is provided so that radio models that require a large
amount of state can only allocate the memory if they're being used;
otherwise, if the state is comprised of global variables in source
files, the process uses up space for every model's bookeeping.

The {\tt int} arguments are mote IDs, while the {\tt char} arguments
are bits (0 or 1).

\subsubsection*{Example Model: static}

Herein I will describe how the static rfm\_model works so people get a
feeling of why such a simple event model allows for very accurate
radio timing simulation.


\section*{External Communication}

TOSSIM includes functionality for introducing and reading network
traffic. This is achieved through the use of TCP sockets that are
opened before the simulation begins. TOSSIM initiates a connection to
localhost on a series of ports. Each of these ports, their use, and
their communication format are detailed below. In addition, TOSSIM
opens a single server (passive) port and waits for external
connections; client sockets connecting to this port can inject
messages into the network.

\subsection*{UART}

\subsubsection*{Reading: Port 10580}

The UART communication channel allows an external process to
communicate with the motes over their UARTs. At startup, the TOSSIM
process reads from the socket until the server side closes it. Data
sent over the socket must have the following format:

\vspace{0.1in}
\begin{tabular}{|c|c|c|}\hline
\hspace{4in} & \hspace{1in} & \hspace{0.5in} \\ \hline
Time (long long)& Mote ID (short) & Data(char) \\ \hline
\end{tabular}
\vspace{0.1in}

These messages cannot be generated dynamically; they are all read in
before the simulation starts. Once the server process has sent all of
its messages, it closes the connection to TOSSIM.

{\bf UART communication is currently not implemented.}

\subsubsection*{Writing: Port 10581}

UART messages are written using the same format with which they are
read. On startup, the simulator opens a connection to port 22577 on
localhost and send messages over this channel.

{\bf UART communication is currently not implemented.}

\subsection*{Radio}
\subsubsection*{Reading: Port 10576}

The radio communication channel allows an external process to
communicate with the motes over the radio. At startup, the TOSSIM
process reads from the socket until the server side closes it. Data
sent over the socket must have the following format:

\vspace{0.1in}
\begin{tabular}{|c|c|c|}\hline
\hspace{2in} & \hspace{2in} & \hspace{0.5in} \\ \hline
Time (long long)& Mote ID(short) & TOS\_Msg (38 bytes) \\ \hline
\end{tabular}
\vspace{0.1in}

These messages cannot be generated dynamically; they are all read in
before the simulation starts. The simulation stores them as
events. Sending messages in this manner bypasses the radio layers in
the mote, directly calling {\tt AM\_RX\_PACKET\_DONE}.

Once the server has sent all of its messages, it closes the connection
to TOSSIM.

This mechanism allows for the initialization of base stations at
simulation startup. Since all motes must run the same code image, base
station logic must be incorporated into the build of TinyOS. A message
handler can be used to receive messages that switch a mote from
standard to base-station mode, and messages of this type can be
introduced into the simulator through this mechanism.

\subsubsection*{Writing: Port 10577}

This socket notifies an external program whenever a mote starts to
transmit a packet. This notification occurs when the mote completes
the transmission; if there is a lot of radio traffic, this might be
significantly after it attempts to start transmitting. The packet is
sent along the socket in this format:

\vspace{0.1in}
\begin{tabular}{|c|c|c|}\hline
\hspace{2in} & \hspace{2in} & \hspace{0.5in} \\ \hline
Time (long long)& Mote ID(short) & Data (38 bytes) \\ \hline
\end{tabular}
\vspace{0.1in}

\subsubsection*{Bit Writing: Port 10578}

Radio messages are written to a connection made to this port. The
messages have the following format:

\vspace{0.1in}
\begin{tabular}{|c|c|c|}\hline
\hspace{4in} & \hspace{1in} & \hspace{0.5in} \\ \hline
Time (long long)& Mote ID(short) & Bit (byte) \\ \hline
\end{tabular}
\vspace{0.1in}

The bit states whether the mote transmits a 1 or doesn't transmit
(0). The mote is considered to be producing the same value until
another message saying otherwise arrives.

This functionality is useful if debugging activity below the packet
layer, such as encodings/start symbols.

\subsubsection*{Real-time Reading: Port 10579}

TOSSIM spawns an additional thread at startup, which opens a server
socket and waits for connections on port 10579. Clients connecting to
this port can inject messages dynamically into the network. Data sent
over this socket should have the following format:

\vspace{0.1in}
\begin{tabular}{|c|c|c|}\hline
\hspace{2in} & \hspace{2in} & \hspace{0.5in} \\ \hline
Time (long long)& Mote ID(short) & TOS\_Msg (38 bytes) \\ \hline
\end{tabular}
\vspace{0.1in}

If the time field specifies a time that has already passed in the
simulation, the message is scheduled to be received immediately. If
the time field specifies a time in the future, the message is
scheduled to be received at the specified time.

If the client closes the connection, TOSSIM will wait for another
connection request.

\section*{cells.txt}

There are currently two network connectivity models in TOSSIM. The simple model places every mote within a single cell; all of the motes can hear one another. This is the default network model. The second network model, the static model, uses a network connectivity graph that is determined at run-time before the simulation starts. The graph is specified by a file named {\tt cells.txt} which must be in the directory where the program is executed. The file format is very simple.

\begin{verbatim}

0:1 1:2 4:5 9:1

\end{verbatim}

defines a graph in which motes 0 and 1 can communicate, 2 and 1 can communicate, 4 and 5 can communicate, and 9 and 1 can communicate. Connectivity is bidirectional: {\tt 0:1} is the same as {\tt 1:0}. A sample file can be found in the root TinyOS directory.

\end{document}
