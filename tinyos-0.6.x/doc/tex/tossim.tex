\documentclass[12pt]{article}

\usepackage{graphicx}
\usepackage{graphics}
\usepackage{multicol}
\usepackage{epsfig,amsmath,amsfonts}

\makeatletter                                   % Make '@' accessible. 
\oddsidemargin=0in                              % Left margin minus 1 inch.
\evensidemargin=0in                             % Same for even-numbered pages.
\marginparsep=0pt                               % Space between margin & text

\renewcommand{\baselinestretch}{1.5}              % Double spacing

\textwidth=6.5in                                % Text width (8.5in - margins).
\textheight=9in                                 % Body height (incl. footnotes)

\topmargin=0in                                  % Top margin minus 1 inch.
\headsep=0.0in                                  % Distance from header to body.
\skip\footins=4ex                               % Space above first footnote.
\hbadness=10000                                 % No "underfull hbox" messages.
\makeatother                                    % Make '@' special again.


\begin{document}

\fontfamily{cmss}                               % Make text sans serif.
\fontseries{m}                                  % Medium spacing.
\fontshape{n}                                   % Normal: not bold, etc.
\fontsize{10}{10}                               % 10pt font, 10pt line spacing 

\title{TOSSIM System Description}
\author{Philip Levis}
\maketitle

\fontfamily{cmr}                                % Make text Roman (serif).
\fontseries{m}                                  % Medium spacing.
\fontshape{n}                                   % Normal: not bold, etc.
\fontsize{10}{10}                               % 10pt font, 10pt line spacing

\section*{System Overview}

TOSSIM is a discrete event simulator for TinyOS. It theoretically
scales with $O(n \cdot log(n))$ time for $n$ motes, and experimental
results show it to scale well up to one thousand motes. TOSSIM is
compiled directly from TinyOS code; compiling a TinyOS application
only requires specifying a different target for {\tt make} (typing
{\tt make pc} instead of {\tt make rene} or {\tt make
mica}). Compiling a TinyOS application to native code allows a user to
take advantage of traditional programming tools such as
debuggers. TOSSIM has an extensive external communication system so
testers can monitor packets being transmitted as well as dynamically
inject packets into the simulated network. TOSSIM also allows finely
grained configuration of debugging output at runtime.

Current general purpose network simulators are inadequate for
TinyOS. Traditional network simulators are usually focused on protocol
simulation for large area networks in which there is a large
connectivity disparity in regions of the network (e.g. backbones
vs. LANs). For example, {\tt ns-2} simulates at a packet
granularity and has detailed node and link specification. In contrast,
TOSSIM is intended for simulating homogeneous sensor networks, in
which each mote runs the same program. TOSSIM simulates at bit
granularity and has an extensible but so far simple network
connectivity model. This is because a good deal of research using
TinyOS looks at data link level mechanisms, whether for power
conservation or more efficient use of the available
bandwidth.

Currently, TOSSIM has three network connectivity models: simple
connectivity (all the motes are in one cell), static connectivity (the
network graph is established at startup and never changes), and space
connectivity (the motes move around randomly in a square area, and
potentiometer settings change their transmission range). All of these
models are very simplistic: e.g. mote transmissions never cancel one
another (they are always in phase), and bits are perfectly
transmitted.

Unless one uses an external program to read in data produced by TOSSIM
(e.g. the TOSSIM Java GUI), very little information is provided. Use
of {\tt dbg} settings by the {\tt DBG} environment variable allow for
some information on the state of the simulation (see {\tt
system/include/dbg\_modes.h}). However, this information can be
produced at a tremendous rate (e.g. if monitoring radio clock
interrupts) and therefore might only be useful for post-run analysis.

\section*{Getting Started}

TOSSIM is automatically compiled when you compile an
application. Applications are compiled by entering an application
directory (e.g. {\tt nest/apps/blink}) and typing {\tt
make}. Alternatively, when in an application directory, you can type
{\tt make pc}, which will only compile a TOSSIM version of the
program.

The TOSSIM executable is named {\tt main}, and resides in {\tt
binpc}. It has the following usage:


{\tt main [-h|--help] [-r $<$static|simple|space$>$] [-l] [-e $<$file$>$] [-kb $<$val$>$] [-p $<$sec$>$] <num\_nodes>}

The {\tt -h} or {\tt --help} options print out a more verbose usage
message, specifying which dbg modes are enabled in the simulator, etc.

The {\tt -r} option specifies the radio model to use. None of the
current radio models take into account interference or noise; signals
are perfectly transmitted and received. The {\tt simple} model places
all of the motes in a single radio cell; every mote can hear every
other mote. The {\tt static} model reads in a file at startup ({\tt
cells.txt}) that specifies the which motes can hear each-other. The
format of this file is defined at the end of this document. The {\tt
space} model is a rudimentary model of 2D space. Motes are randomly
placed at startup, then move in a random, Brownian motion-like
fashion. Signal reception is based on distance between motes, with
potentiometer settings mapping to a distance table that has no basis
in reality. The default model is {\tt simple}.

The {\tt -l} option enables invocation logging. Every event signaled,
command called, and task posted by mote 0 (and only mote 0) will be
written out to a TCP socket. These writes are synchronous; after
writing out a message, the simulator will block until it reads a
single byte in reply (the byte itself is unimportant). The invocation
logging data is provided on port 10583. Using telnet to connect to
port 10583 allows a user to slowly read the output; the return or
enter key can be used to send 2-byte replies, causing the simulator to
advance.

The {\tt -p} option allows simulation text output to be more easily
understood. If it is specified, then the simulation pauses at every
system clock (not radio clock) event for the specified number of
seconds. Since system clock events are often major instigators of mote
behavior, this allows a user to make the simulation output be broken
up enough to be read, instead of the standard (far too fast to read)
constant stream.

The {\tt -kb} options specifies the radio throughput. The default is
10 Kb/s (rene motes). It can be set to 10, 25, or 50 Kb/s. One can
therefore simulate a network as if it were using a high-speed
mica-specific stack; one still uses the 10 Kb stack and sets its
throughput to 50 Kb/s.

The {\tt -e} option is for named EEPROM files. If {\tt -e} isn't
specified, the logger component stores and reads data, but this data
is not persistent across simulator invocations: it uses an anonymous
file. If {\tt -e} is specified, the simulator uses the specified
file. This allows logger data to be persistent across invocations of
the simulator, and also allows algorithms based on logger data to be
easily compared. The file grows as needed, and is initially a sparse
file (if mote 1000 writes a single byte to its last EEPROM line in an
otherwise empty EEPROM, only one disk block is allocated). The TOSSIM
logger component resembles the mica hardware; it is 512KB in size.

The {\tt num\_nodes} option specifies how many nodes should be
simulated. A compile-time upper bound ({\tt TOSNODES}) is specified in
{\tt TOSSIM.h}. The standard TinyOS distribution sets this value to be
1000.

\section*{dbg}

TOSSIM provides configuration of debugging output at run-time. We have
added debugging statements to most TinyOS system components. Every
debugging statement is accompanied by one or more modal flags. When
the simulator starts, it reads in the DBG environment variable to determine
which modes should be enabled. Modes are stored and processed as
entries in a bit-mask, so a single output can be enabled for multiple
modes, and a user can specify multiple modes to be displayed. For
example, the statement


\begin{verbatim}
dbg(DBG_CRC, ("crc check failed: \%x, \%x\n", crc, mcrc));
\end{verbatim}


will print out the failed CRC check message if the user has enabled
CRC debug messages, while


\begin{verbatim}
dbg(DBG_BOOT, ("AM Module initialized\n"));
\end{verbatim}


will be printed to standard out if boot messages are enabled. DBG
flags can also be combined, as in this example:


\begin{verbatim}
dbg(DBG_ROUTE|DBG_ERROR, ("Received control message: \
                           lose our network name!.\n"));
\end{verbatim}


This will print out if either route or error messages are enabled.

There are a set of bindings from ASCII strings to debug modes. For
example, setting DBG to {\tt boot} will enable boot message (by
setting the appropriate bit in the mask). Setting DBG to {\tt
boot,route,am} will enable boot, routing and active message output.

This system allows a user to reconfigure the type of output displayed
on each run of the simulator without needing to modify source files
and recompile. There are several modes reserved for applications or
temporary testing use, which can be safely used when writing new
components. There are also modes that provide output on the internals
of the simulator, such as memory allocation or the event queue. When
TinyOS is compiled for mote hardware, all of the debug statements are
removed through redefinition of the macro; we have verified that they
add no instructions to the TinyOS code image.


\section*{The GUI}

The TOSSIM visualization GUI is included in the default TinyOS release
as a Java application. Currently, the simulation only supports radio
packet messages (incoming and outgoing). It resides in the {\tt
tools/} directory of the default TinyOS release, and can be run by
typing {\tt java TossimGUI} when in that directory. It has a single
optional parameter, {\tt -r}, for specifying a file of radio messages
to introduce to the simulation.

\subsection*{File Format}

TossimGUI allows a file of radio packets to be specified at the
command line. These packets are read in and then sent over port
10576. The file has the following format:

\vspace{0.1in}
$<$time (decimal long long)$>$ $<$mote (decimal short) $<$data0 (hex byte)$>$ ... $<$data36 (hex byte)$>$.
\vspace{0.1in}

Each value is separated from the other by whitespace. All whitespace
is insignificant. A sample file exists at {\tt tools/radio.txt}.

\section*{TOSSIM Internals}

Currently, TOSSIM compiles by using alternative implementations of
some core system components (e.g. {\tt MAIN.c}, {\tt CLOCK.c}) and
incorporating a few extra files (e.g. {\tt event\_queue.c}). Also,
TinyOS macros such as {\tt VAR} are provided alternative definitions
in {\tt tos.h}.

The basic TOSSIM structure definitions exist in {\tt
system/tossim/tossim.h}. These include the global simulation state
(event queue, time, etc.) and individual node state. Hardware settings
that affect multiple components (such as the potentiometer setting)
currently must be modeled as global node state; the {\tt static}
nature of TinyOS frames prevents otherwise. {\tt tossim.h} also
includes a few useful macros, such as {\tt NODE\_NUM} and {\tt
THIS\_NODE}; these should be used whenever possible, so minimal code
modifications will be necessary if the data representation
changes. They will probably soon be transitioned to functions for this
reason.

{\tt MAIN.c} is where the TOSSIM global state is defined. The {\tt
main()} function parses the command line options, initializes the
TOSSIM global state, initializes the external communication
functionality (defined in {\tt external\_comm.c}), then initializes the
state of each of the simulated motes with {\tt MAIN\_SUB\_INIT} and {\tt
MAIN\_SUB\_START}. It then starts executing a small loop that handles
events and schedules tasks. Currently, tasks are un-interruptible and
execute instantaneously (in simulation time). We hope to model tasks
more accurately soon.

TOSSIM catches SIGINT (control-C) to exit cleanly. This is useful when
profiling code.

\subsection*{TOSSIM Architecture}

TOSSIM replaces a small number of TinyOS components, the components
that handle interrupts and the {\tt MAIN} component. Interrupts are
modeled as discrete events. Normally, the core TinyOS loop that runs
on motes is this:


\begin{verbatim}
while(1){
  while(!TOS_schedule_task()) { };
  sbi(MCUCR, SE);
  asm volatile ("sleep" ::);
  asm volatile ("nop" ::);
  asm volatile ("nop" ::);
}
\end{verbatim}


If there are no tasks to run, the mote sleeps. An interrupt will wake
the mote from sleep. If the interrupt has caused a task to be
scheduled, then that task will be run. While that task is running,
interrupts can be handled.

The core TOSSIM loop is slightly different:


\begin{verbatim}
while(!queue_is_empty(&(tos_state.queue))) {
  while(!TOS_schedule_task()) {}

  tos_state.tos_time =
    queue_peek_event_time(&(tos_state.queue));
  queue_handle_next_event(&(tos_state.queue));
}
\end{verbatim}



A notion of virtual time (stored as a 64-bit integer) is kept in the
simulator (stored in {\tt tos\_state.tos\_time}), and every event is
associated with a specific mote. Most events are emulations of
hardware interrupts. For example, when the CLOCK component is
initialized to certain counter values, TOSSIM's implementation
enqueues a clock interrupt event, whose handler calls the function
normally registered as the clock interrupt handler; from this point,
normal TinyOS code takes over (in terms of events being signaled,
etc.) In addition, the event enqueues another clock event for the
future, with its time (for the event queue) being the current time
plus the time between interrupts.

TOSSIM redefines several of the TinyOS programming macros to account
for a large number of motes. For example, the {\tt TOS\_FRAME} macro
produces not a single structure, but an array of them. The {\tt VAR}
macro (used to access the fields of a frame) indexes into the array
using the {\tt current\_node} field of the global {\tt tos\_state}
structure. The maximum number of motes that can be simulated at once
is set at compile time by the size of this array; the default value is
1024. Since every event is associated with a specific mote, this field
is set just before the event handler function is called, and all
modifications to frame variables affect the correct mote's
state. Since all TinyOS tasks enqueued by an event are run to
completion before the next event is handled, the tasks all execute
with the same node number and access the enqueuing mote's state.

\subsection*{RFM}

Instead of requiring a certain level of radio simulation accuracy,
TOSSIM uses an extensible radio model. By doing so, the simulation can
be run with any level of accuracy a user wants (and is willing to
write). While simple protocol correctness tests might be satisfied
with a model that assumes perfect signal transmission, protocol
performance tuning tests might want to introduce bit error and
transient interference.

The interface to the radio model is defined in {\tt rfm\_model.h}:

\begin{verbatim}
typedef struct {
  void(*init)();
  void(*transmit)(int, char);
  void(*stop\_transmit)(int);
  char(*hears)(int);
  void* data;
} rfm_model;
\end{verbatim}

The data pointer is provided so that radio models that require a large
amount of state can only allocate the memory if they're being used;
otherwise, if the state is comprised of global variables in source
files, the process uses up space for every model's bookkeeping.

The {\tt int} arguments are mote IDs, while the {\tt char} arguments
are bits (0 or 1).

{\tt init} is called at simulation startup. {\tt transmit} is called
whenever a mote calls the transmit command on the lowest level radio
abstraction. {\tt stop\_transmit} is called on every radio interrupt
to clear status if the mote was previously transmitting. {\tt hears}
is called on every radio interrupt when the radio is in receive state,
and returns the value that is heard. The {\tt data} pointer is
provided so that large data structures can be allocated dynamically at
{\tt init}; otherwise, if the simulation includes many models, a lot
of memory could be wasted.

We have implemented several radio models; we will describe a simple
one that we call {\bf static}. When the {\tt init} method of the
static model is called, the simulator tries to open a file named {\tt
cells.txt} that defines an undirected network connectivity graph in
terms of mote IDs. {\tt init} reads and builds this graph. If no file
is found, every mote is made adjacent to every other. The model stores
two pieces of state for every mote: whether it is currently {\bf
transmitting }a one, and a {\bf radio\_active} value. Both values are
initialized to zero in {\tt init}.

When a mote transmits a 1, its {\bf transmitting} is set to true, and
{\bf radio\_active} of every mote adjacent to it in the graph is
incremented. If {\bf transmitting} is true when {\tt stop\_transmit}
is called, {\bf radio\_active} of every adjacent mote is
decremented. When a mote {\tt hears}, it tests {\bf radio\_active}; if
greater than zero, the mote hears a bit. This assumes that two motes
transmitting at the same time will never cancel each-other's
signals. The increment/decrement element is necessary to determine
when the last of the transmitters a mote can hear has finished.

The static model has many shortcomings; it assumes perfect signal
propagation (there are no bit errors) and the connectivity graph
doesn't change (hence the name). We have also implemented a simplistic
spatial model that handles mote movement and transmit distances; motes
begin at random positions and move in a random walk
manner. Connectivity for a mote is recomputed at each radio clock
event. Our intention is not to provide realistic world models that
account for the complexities of RF propagation, instead, to provide a
framework for users to develop models of complexity as fine or coarse
as their work needs; we believe the models we have developed show the
feasibility of more realistic models.

The only radio stack component that required modification for radio
simulation was RFM, the bit-level hardware interface. All other
components in the radio stack could remain unchanged. We added a hooks
in higher level components for the external communication system, but
their logic is unchanged.

\section*{ADC and Spatial Models}

TOSSIM also has extensible ADC and spatial models. The provided
implementations of these models return identical data on every call;
every ADC port has a value of zero, and all motes are always at the
same coordinates. This is the interface to a spatial model:

\begin{verbatim}
typedef struct {
  double xCoordinate;
  double yCoordinate;
  double zCoordinate;
} point3D;

typedef struct {
  void(*init)();
  void (*get_position)(int, long long, point3D*);   // int moteID,
                                                    // long long time
                                                    // point3D* storage
  void* data;                
} spatial_model;
\end{verbatim}

This is the interface to an ADC model:

\begin{verbatim}
typedef struct {
  void(*init)();
  short (*read)(int, char, long long);  // int mote, char port, long long time
  void* data;                
} adc_model;
\end{verbatim}

The global {\tt tos\_state} structure holds a pointer to all of the
three models: rfm, spatial and adc.

The spatial model is separate from the other two so that they can
easily share state and can provide correlated readings.

\section*{External Communication}

IO is a major limitation in debugging sensor networks. If a network
has a complex topology (e.g. motes can be several network hops from
one another), gathering real-time information on network traffic can
be difficult. For example, a single mote might be producing packets
with corrupted application data due to a bug. These corrupted packets
could then have a ripple effect through the network. Neither of the
two options to deal with this kind of bug are very appealing.

One can incorporate network traffic logging on the motes, storing
packets for later retrieval. The difficulty arises when correlating
the traffic observed by multiple motes; doing so requires global time
synchronization, which is not incorporated into the motes by default,
and adding it might make the bug disappear, just as adding logging might. 

Alternatively, one could add a number of motes that solely listen and
log traffic. These monitor motes could be connected to PCs, allowing
real-time observation of traffic. However, if the network is composed
of many transmit cells (many hops), one must use a large number of
monitor motes, or at least move them around.

Instead, TOSSIM allows a global view of the network that is distinct
from what the motes hear. It sends data on network and UART traffic
over TCP sockets, including AM-level packet transmissions as well as
bit transmissions. The former is useful for application debugging,
while the latter can be used to debug data link level encoding and
decoding. When TOSSIM starts, it attempts to connect to ports on the
local machine, each of which is associated with a certain kind of IO
data. If a connection attempts fails, TOSSIM does not retry to connect
and doesn't output that data during its run.

TOSSIM also opens a server socket that runs in a separate
thread. Connections made to this socket allow a user to dynamically
inject packets into the network; each packet has a time stamp and a
mote ID; that mote will receive the packet at that time, bypassing all
of the protocol stack below the AM component (an event is enqueued
whose handler signals the event that tells AM a packet has been
received). If the timestamp is set for a point in the past, the packet
is received at the current time in the simulator (the time of the next
event in the queue).

In addition, TOSSIM has a command-line option that allows events,
commands, and tasks enqueued to be monitored. If the {\tt -l} option
is specified, TOSSIM waits for a TCP connection request on a specified
port before it commences the mote boot sequence -- this synchronous
behavior ensures that the commands issued at boot (a common source of
bugs) will not be missed. The {\tt CALL\_COMMAND}, {\tt SIGNAL\_EVENT}
and {\tt POST\_TASK} macros are redefined to also log each call for
mote 0. Connecting to the simulator via telnet yields output such as
this:


\begin{verbatim}
452497954 CLOCK_FIRE_EVENT
452497954 COUNTER_OUTPUT
452497954 INT_TO_LEDS_LEDy_off
452497954 INT_TO_LEDS_LEDg_off
452497954 INT_TO_LEDS_LEDr_off
452997922 CLOCK_FIRE_EVENT
452997922 COUNTER_OUTPUT
452997922 INT_TO_LEDS_LEDy_on
452997922 INT_TO_LEDS_LEDg_off
452997922 INT_TO_LEDS_LEDr_off
\end{verbatim}


This output is from {\tt cnt\_to\_leds}. The yellow LED is the lowest
bit of the counter. The first value displayed is zero, while the
second value displayed is 1. The blinks are 500,000 simulator time
ticks apart, which means that in this program, the clock interrupt has
been configured to interrupt at 4 Hz (as there are two million ticks
in a second).


TOSSIM includes functionality for introducing and reading network
traffic. This is achieved through the use of TCP sockets that are
opened before the simulation begins. TOSSIM initiates a connection to
localhost on a series of ports. Each of these ports, their use, and
their communication format are detailed below. In addition, TOSSIM
opens a single server (passive) port and waits for external
connections; client sockets connecting to this port can inject
messages into the network.

\subsection*{UART}

\subsubsection*{Reading: Port 10580}

The UART communication channel allows an external process to
communicate with the motes over their UARTs. At startup, the TOSSIM
process reads from the socket until the server side closes it. Data
sent over the socket must have the following format:

\vspace{0.1in}
\begin{tabular}{|c|c|c|}\hline
\hspace{4in} & \hspace{1in} & \hspace{0.5in} \\ \hline
Time (long long)& Mote ID (short) & Data(char) \\ \hline
\end{tabular}
\vspace{0.1in}

These messages cannot be generated dynamically; they are all read in
before the simulation starts. Once the server process has sent all of
its messages, it closes the connection to TOSSIM.

{\bf UART communication is currently not implemented.}

\subsubsection*{Writing: Port 10581}

UART messages are written using the same format with which they are
read. On startup, the simulator opens a connection to port 22577 on
localhost and send messages over this channel.

{\bf UART communication is currently not implemented.}

\subsection*{Radio}
\subsubsection*{Reading: Port 10576}

The radio communication channel allows an external process to
communicate with the motes over the radio. At startup, the TOSSIM
process reads from the socket until the server side closes it. Data
sent over the socket must have the following format:

\vspace{0.1in}
\begin{tabular}{|c|c|c|}\hline
\hspace{2in} & \hspace{2in} & \hspace{0.5in} \\ \hline
Time (long long)& Mote ID(short) & TOS\_Msg (36 bytes) \\ \hline
\end{tabular}
\vspace{0.1in}

These messages cannot be generated dynamically; they are all read in
before the simulation starts. The simulation stores them as
events. Sending messages in this manner bypasses the radio layers in
the mote, directly calling {\tt AM\_RX\_PACKET\_DONE}.

Once the server has sent all of its messages, it closes the connection
to TOSSIM.

This mechanism allows for the initialization of base stations at
simulation startup. Since all motes must run the same code image, base
station logic must be incorporated into the build of TinyOS. A message
handler can be used to receive messages that switch a mote from
standard to base-station mode, and messages of this type can be
introduced into the simulator through this mechanism.

\subsubsection*{Writing: Port 10577}

This socket notifies an external program whenever a mote starts to
transmit a packet. This notification occurs when the mote completes
the transmission; if there is a lot of radio traffic, this might be
significantly after it attempts to start transmitting. The packet is
sent along the socket in this format:

\vspace{0.1in}
\begin{tabular}{|c|c|c|}\hline
\hspace{2in} & \hspace{2in} & \hspace{0.5in} \\ \hline
Time (long long)& Mote ID(short) & Data (36 bytes) \\ \hline
\end{tabular}
\vspace{0.1in}

\subsubsection*{Bit Writing: Port 10578}

Radio messages are written to a connection made to this port. The
messages have the following format:

\vspace{0.1in}
\begin{tabular}{|c|c|c|}\hline
\hspace{4in} & \hspace{1in} & \hspace{0.5in} \\ \hline
Time (long long)& Mote ID(short) & Bit (byte) \\ \hline
\end{tabular}
\vspace{0.1in}

The bit states whether the mote transmits a 1 or doesn't transmit
(0). The mote is considered to be producing the same value until
another message saying otherwise arrives.

This functionality is useful if debugging activity below the packet
layer, such as encodings/start symbols.

\subsubsection*{Real-time Reading: Port 10579}

TOSSIM spawns a thread at startup that opens a server socket and waits
for connections on port 10579. Clients connecting to this port can
inject messages dynamically into the network. Data sent over this
socket should have the following format:

\vspace{0.1in}
\begin{tabular}{|c|c|c|}\hline
\hspace{2in} & \hspace{2in} & \hspace{0.5in} \\ \hline
Time (long long)& Mote ID(short) & TOS\_Msg (36 bytes) \\ \hline
\end{tabular}
\vspace{0.1in}

If the time field specifies a time that has already passed in the
simulation, the message is scheduled to be received immediately. If
the time field specifies a time in the future, the message is
scheduled to be received at the specified time.

If the client closes the connection, TOSSIM will wait for another
connection request.

A GUI for injecting packets is included in the TinyOS release: {\tt
net.tinyos.tossim.NetworkInjector}. It uses the {\tt
net.tinyos.packet} package to provide a packet type specific GUI. If
you want to add a new packet type, merely write a new packet class and
include it in the {\tt main} of {\tt NetworkInjector}. This GUI sends
all packets to mote 0.

\section*{cells.txt}

There are currently two network connectivity models in TOSSIM. The simple model places every mote within a single cell; all of the motes can hear one another. This is the default network model. The second network model, the static model, uses a network connectivity graph that is determined at run-time before the simulation starts. The graph is specified by a file named {\tt cells.txt} which must be in the directory where the program is executed. The file format is very simple.

\begin{verbatim}

0:1 1:2 4:5 9:1

\end{verbatim}

defines a graph in which motes 0 and 1 can communicate, 2 and 1 can communicate, 4 and 5 can communicate, and 9 and 1 can communicate. Connectivity is bidirectional: {\tt 0:1} is the same as {\tt 1:0}. A sample file can be found in the root TinyOS directory.

\end{document}
