\documentclass[10pt]{article}

\usepackage{graphicx}
\usepackage{graphics}
\usepackage{multicol}
\usepackage{epsfig,amsmath,amsfonts}

\makeatletter                                   % Make '@' accessible. 
\oddsidemargin=0in                              % Left margin minus 1 inch.
\evensidemargin=0in                             % Same for even-numbered pages.
\marginparsep=0pt                               % Space between margin & text

\renewcommand{\baselinestretch}{1}              % Double spacing

\textwidth=6.5in                                % Text width (8.5in - margins).
\textheight=9in                                 % Body height (incl. footnotes)

\topmargin=0in                                  % Top margin minus 1 inch.
\headsep=0.0in                                  % Distance from header to body.
\skip\footins=4ex                               % Space above first footnote.
\hbadness=10000                                 % No "underfull hbox" messages.
\makeatother                                    % Make '@' special again.

\def\Mate{{Mat\'{e} }}
\def\mate{{Mat\'{e}}}

\newenvironment{instr}[6]{\subsubsection*{#1}\begin{tabular}{p{5in}}Effect: #2\\Class: #3\\ Operands: #4\\ Results: #5\\Bytecode: #6\\Description: }{\\\end{tabular}}



\begin{document}

\fontfamily{cmss}                               % Make text sans serif.
\fontseries{m}                                  % Medium spacing.
\fontshape{n}                                   % Normal: not bold, etc.
\fontsize{10}{10}                               % 10pt font, 10pt line spacing 

\title{A Quick User's Guide to \Mate}
\author{Philip Levis}
\maketitle

\fontfamily{cmr}                                % Make text Roman (serif).
\fontseries{m}                                  % Medium spacing.
\fontshape{n}                                   % Normal: not bold, etc.
\fontsize{10}{10}                               % 10pt font, 10pt line spacing

\section*{Introduction}

This document is a simple overview of the \Mate virtual machine for
TinyOS. It is intended for people who want to write simple
clock-driven \Mate applications. Writing ad-hoc routing algorithms in
\Mate is beyond the scope of this document. This document explains how
to put \Mate on a mote, how to write \Mate programs, how to upload
those programs onto a \Mate mote, and gives a brief description of
some basic \Mate instructions.

\section*{\Mate Overview}

\Mate is a bytecode interpreter that runs on TinyOS. It presents a
virtual stack-based architecture to programs. Code is contained in
capsules, which are at most 24 instructions and fit in a single TinyOS
packet. There are capsules that are run in response to an event
(e.g. the clock timer) and capsules that can be called from other
capsules (subroutines).

The complete \Mate instruction set can be found in {\tt
nest/tos/include/tos-vm2.h}. The complete \Mate source can be found in
{\tt nest/tos/system/VM2.\{c,comp\}}. \Mate has functionality beyond
what is described in this document. A Java \Mate assembler is included
in the current TinyOS release: {\tt net.tinyos.asm.Assembler.java}. It
also has a GUI interface that sends capsules out over the serial port:
{\tt net.tinyos.vmGUI.MainWindow}. {\tt MainWindow} uses COM2 by
default -- changing this requires changing the source file. By
connecting a {\tt generic\_base} mote to the serial port, you can send
\Mate capsules.

\section*{\Mate Architecture and Execution Model}

\Mate is a two-stack architecture: each execution context has an
operand and a return address stack. The former is used for data
operations, while the latter is used to keep track of subroutine
invocations. The operand stack has a maximum depth of 16, while the
return address stack as a maximum depth of 8. There is also a single
variable heap that persists across all executions. There are three
types of variables: values, sensor readings, and messages.

All \Mate code exists in capsules, which hold up to 24
instructions. Every instruction is a single byte long. Every capsule
has a numeric ID; capsules 0-3 are the four subroutine capsules, while
capsule 4 is the clock capsule that is called in response to a clock
event. Every capsule also has a version number, which \Mate uses to
determine whether it should install the capsule or not.

\Mate processes data by pushing operands onto its operand stack then
executing instructions which consume them. For example, the program

\begin{verbatim}
pushc 1
pushc 2
add
halt
\end{verbatim}

will result in a 3 being on the top of the operand stack. Besides
pushing constants with {\tt pushc}, sensor readings can be read with
{\tt sense} and the message buffer can be pushed onto the stack with
{\tt pushm}.

The clock context's operand stack persists across invocations. For
example, if the above program were the clock capsule, then each
invocation would push 3 onto the operand stack until it reached its
maximum depth of 16, after which pushed values would be ignored. When
a new clock capsule is installed by \mate, the clock operand stack is
cleared and a single value variable of zero is pushed onto it -- this
provides an easy way to keep track of a counter. The clock context is
also reset in this manner whenever a new subroutine capsule is
installed.

If a clock event goes off while the context is still executing, then
\Mate clears the return address stack and forces the context to jump
to the first instruction of the clock capsule. This prevents a context
from entering an infinite loop. The rate at which clock events arrive
can be manipulated with the {\tt clockc} and {\tt clockf}
instructions.

Subroutine capsules are invoked with the {\tt call} instruction, and
should return with the {\tt return} instruction. Call consumes a
single value operand, which indicates which subroutine to call. If a
subroutine issues the halt instruction the context running it halts.

Branches are made with the {\tt blez} instruction, which branches if
the operand on top of the stack (which is consumed) is less than or
equal to zero. A message operand will always cause {\tt blez} to
branch. Like {\tt pushc}, blez has an embedded operand stating which
instruction (absolute, not relative) to jump to. For example

\begin{verbatim}
pushc 0
blez 3
pushc 1
halt
\end{verbatim}

will never push 1 on the operand stack.

All capsules must have an explicit halt at their end.

\section*{\Mate Capsules and Capsule Propagation}

\Mate capsules can be written with the {\tt MainWindow} program. On
the right is a small assembler window where programs can be
entered. Pressing ``Create'' will generate a capsule with that
program. Each capsule written is placed in the payload of a new
AMPacket -- one can flip through the generated capsules with the
tabbed pane. The first two byes of the AM payload are the capsule's
meta-information -- it's capsule type (byte 0) and version (byte
1). Types 0-3 are the four subroutines, while type 4 is the clock
capsule.

\Mate will only install capsules that have a more recent version
number than the capsule currently running. More recent is defined to
either have a version number which is up to 64 greater or more than
128 less (for 8 bit wrap-around).

For example, if a \Mate mote is running a clock capsule with version
23, then after writing a program and clicking the create button, one
would have to type 4 in the first byte of the packet then 24 in the
second before sending it -- otherwise it will possibly be installed as
the wrong type of capsule or ignored because of its version.

Sending capsules from a generic base is not very useful when there's a
very wide area sensor network. Therefore, capsules can forward
themselves. When a capsule issues the {\tt forw} instruction, it
broadcasts a packet containing itself; motes that hear it may then
install the capsule. Capsules can also issue the {\tt forwo}
instruction to forward an arbitrary capsule -- {\tt forwo} consumes a
value single operand off the stack to determine which capsule to
forward (0-4). For example,

\begin{verbatim}
pushc 2
forwo
\end{verbatim}

will forward subroutine capsule 2.

\section*{\Mate Message Buffers}

\Mate assumes that an ad-hoc routing algorithm will use eight bytes of
header in a 30-byte AM payload. Therefore, the message data buffer is
22 bytes long. One byte of this is used to store how many values are
in the payload, and a second byte is used to give the lower 8 bits of
the sender's ID (it would have been padding otherwise). The remaining
20 bytes contain up to 5 variables, each of which takes up 4
bytes. One byte specifies what kind of variable it is (value or sensor
reading); if it's a sensor reading, the second byte indicates which
type. The third and fourth bytes store a 16-bit data value.

\vspace{0.2cm}

\begin{tabular}{|p{1in}|p{1in}|p{1in}|p{1in}|}\hline 
Header0 & Header1 & Header2 & Header3 \\ \hline
Header4 & Header5 & Header6 & Header7 \\ \hline
Num    &  ID    & Type0   & Sensor Type0? \\ \hline
\multicolumn{2}{|c|}{Data Value0} & Type1 & Sensor Type1? \\ \hline
\multicolumn{2}{|c|}{Data Value1} & Type2 & Sensor Type2? \\ \hline
\multicolumn{2}{|c|}{Data Value2} & Type3 & Sensor Type3? \\ \hline
\multicolumn{2}{|c|}{Data Value3} & Type4 & Sensor Type4? \\ \hline
\multicolumn{2}{|c|}{Data Value4} & \multicolumn{2}{|c|}{}\\\hline
\end{tabular}
\section*{\Mate Instructions}

The \Mate instruction set has three classes of instructions: basic,
s-class, and x-class. The s-class instructions are beyond the scope of
this document -- they only apply to ad-hoc networking. Basic
instructions and x-class differ only in that x-class instructions
({\tt blez} and {\tt pushc}) have embedded operands. Basic
instructions obtain all of their operands from the operand stack.

\begin{instr}{Sample Instruction}{Brief description of instruction}{instruction class (basic or x-class)}{Input operands -- first operand is top of stack, next is 2nd on stack, etc. Type (value, sense, msg, any) is specified.}{Output operands -- first operand is top of stack, etc. Type is specified.}{Hexidecimal opcode}
Detailed description.
\end{instr}

\subsection*{Production Instructions}

\begin{instr}{pushc}{Push a constant on the operand stack}{basic}{none}{value}{0xC0-0xff}
When this instruction is issued, its embedded six bit operand is
pushed onto the operand stack as an unsigned value variable.
\end{instr}

\begin{instr}{sense}{Read a sensor}{basic}{value}{sensor}{0x0D}
The operand specifies which sensor. Currently, only two are supported:
sensor 1 is light and sensor 2 is temperature.
\end{instr}

\begin{instr}{id}{Push mote's ID on the operand stack}{basic}{none}{value}{0x09}
\end{instr}

\begin{instr}{rand}{Push a 16-bit random value on the operand stack}{basic}{none}{value}{0x1c}
\end{instr}

\begin{instr}{pushm}{Push context's message buffer}{basic}{none}{msg}{0x11}
The context has a single message buffer, which is placed on the
operand stack with this instruction. Issuing this instruction multiple
times will place the same buffer on the operand stack multiple times.
\end{instr}

\subsection*{Data Instructions}
\begin{instr}{and}{Binary and}{basic}{value1,value2,...}{value}{0x02}
Push (value1 \& value2).
\end{instr}

\begin{instr}{or}{Binary or}{basic}{value1,value2,...}{value}{0x03}
Push (value1 $\mid$ value2).
\end{instr}

\begin{instr}{shiftr}{Bitshift variable right}{basic}{value1,value2,...}{value}{0x04}
Shift the second operand right a number of bits specified by the first
operand and push the result on the stack.
\end{instr}

\begin{instr}{shiftl}{Bitshift variable left}{basic}{value1,value2,...}{value}{0x05}
Shift the second operand left a number of bits specified by the first
operand and push the result on the stack.
\end{instr}

\begin{instr}{add}{Add two variables (polymorphic)}{basic}{any,any,...}{any}{0x06}
The add instruction operates on all three types. Two values added will
result in a value. A value added to a sensor will result in a sensor
reading adjusted by the value. Two sensor readings of the same type
can be added, but different types (e.g. light, temperature)
cannot. Adding a value or a sensor reading to a message buffer will
append the variable if there is space, do nothing if there is not, and
result in a buffer. Adding two message buffers will append the
variables of the second onto the first and result in a buffer.
\end{instr}

\begin{instr}{inv}{Invert top of operand stack}{basic}{value}{value}{0x0a}
Push the operand multiplied by -1.
\end{instr}

\begin{instr}{not}{Binary not}{basic}{value1,value2,...}{value}{0x16}
Push \verb;~;(value1).
\end{instr}

\subsection*{Control Instructions}

\begin{instr}{call}{Call subroutine}{basic}{value}{none}{0x1f}
Call subroutine specified by operand. Must be 0-4.
\end{instr}

\begin{instr}{return}{Return from subroutine}{basic}{none}{none}{0x3f}
Return from subroutine to execute instruction after call.
\end{instr}

\begin{instr}{halt}{Halt execution}{basic}{none}{none}{0x00}
Halt execution of the context.
\end{instr}

\begin{instr}{blez}{Branch if less than or equal to zero}{x-class}{value}{none}{0x80-0xBF}
Branch to absolute instruction address specified by embedded operand
if value on top of operand stack is less then or equal to zero.
\end{instr}

\subsection*{Code Propagation Instructions}

\begin{instr}{forw}{Forward current capsule}{basic}{none}{none}{0x00}
Broadcast the current capsule for others to possibly install.
\end{instr}

\begin{instr}{forwo}{Forward other capsule}{basic}{value}{none}{0x00}
Broadcast capsule specified by operand for others to possibly
install. Value operand must be 0-4.
\end{instr}

\subsection*{Other Instructions}

\begin{instr}{putled}{Control LEDs}{basic}{value}{none}{0x09}
{\tt putled} takes a five bit operand. The bottom three bits specify
the LEDS: red, yellow, and green. The fourth and fifth bits specify
the operation to perform. 0 sets the LEDs; 1 turns the specified LEDs
on (leaving the others alone); 2 turns the specified LEDS off (leaving
the others alone); 3 toggles the specified LEDS. For example, 0x02
will turn on off the red and green LEDS and turn on the yellow, 0x19
will toggle the red led, and 0x13 will turn off the yellow and red
leds while leaving the green one alone.
\end{instr}

\begin{instr}{send}{Send packet}{basic}{msg}{none}{0x09}
Send the data contained in the message buffer with \mate's built-in
ad-hoc routing component. By default, this is BLESS. You must
therefore have a BLESS base station for packets to be sent (otherwise
BLESS will not be able to find a route). You can redefine the ad-hoc
routing algorithm used by changing the corresponding application {\tt
.desc} file.
\end{instr}

\begin{instr}{clear}{Clear variable}{basic}{any}{operand}{0x13}
Clear the variable on top of the operand stack, leaving it there. This
is especially useful for the message buffer, as you want to clear out
variables that have been previously put in it, as it persists across
invocations. Issuing clear on a sensor reading or a value will set it
to be zero.
\end{instr}

\begin{instr}{log}{Write variable to non-volatile storage}{basic}{any}{none}{0x09}
If the operand is a message buffer with 5 variables in it, only the
first four are written. \Mate buffers variables until it has enough
for a full log line (4 variables), then writes it to EEPROM. Therefore, if a program logs three sensor values, they will not be written to EEPROM until a fourth variable is logged.
\end{instr}

\begin{instr}{logr}{Read log line into buffer}{basic}{msg,value1,...}{msg}{0x09}
Read log line specified by value1 into the message buffer.
\end{instr}

\begin{instr}{sets}{Set shared variable}{basic}{any}{none}{0x09}
\end{instr}

\begin{instr}{gets}{Get shared variable}{basic}{none}{any}{0x09}
\end{instr}

\begin{instr}{swap}{Swap top two operands on stack}{basic}{any1,any2,...}{any2,any1,...}{0x09}
\end{instr}

\begin{instr}{clockc}{Set clock counter for capsule invocation}{basic}{value}{none}{0x3f}
{\tt clockc} sets how many clock ticks must occur before the clock
context is triggered. For example, if the clock tick frequency is once
per second and the clock counter is 4, the clock capsule will run
every 4 seconds. If the clock tick frequency were changed to eight per
second (with {\tt clockf}), the capsule would start running twice per
second.
\end{instr}

\begin{instr}{clockf}{Set clock tick frequency in $\frac{1}{32}$s of seconds}{basic}{value}{none}{0x3e}
{\tt clockf} sets how often the \Mate clock ticks. The number of ticks
that must occur for the clock context to run is specified by {\tt
clockc}. The operand for {\tt clockf} must be within the range 1-255,
and represents the frequency of the clock ticks in thirty-seconds of a
second. For example, {\tt clockf} with a parameter of 160 will make
the \Mate clock tick once every five seconds.
\end{instr}

\section*{Sample \Mate Programs}

\subsection*{{\tt cnt_to_leds}}
Clock capsule:
\begin{verbatim}
pushc 1
add      # Add 1 to counter on top of stack
copy
pushc 7
and      # Take bottom 3 bits of copy
putled   # Set LEDs to bits
halt
\end{verbatim}

\subsection*{{\tt cnt_to_leds}, where increment is in a subroutine}
Clock capsule:
\begin{verbatim}
pushc 0
call
copy
pushc 7
and      # Take bottom 3 bits of copy
putled   # Set LEDs to bits
halt
\end{verbatim}

Subroutine 0:
\begin{verbatim}
pushc 1
add
copy
\end{verbatim}

\subsection*{{\tt sens_to_leds}}
Clock capsule:
\begin{verbatim}
pushc 1
sense          # Read light value
push 7
shiftr         # Only want top 3 bits of reading
putled
halt
\end{verbatim}

\subsection*{Sending a Sensor Reading in a Packet}
Clock capsule:
\begin{verbatim}
pushc 1
sense
pushm
clear        # Clear out the message buffer
add          # Put reading in message buffer
send         # Send buffer
halt
\end{verbatim}

\subsection*{Saving 4 readings on operand stack, averaging them, sending}
Clock capsule:
\begin{verbatim}
pushc 1
sense       # Push a light reading on top of operand stack
swap        # Keep counter on top of operand stack
pushc 1
add         # Increment counter
copy
inv
pushc 4
add
blez 11     # If counter >= 4, jump past halt
halt
pop         # Jump-to point; pop counter
add
add
add         # Add up 4 readings
pushc 2
shiftr      # Divide by 4 (bitshift 2 right
pushm
clear       # Clear out a message buffer
add         # Put averaged reading in buffer
send        # Send packet
pushc 0     # Reset counter to 0
halt
\end{verbatim}

\end{document}
