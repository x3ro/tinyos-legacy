\documentclass[12pt]{article}

\usepackage{graphicx}
\usepackage{graphics}
\usepackage{multicol}
\usepackage{epsfig,amsmath,amsfonts}

\makeatletter                                   % Make '@' accessible. 
\oddsidemargin=0in                              % Left margin minus 1 inch.
\evensidemargin=0in                             % Same for even-numbered pages.
\marginparsep=0pt                               % Space between margin & text

\renewcommand{\baselinestretch}{1}              % Double spacing

\textwidth=6.5in                                % Text width (8.5in - margins).
\textheight=9in                                 % Body height (incl. footnotes)

\topmargin=0in                                  % Top margin minus 1 inch.
\headsep=0.0in                                  % Distance from header to body.
\skip\footins=4ex                               % Space above first footnote.
\hbadness=10000                                 % No "underfull hbox" messages.
\makeatother                                    % Make '@' special again.


\begin{document}

\fontfamily{cmss}                               % Make text sans serif.
\fontseries{m}                                  % Medium spacing.
\fontshape{n}                                   % Normal: not bold, etc.
\fontsize{10}{10}                               % 10pt font, 10pt line spacing 

\title{TinyOS Call Graph Verification}
\author{Philip Levis}
\maketitle

\fontfamily{cmr}                                % Make text Roman (serif).
\fontseries{m}                                  % Medium spacing.
\fontshape{n}                                   % Normal: not bold, etc.
\fontsize{10}{10}                               % 10pt font, 10pt line spacing

\section*{Overview}

The TinyOS programming model constrains the call graph of a TinyOS
component: TinyOS commands cannot signal events. Originally, the
TinyOS build process assumed this rule was followed. We have
implemented a system that checks TinyOS source for this rule. The
system is very flexible and can be easily extended to include more
constraints on TinyOS source. We explain how this system works, how it
is used to check call graphs, and it could check an additional TinyOS
call graph constraint that we are considering incorporating.

\section*{{\tt cqual}}

Call graph checking is performed by a customized version of {\tt
cqual}, an application developed by Jeff Foster at UC Berkeley. {\tt
cqual} allows a user to specify new type qualifiers for C programs,
establish rules for their conversion, then check that these rules are
not violated.

For example, one can add a pair of qualifiers, {\tt \$tainted} and
{\tt \$untainted}. Some function parameters can only take {\tt
\$untainted}, or trusted data; the format string of {\tt printf()} is
an example. Arbitrary {\tt printf()} format strings
(e.g. user-provided) are a known security problem. Some functions
produced {\tt \$tainted}, or untrusted data; an example is {\tt
getenv()}, as it's a user specified string. One can define a rule that
{\tt \$untainted} data can be converted to {\tt \$tainted}, but not
vice-versa. This rule is written by placing both qualifiers in a type
hierarchy: {\tt \$untainted} is a subtype of {\tt \$tainted}. If a
variable is untainted, one can also considered it tainted, just as
Volvo can be considered a car; however, a tainted variable is not
necessarily untainted, just as a car is not necessarily a Volvo.

If one annotates functions properly, {\tt cqual} can
detect when the result of a {\tt getenv()} is passed as the format
string to {\tt printf()}, and throw an error accordingly.

For a more in-depth explanation of description of the {\tt cqual}
system, please refer to its web site: {\tt
http://www.cs.berkeley.edu/Research/Aiken/cqual/}. Please note that we
have modified {\tt cqual} for TinyOS, and so bug reports or issues
should be sent to the TinyOS team, not Jeff Foster, the {\tt cqual}
maintainer.

\section*{Modifications}

We have modified {\tt cqual} to also apply qualifier rules on
procedure invocation. A procedure can only invoke procedures of the
same type or supertype. By specifying an event as a subtype of
command, the TinyOS call graph constraints are enforced; an event can
invoke commands (as command is its supertype), but commands cannot
signal events (as they are a subtype).

We have also removed type checking from {\tt cqual}; if there are type
errors, they will be discovered and dealt with during compilation.

\section*{Qualifier Lattice}

The TinyOS qualifier lattice is this:

\begin{verbatim}
partial order {
$aevent [level=value, color = "pam-color-tainted", sign = neg]
$sevent [level=value, color = "pam-color-tainted", sign = neg]
$command [level=value, color = "pam-color-untainted", sign = pos]

$aevent < $command
$sevent < $command
}
\end{verbatim}

Refer to the cqual documentation for an in-depth description of the
file format. The important part is that three qualifiers are defined:
aevent, sevent, and command. Currently, only aevent and command are
used, to represent events and commands. The difference between aevents
and sevents is described in the final section of this document, in
regards to more fine-grained type qualification system we are
considering that will hopefully reduce race conditions in TinyOS code.

Both event qualifiers are subtypes of the command qualifier. This
allows both events to call commands (their supertype), while commands
cannot signal events and the two types of events cannot signal one
another.

{\tt cqual} propagates qualifiers. For example, if one defines a
simple C helper function that signals an event, then it will be given
the event qualifier can cannot be called by commands.

\section*{Checking Process}

Makefile.qual checks a TinyOS application call graph through several
steps. Executing Makefile.qual builds the file lists needed for the
application, storing them in a separate file. It then executes
Makefile.check, which imports this file and uses the file
lists. Because a traditional C compiler will not understand {\tt
cqual} qualifiers, copies are made of all the necessary files, which
are then annotated with the necessary qualifiers. The steps taken are
as follows:

\begin{itemize}
\item Make a local {\tt cqual} directory to store files in.
\item Walk the component description graph and compute which preprocessed source files will be used by {\tt cqual} in the local {\tt cqual/} directory.
\item Walk the component description graph and compute which source files are to be copied into the local {\tt cqual/} directory.
\item Walk the component description graph and compute which header files are to be copied into the local {\tt cqual/} directory.
\item Store these lists in {\tt cqual/cqual.objs}.
\item Execute {\tt Makefile.check}
\item {\tt Makefile.check} imports {\tt cqual/cqual.objs}.
\item Remove all relevant component header files from the TinyOS source directories.
\item Build new annotated copies of relevant component header files.
\item Move relevant header files into the local {\tt cqual/} directory.
\item Copy relevant source files into the local {\tt cqual/} directory.
\item Add qualifiers to function definitions in the source files in the local {\tt cqual/} directory.
\item Run the qualified source files through the C preprocessor.
\item Run {\tt cqual} on the preprocessed source files.
\end{itemize}


\section*{{\tt aevent}s and {\tt sevent}s}
Two kinds of events are defined in the TinyOS qualifier lattice: {\tt
aevent}, or asynchronous events, and {\tt sevent}s or synchronous
events. Currently, the TinyOS C annotations do not distinguish between
the two, and the call graph checker only uses {\tt aevent}s.

The lattice distinction exists because of the observation that a
TinyOS component currently has no way of knowing whether events is
handles execute asynchronously (in an interrupt context) or
synchronously (from a task). Distinguishing the two is important for
determining possible race conditions and knowing which state variables
must be protected from interrupts.

By having two separate sub-types, the task boundary between the two
can be made explicit. Because tasks are executed through function
pointers (and this modified version of cqual does not propagate
qualifiers through function pointers), a task that fires synchronous
events can be enqueued by an asynchronous event. An asynchronous event
cannot signal a synchronous event, or vice versa.

\end{document}
