%%% netprog.tex --- 

%% Author: szewczyk@cs.berkeley.edu
%% Version: $Id: netprog.tex,v 1.2 2002/01/28 19:51:29 jpolastre Exp $


\documentclass[12pt,fullpage]{article}
\newif\ifpdf
\ifx\pdfoutput\undefined
 \pdffalse
\else
 \pdfoutput=1
 \pdftrue
\fi

\ifpdf
 \usepackage[pdftex]{graphicx}
 \pdfcompresslevel=9
 \DeclareGraphicsExtensions{.pdf,.jpg }
\usepackage{thumbpdf}
\else
 \usepackage{graphics}
 \DeclareGraphicsExtensions{.eps,.jpg}
\fi
\setlength{\oddsidemargin}{0 in}
\setlength{\evensidemargin}{0 in}
\setlength{\topmargin}{-1 in}
\setlength{\textwidth}{6.5 in}
\setlength{\textheight}{9 in}
\setlength{\headsep}{0.75 in}
\setlength{\parindent}{0 in}
\setlength{\parskip}{0.1 in}
\usepackage{amsmath}
\usepackage{xspace}
\usepackage{booktabs}
\usepackage{epic}
\usepackage{eepic}

\usepackage{units}

\usepackage{hyperref}
\usepackage{graphicx}
\usepackage{color}
\usepackage{alltt}
\hypersetup{bookmarks=true, 
  bookmarksnumbered=true,
  bookmarksopen=true,
  backref=true,
  colorlinks=false,
  pdftitle={Network Programming for TinyOS},
  pdfkeywords={sensor,network,mote,active networks},
  pdfauthor={ Robert Szewczyk}}

%\ifx\pdfoutput\undefined\relax
%\else
%  \usepackage{thumbpdf}
%\fi

\newcommand{\rene}{ Ren{\'e}\xspace}
\newcommand{\wec}{ WeC\xspace}
\newcommand{\mica}{ Mica\xspace}
\newcommand{\component}[1]{{\tt #1}}
\newcommand{\application}[1]{{\tt #1}}
\newcommand{\nest} {NEST\xspace}
\makeatletter
%%\usepackage[debugshow,final]{graphics}
\title{Network Programming for TinyOS}
\author{Robert Szewczyk}


%%\revision$Header: /cvsroot/tinyos/tinyos-0.6.x/doc/tex/netprog.tex,v 1.2 2002/01/28 19:51:29 jpolastre Exp $

\begin{document}
\maketitle



%%%%##########################################################################

\section*{Network Programming Overview}

A big advantage of ``smart'' sensor networks over the more traditional sensors
is the placement of intelligence.  The traditional sensors merely report the
measurements to some data acquisition system, whereas the sensor nodes offer
local computation ability.  This feature of sensor networks is exploited only
when we have the ability to reprogram the nodes in the sensor network.  In
most application scenarios, it is either impractical or impossible to manually
reprogram each sensor; consequently it is necessary that we provide a
mechanism for reprogramming the sensors over the air.  In this document we
focus on remotely replacing the entire image of the running system.  This
approach allows us to experiment with arbitrary components of the system. 

The entire family of UCB motes has been designed with network programming in
mind.  \wec, \rene and \mica motes use a processor which cannot write to its
own flash. 

\section*{The User's View}

In this section we'll describe how a mote is rendered network reprogrammable
from the user's point of view. 

\subsection*{Mote Startup}

There are several steps one must take in order to make the mote network
programmable. 
\begin{itemize}
\item{\bf Program the ``Little Guy''.} To do that, plug the mote being
programmed into the programming board, and set the switch on the programming
board to the ``Little Guy'' position as shown in
Figure~\ref{prog_board}.  From your \nest directory, execute:
\begin{small}
\begin{verbatim}
[szewczyk@okocim nest]$ cd tos/platform/littleguy/
[szewczyk@okocim littleguy]$ make install
avr-gcc  -c -Os -Wall -I/usr/local/avr/include -I. -mmcu=at90s2343 -o I2CSPI.o
I2CSPI.c
In file included from hardware.h:45,
                 from I2CSPI.c:2:
/usr/local/avr/include/signal.h:7:2: warning: #warning "please include
sig-avr.h instead"
avr-gcc -mmcu=at90s2343 -o I2CSPI I2CSPI.o
avr-objcopy --output-target=srec I2CSPI I2CSPI.srec
sleep 1
uisp -dprog=dapa   -dno-poll --erase
Atmel AVR AT90S2343 is found.
Erasing device ...
Reinitializing device
Atmel AVR AT90S2343 is found.
sleep 1
uisp -dprog=dapa   --upload if=I2CSPI.srec
Atmel AVR AT90S2343 is found.
Uploading: flash
sleep 1
uisp -dprog=dapa   -dno-poll --verify if=I2CSPI.srec
Atmel AVR AT90S2343 is found.
Verifying: flash
\end{verbatim}
\end{small}
\item{\bf Load primordial network programming stack.} The \nest distribution
provides a sample application which contains just the programming
stack. Flip the switch on the programming board to the ``Big Guy'' position,
and load the netprog application from the distribution:
\begin{small}
\begin{verbatim}
[szewczyk@okocim nest]$ cd apps/netprog/
[szewczyk@okocim netprog]$ make rene install
Makefile:4: warning: overriding commands for target `all'
../Makeinclude:161: warning: ignoring old commands for target `all'
make: Nothing to be done for `rene'.
uisp -dprog=dapa -dno-poll  --erase
Atmel AVR AT90S8535 is found.
Erasing device ...
Reinitializing device
Atmel AVR AT90S8535 is found.
sleep 1
uisp -dprog=dapa -dno-poll  --upload if=binrene/main.srec
Atmel AVR AT90S8535 is found.
Uploading: flash
sleep 1
uisp -dprog=dapa -dno-poll  --verify if=binrene/main.srec
Atmel AVR AT90S8535 is found.
Verifying: flash
\end{verbatim}
\end{small}

Remove the mote from the programming board and power cycle it.

\item{\bf Set the mote id.} In many applications it is useful to assign a
persistent mote id to every mote. For this purpose, we reserved a piece of the
EEPROM to store per node, non-volatile data that should persist across the
reprogramming of the mote.  To do this part, you'll need a
mote with \application{generic\_base} programmed onto it, in addition to your
network programmable node. Plug in your
\application{generic\_base} into a programming board and make sure that the
whole subsystem is plugged into the serial port of your computer.  Now, that a
node is network programmable, we'll interact with it primarily over the
radio.  The {\tt CodeInjector} application will allow you to remotely set an
ID of a mote.  {\em Make sure that only your base station and the mote  being
programmed are on.} 

\begin{small}
\begin{verbatim}
[szewczyk@okocim nest]$ cd tools/
[szewczyk@okocim nest]$ make
make: Nothing to be done for `all'.
[szewczyk@okocim tools]$ java net.tinyos.SerialForwarder.SerialForward  &
[1] 3181
Initializing SerialForwarder Server 1.1
Starting in GUI mode
[szewczyk@okocim tools]$ java net.tinyos.codeGUI.CodeInjector id
\end{verbatim}
\end{small}

At this point, the yellow light on remote mote should go on, and you should
see an output which might look like this:
\begin{small}
\begin{verbatim}
7e 0 31 7d 0 0 c0 7f 0 0 0 0 0 0 0 0 0 0 0 0 0 0 0 0 0 0 0 0 85 e1 7e 0 31 7d 0
 0
Node ID: 0
Next Program ID: 0
Next Program Length: 0
\end{verbatim}
\end{small}

If you see multiple entries, that indicates that there are multiple network
programmable nodes in the vicinity; you should turn off all the nodes except
for the base station and the single remote node. Now, you can set the ID by
executing: 
\begin{small}
\begin{verbatim}
[szewczyk@okocim tools]$ java net.tinyos.codeGUI.CodeInjector setid 162
ff ff ff ff ff ff ff ff ff ff ff ff ff ff ff ff ff ff ff ff ff ff ff ff ff ff f
f ff ff ff ff ff ff ff ff ff ff ff ff ff ff ff ff ff ff ff ff ff ff ff ff ff ff
 ff ff ff ff ff ff ff ff ff ff ff
++++ff ff ff ff ff ff ff ff ff ff ff ff ff ff ff ff ff ff ff ff ff ff ff ff ff
ff ff ff ff ff ff ff ff ff ff ff ff ff ff ff ff ff ff ff ff ff ff ff ff ff ff f
f
ff ff ff ff ff ff ff ff ff ff ff ff
++++ff ff ff ff ff ff ff ff ff ff ff ff ff ff ff ff ff ff ff ff ff ff ff ff ff
ff ff ff ff ff ff ff ff ff ff ff ff ff ff ff ff ff ff ff ff ff ff ff ff ff ff f
f
ff ff ff ff ff ff ff ff ff ff ff ff
++++
\end{verbatim}
\end{small} 

The yellow LED on the remote node should be off, and the green LED should be
on. At this point, power cycle the remote, and recheck its ID:

\begin{small}
\begin{verbatim}
[szewczyk@okocim tools]$ java net.tinyos.codeGUI.CodeInjector id
.7e 0 31 7d 0 0 c0 7f a2 0 1 0 0 0 0 0 0 0 0 0 0 0 0 0 0 0 0 0 0 0 0 0 0 0 8f e
4

Node ID: 162
Next Program ID: 1
Next Program Length: 0
\end{verbatim}
\end{small}
\end{itemize}

At this point, the node is fully network programmable, and we can test this
functionality on a sample application.

\subsection*{Remote programming}

In order to keep the mote continually network reprogrammable, we must ensure
that the application loaded onto the mote contains the \component{PROG\_COMM}
component\footnote{We're actually planning to lift this restriction in the
future versions of the network programming subsystem. That will be accomplished
by periodically resetting the node with the primordial stack.}  From the
user's point of view the network programming stack looks like a
\component{GENERIC\_COMM} combined with \component{LOGGER}.  As a short
example, let's consider how to turn a simple application, like
\application{cnt\_to\_leds} into a network reprogrammable entity. 

If you look at the description file for \application{cnt\_to\_leds}, you'll
note that it does not use a communication module.  To add the
\component{PROG\_COMM} we need to add four lines to the description file (new
lines are highlighted in red):
\begin{small}
\begin{alltt}
include modules{
MAIN;
COUNTER;
INT_TO_LEDS;
CLOCK;
\textcolor{red}{PROG_COMM;}
};

MAIN:MAIN_SUB_INIT COUNTER:COUNTER_INIT
\textcolor{red}{MAIN:MAIN_SUB_INIT PROG_COMM:PROG_COMM_INIT}
MAIN:MAIN_SUB_START COUNTER:COUNTER_START
\textcolor{red}{MAIN:MAIN_SUB_START PROG_COMM:PROG_COMM_START}

COUNTER:COUNTER_CLOCK_EVENT CLOCK:CLOCK_FIRE_EVENT
COUNTER:COUNTER_SUB_CLOCK_INIT CLOCK:CLOCK_INIT

COUNTER:COUNTER_SUB_OUTPUT_INIT INT_TO_LEDS:INT_TO_LEDS_INIT
COUNTER:COUNTER_OUTPUT INT_TO_LEDS:INT_TO_LEDS_OUTPUT

\textcolor{red}{PROG_COMM:PROG_COMM_MSG_SEND_DONE MAIN:MAIN_SUB_SEND_DONE}
\end{alltt}
\end{small}
\application{cnt\_to\_leds\_prog} already incorporates these changes. Now,
lets build the binary to be downloaded to the mote: 
\begin{small}
\begin{verbatim}
[szewczyk@okocim nest]$ cd apps/cnt_to_leds_prog/
[szewczyk@okocim cnt_to_leds_prog]$ make rene binrene/main.srec
make: Nothing to be done for `rene'.
avr-objcopy --output-target=srec binrene/main.exe binrene/main.srec
\end{verbatim}
\end{small}
The next step is to send that binary to the mote. We'll assume that the
{\tt SerialForwarder} is already running. The downloading of the program to
the remote mote has 3 steps: 

\begin{itemize}
\item Inform the mote that a new program is available. That is done with 
\begin{alltt} 
java net.tinyos.codeGUI.CodeInjector new {\em srec\_file} {\em destination}
\end{alltt}
where {\em srec\_file} is the file you're about to download to the mote, and
the destination is the address of the remote mote.  In most cases you'll want
to use the broadcast destination, $-1$. Here is what you might expect to see:
\begin{small}
\begin{verbatim}
[szewczyk@okocim tools]$ java net/tinyos/codeGUI/CodeInjector new ../apps/cnt_t
o_leds_prog/binrene/main.srec -1
Program ID:5FC5
0 0 0 0 0 0 0 0 0 0 0 0 0 0 0 0 0 0 0 0 0 0 0 0 0 0 0 0 0 0 0 0 0 0 0 0 0 0 0 0
0 0 0 0 0 0 0 0 0 0 0 0 0 ff ff ff ff ff ff ff ff ff ff ff
++++
\end{verbatim}
\end{small}

At this point the remote mote should have a green LED on. Note that this
command is idempotent, so it is safe to repeat it multiple times, in fact it
is a good idea to repeat this command when programming multiple motes.
\item Broadcast the actual packets. The {\tt CodeInjector} class provides you
with several ways of doing that, depending on the actual scenario:
\begin{alltt}
java net.tinyos.codeGUI.CodeInjector write {\em srec_file} {\em destination}
\end{alltt}
will perform a single unreliable broadcast of the program contained in the
{\em srec\_file} to {\em destrination} motes. Again, you may find it useful to set
the {\em destination} to the broadcast address. 
\begin{alltt}
java net.tinyos.codeGUI.CodeInjector check {\em srec_file} {\em destination}
\end{alltt}
This invocation of {\tt CodeInjector} will perform reliable point-to-point
transmission of {\em srec\_file} to the {\em destination}. In this case, the
{\em destination} field should not be set to the broadcast address.
\begin{alltt}
java net/tinyos/codeGUI/CodeInjector multicheck {\em srec_file} {} {}
\end{alltt}

If you're programming a single mote, you may see a sequence like:

\begin{scriptsize}
\begin{verbatim}
[szewczyk@okocim tools]$ java net/tinyos/codeGUI/CodeInjector write ../apps/cnt
_to_leds_prog/binrene/main.srec 162
Program ID:5FC5
+
+!+++++++++++++++++++++++++++++++++++++++++++++++++++++++++++++++++++++++++++++
++
+++++++++++++++++++++++++++++++++++++++++++++++++!+++++++++++++++++++++++++++++
++
+++++++++++++++++++++++++++++++++++++++++++++++++++++++++++++++++++++++++++++++
++++++++++++++++++!++++++++++++++++++++++++++++++++++++++++++++++++++++++++++++
+++
+++++++++++++++++++++++++++++++++++++++++++++++++++++++++++++++++!+++++++++++++
++
+++++++++++++++++++++++
[szewczyk@okocim tools]$ java net/tinyos/codeGUI/CodeInjector check ../apps/cnt
_to_leds_prog/binrene/main.srec 162
Program ID:5FC5
++.7e 0 31 7d c5 5f 0 20 ff ff ff ff ff ff ff ff ff ff ff ff ff ff ff ff 0 0 0
0 0 0 0 0 0 0 bc 18
++.7e 0 31 7d c5 5f 0 20 ff ff ff ff ff ff ff ff ff ff ff ff ff ff ff ff 0 0 0
0 0 0 0 0 0 0 bc 18
++.7e 0 31 7d c5 5f 20 20 ff ff bf df ff ff ff ff ff 7f ff ff ff ff ff ff 0 0 0
 0 0 0 0 0 0 0 31 4b
+.7e 0 31 7d c5 5f 20 20 ff ff bf df ff ff ff ff ff 7f ff ff ff ff ff ff 0 0 0
0 0 0 0 0 0 0 31 4b
Missing packets:278 285 335
[szewczyk@okocim tools]$ java net/tinyos/codeGUI/CodeInjector check ../apps/cnt
_to_leds_prog/binrene/main.srec 162
Program ID:5FC5
++.7e 0 31 7d c5 5f 0 20 ff ff ff ff ff ff ff ff ff ff ff ff ff ff ff ff 0 0 0
 0 0 0 0 0 0 0 bc 18
+.7e 0 31 7d c5 5f 10 20 ff ff ff ff ff ff ff ff ff ff ff ff ff ff ff ff 0 0 0
 0 0 0 0 0 0 0 be 71
+.7e 0 31 7d c5 5f 20 20 ff ff ff ff ff ff ff ff ff ff ff ff ff ff ff ff 0 0 0
 0 0 0 0 0 0 0 b8 ca
+.7e 0 31 7d c5 5f 30 20 ff ff ff ff ff ff ff ff ff ff ff ff ff ff ff ff 0 0 0
 0 0 0 0 0 0 0 ba a3
\end{verbatim}
\end{scriptsize}
\item Issue the reprogramming command. This command will reset the mote, erase
the currently running program, and transfer the new program into memory. The
mote will be unresponsive for about 2 minutes. To perform this task, you
should invoke:
\begin{alltt}
java net/tinyos/codeGUI/CodeInjector start {\em srec_file} {\em destination}
\end{alltt}
\end{itemize}



\section*{Gory Details}

Now we 


%%%%##########################################################################

\end{document}
