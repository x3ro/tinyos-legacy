\documentclass[11pt,onecolumn]{article}
\usepackage{times}
\usepackage{verbatim}

%\usepackage{draftcopy}

\addtolength{\topmargin}{-.5in}        % repairing LaTeX's huge margins...
\addtolength{\textheight}{1.5in}        % more margin hacking
\addtolength{\textwidth}{1in}
\addtolength{\oddsidemargin}{-0.5in}
\addtolength{\evensidemargin}{-0.5in}

%\newenvironment{sh_out}{\begin{small}\begin{verbatim}}{\end{verbatim}\end{small}}

\begin{document}
\title{TinyDiffusion HOWTO\\version 1.0}

\author{Vladimir Bychkovskiy\\vladimir@lecs.cs.ucla.edu}
\date{}

\maketitle

\tableofcontents

\section{Introduction}
This HOWTO is for people who are interested in wireless embedded
sensor network data communication/coordination algorithms. This
document is a practical introduction into the internals of a Directed
Diffusion\cite{diffpaper} implementation on a low-end wireless sensor
node, UCB Mote running TinyOS\cite{tinyos}.

\subsection{Target Audience}
This document is meant for people who are interested in
exploring, using, and extending our implementation of Directed
Diffusion. Therefore it assumes a fair degree of familiarity with the
concepts of Directed Diffusion. The reader is also assumed to at least
have some experience with UCB Motes and TinyOS\cite{tinyos}. If you
are not yet familiar with TinyOS, it is recommended to suspend
reading of this document and come back to it after getting acquainted
with the aforementioned software and hardware\cite{tinyoswebsite}.

\subsection{\label{background}Background and Related Work}
Directed Diffusion\cite{diffpaper} is a scalable coordination
algorithm for wireless sensor networks. Initially, Directed Diffusion
was implemented\footnote{Implemented by Fabio Silva and et.\ al at
USC/ISI.} under Linux running on emmbedded PCs\cite{fabiocode}.It also
runs under Linux on iPAQ (StrongARM-based handheld) platform. Code and
memory limitation of motes have forced a complete re-implementation of
Directed Diffusion. The result, TinyDiffusion, is described
here\footnote{The very first implementation of TinyDiffusion was done
by Deepak in April 2001.}. One of the reasons for this implementation
of Directed Diffusion is proof of concept. Therefore, it is not yet
implemented as a library. In the future we plan to make it more
flexible (see section~\ref{futurework} on page~\pageref{futurework}).

\subsection{Features}
The feature set of the current version of TinyDiffusion is somewhat
limited for the reasons outlined above.

\begin{itemize}
\item{Algorithm}
	\begin{itemize}
		\item Initial \emph{interests} are distributed by flooding.
		\item All data messages are \emph{broadcasted}
		\item Only \emph{one path} is reinforced 
	\end{itemize}
\item{Attributes}
	\begin{itemize}
		\item A total of 250 data types are available to users
		\item Data is limited to 1-byte integer type
	\end{itemize}
\end{itemize}

Out of the box TinyDiffusion is able to route queries and provide data
from light and temperature sensors. It should be easy to extend
TinyDiffusion to provide data from other sensors, if such sensors are
available. See section~\ref{extend} on page~\pageref{extend} for more
information.

\subsection{Prerequisites}
In order to use TinyDiffusion you need
\begin{itemize}
	\item[-] at least 4-5 MICA\footnote{Current version of
	TinyDiffusion can run on rene163 as well, but it is not
	recommended.} motes 
	\item[-] a working TinyOS environment and
	AVR-GCC toolchain (see \cite{tinyos} for more information)
	\item[-] Motenic application, \emph{moted} driver, \emph{kfusd}
	kernel module (all of the above can be obtained from the SCADDS CVS
	repository\cite{scadds}\footnote{At the time of writing these
	drivers are only available for Linux kernel 2.4+, so you need an
	appropriate workstation})
	\item[-] TinyDiffusion source code. The latest version of the code
	can be obtained from \cite{scadds}, in
	\emph{microteers/mote/tos/nest/apps/diffusion\_sensor} subdirectory 
	\item[-] TinyDiffusion host side tools. A corresponding version of
	the source code is contained in the
	\emph{difftools} subdirectory in the TinyDiffusion source tree.
\end{itemize}

\section{Getting Started}
In this section you will learn how to set up a simple sensor
network and how to get sensor data out of it.

\subsection{Compilation and Building}
Before you proceed any further, you need to learn how to compile and
load TinyDiffusion and Motenic on the motes and how to set up host
side tools.

\subsubsection{\label{motenic}MoteNIC}
In order to be able to interact with TinyDiffusion you need a gateway
into your network. MoteNIC stands for Mote Network Interface Card. It
was originally developed by Jeremy Elson in the LECS at UCLA. MoteNIC
comes with a userspace device driver \emph{moted} for Linux v2.4+. In
order to use this driver you will also need to install
\emph{kfusd}\footnote{This is a framework for developing userspace
drivers, written by Jeremy Elson}. \emph{Kfusd} and moted can be found
in\cite{scadds} (see \emph{microteers/fusd} and
\emph{microteers/mote/driver/}).  In order to make MoteNIC do the
following:

\begin{verbatim}
 $cd (SCADDS CVS)/microteers/mote/tos/nest/apps/transceiver
 $make install mica
\end{verbatim}

This should produce a mote that is capable of sending and receiving
messages. You need to connect this mote to a serial port of
the computer that has \emph{moted} installed on it.

Before you proceed any further you need to test your MoteNIC. Program
a second mote with MoteNIC in the same way as above. Connect the
second mote to another serial port. Note that you could use serial
ports of the same or different computers; to simplify further
description, I will assume that you are using two serial ports of a
single computer. Run \emph{moted} for each MoteNIC. To increase
probability of packet delivery increase the transmission power:
\begin{verbatim}
 $echo `pot=5' > /dev/mote/1/command
\end{verbatim}
%$
Do the following the to test their
ability to communicate in the following way:

For motenic 1 do:
\begin{verbatim}
 $cat /dev/mote/0/frag
\end{verbatim}

For motenic 2 do:
\begin{verbatim}
 $echo `Hello there!' > /dev/mote/1/frag
\end{verbatim}
%$
(note that mote device numbers depend on the parameters that you have
used to run \emph{moted})

If everything worked, the output \emph{Hello there!} should appear in the
first window. If the output does not appear after multiple
\emph{echo} commands, you may need to go back
and check your setup.

\subsubsection{TinyDiffusion}
\label{tinydiffusion}
To compile TinyDiffusion, do the following:
\begin{verbatim}
 $cd (SCADDS CVS)/microteers/mote/tos/nest/apps/diffusion_sensor
 $make install mica
\end{verbatim}
Program as many of these nodes as you need.
 
\subsubsection{Host Tools}
Before you can test TinyDiffusion, you need to compile the host side
tools. These tools reside in the \emph{diffcntl} subdirectory of
\emph{diffusion\_sensor} directory.

\begin{verbatim}
 $cd diffcntl
 $make 
\end{verbatim}
%$
Now lets test your setup again. Open a new terminal window and run
\emph{diffdump} from \emph{diffcntl} directory. This program will
parse all TinyDiffusion and related messages that MoteNIC was able to
receive and present them in a human readable form.
\begin{verbatim}
 $./diffdump 
\end{verbatim}%$

Note: \emph{diffdump} displays no output if there are no messages coming
from MoteNIC and the only way to terminate it is Ctrl-C.
 
In another terminal window set the transmission power of all the motes to maximum
\begin{verbatim}
 $./txpower -1 5
\end{verbatim}
%$
At this point you should see a few responses in the diffdump window.
% should I insert a output here?
If messages are being transmitted by the TinyDiffusion nodes, they
change the state of the red led. MoteNIC signals transmission with a
state change in the yellow led; it signals reception with the green
led. Use this information to verify your set up and to detect problems.

\subsection{Preparation}
Before you can use TinyDiffusion you need to assign IDs and
coordinates to the nodes. You also need to set an appropriate
transmission power in order to run multi-hop experiments.

\subsubsection{ID assignment}
\label{idassign}
In the current implementation of TinyDiffusion each node needs to have
a unique ID (a MAC address). \emph{Please note that, these IDs are
stored in RAM.\footnote{On RENE motes this enabled parallel programming
of nodes since the same FLASH image is used by everyone.} ID needs to
be reset every time the TinyDiffusion node is powered on.} Another piece
of unique information that is necessary for correct operation of
TinyDiffusion is the location of a node. Locations are derived from ID
using the following formulae:\\
\begin{math}
 x = ID mod 100\\
 y = ID / 100\\
\end{math}
Note: \(x\) and \(y\) are integers

Use the following procedure to assign IDs to all of the nodes:

\begin{enumerate}
	\item Turn off all of the diffusion nodes 
	\item Turn on one of the turned off nodes
	\item Set its ID by running:
	\begin{verbatim}
	 $./txid <id>
	\end{verbatim}
	%$
	You may not see a response, in the \emph{diffdump}
	window\footnote{Node will only respond once.}, but you should see
	the change of state of the red led.
	\item If there are turned off motes, go back to step 2
\end{enumerate}

It is time for another, more interesting, test. This time you are
going to send a very simple query. Lets assume that you have a
diffusion node with ID 0101 and coordinates (1,1). Use the
\emph{interest} program to send an exploratory interest message:

\begin{verbatim}
	$./interest 
	Creating a new interest (Ctrl-C to exit) :
	Enter the region (lower left and upper right: x1,y1,x2,y2):1,1,1,1
	Enter the type:1
	Enter the interval:0
	Enter the expiration:100
	Enter the send address:12
	Enter the dest. address (-1 for broadcast == exploratory):-1
	Sending the following interest:
	[INTEREST] dest:65535 type:1 (1,1)-(1,1) interval:0 expiration:100 sender:12
\end{verbatim}
%$
(meanings of all fields will be explained in a later section)

You may need to send this \emph{interest} multiple times due to packet
loss. Responses should appear in \emph{diffdump} window. All nodes
should rebroadcast the exploratory \emph{interest}; node 0101 should
responds with data from it's light sensor (because we have specified
type $1$ above).

\subsubsection{The Hard Part: Setting Up A Communication Topology}
The last and the hardest step in preparation is setting up the network
topology. Use \emph{txpower} program to set and test the level of the
power of each of the nodes in the network. \emph{The good news is that
this setup need to be done only once, because power setting is stored
in the EEPROM.} General procedure for setting up a multi-hop topology is
the following\footnote{We have attempted to automate this
process. Take a look at \emph{calibrate} program in
\emph{diffcntl}. However it may need a little tuning.}:
\begin{itemize}
	\item Arrange all nodes on a 2D plane
	\item For each node
	\begin{itemize}
		\item Place MoteNIC next to the geometrically furthest node
		that \emph{should} be able to receive and pick a lowest power
		level at which such reception is possible

		\item Place MoteNIC next to the geometrically closest node
		that \emph{should not} be able to receive and increase the
		power level just below the level at which this node is able to
		receive 
	\end{itemize}
\end{itemize}

We have found it very hard to create more than 2-3 hop topologies
within a small area (4x4 sq. ft.). The potentiometer on the mote may not provide
sufficient granularity to establish good topologies at these small
scales\footnote{This problem is not specific to our application. It is
a hardware problem.}. You may need to rearrange nodes to achieve
desired connectivity.

Asymmetric links is another problem that arises when transmit power is
tuned down. As the time of writing we have not found an elegant
solution to this problem. The best approximations so far have been 
neighbor list based techniques\footnote{Link layer component that
implements this technique is available in SCADDS CVS, but has not been used
with TinyDiffusion}.

\subsection{Running Diffusion Queries}
After all the preparation and testing you are ready to use your
wireless sensor network to run some queries. As you already know
querying process in Directed Diffusion consists of two steps:
discovery, and reinforcement.

\subsubsection{Discovery}
Before you can retrieve sensor data from a sensor network you need to
find a path to the appropriate sensors. This is accomplished by
sensing and exploratory \emph{interest}. You already have experience
with sending interest from section~\ref{idassign}. Lets take a more
detailed look at interest sending:

\begin{verbatim}
	$./interest 
	Creating a new interest (Ctrl-C to exit) :
	Enter the region (lower left and upper right: x1,y1,x2,y2):0,0,10,10
\end{verbatim}%$
Here you specify a \emph{geographical region} for your query. 

\begin{verbatim}
	Enter the type:2
\end{verbatim}
\emph{Type} denotes the type of data of your query. 1 -- light, 2 --
temperature\footnote{These types are defined in DIFFNODE.inc}.
\begin{verbatim}
	Enter the interval:4
	Enter the expiration:100
\end{verbatim}
\emph{Interval} and \emph{expiration} denote time interval between
samples and total active time of query, respectively. The time unit is
1 second\footnote{Timing can be adjusted by changing the value of
CLOCK\_DIV in DIFFNODE.inc.}. Note that interval of 0 implies 1 second
delay between samples (delay b/w samples = interval + 1) . Exploratory
interests usually specify low data rates.
\begin{verbatim}
	Enter the send address:12
\end{verbatim}
\emph{Send address} denotes the address from which your message is going to
appear to be. This could be a virtual address that you assign to your
base station (MoteNIC). In general, it is recommended to use a unique
node address\footnote{For debugging purposes it may useful to
\emph{fake} a message from one of the nodes of the network, but you
must know what you are doing.}.
\begin{verbatim}
	Enter the dest. address (-1 for broadcast == exploratory):-1
\end{verbatim}
\emph{Destination address} is the destination of the interest. In the
current implementation of TinyDiffusion, \emph{exploratory} is it
broadcasted\footnote{Alternatively, scoped flooding or geographic
routing could be used.} to all of the neighbors. If you want to
reinforce and particular path, the destination address would be set to
the address of that node.
\begin{verbatim}
	Sending the following interest:
	[INTEREST] dest:65535 type:2 (0,0)-(10,10) interval:4 expiration:100 sender:12
\end{verbatim}
This line simply restates all the input values provided by you.

After you send this interest you should see rebroadcasted interests
from other nodes in diffdump window. After interest reaches the
destination region, the data should start flowing back.

\subsubsection{Reinforcement}
After you see a few data messages directed to the base state
(destination address == 12) in the diffdump window, you can choose to
increase the data rate and expiration one or more data sources. This
process is called reinforcement in Directed Diffusion. You can send
\emph{reinforcements} using the same interest program as you used to
send initial \emph{interest}. If you want to reinforce temperature
data from a node at location (3,4) and the shortest
path\footnote{Based on the hops to source field in exploratory data
messages} lies through node with address 31, your interaction with
interest program would look as following:
\begin{verbatim}
	$./interest 
	Creating a new interest (Ctrl-C to exit) :
	Enter the region (lower left and upper right: x1,y1,x2,y2):3,4,3,4
	Enter the type:2
	Enter the interval:0
	Enter the expiration:200
	Enter the send address:12
	Enter the dest. address (-1 for broadcast == exploratory):31
	Sending the following interest:
	[INTEREST] dest:31 type:1 (3,4)-(3,4) interval:0 expiration:200 sender:12
\end{verbatim}
%$
Note that region includes only one point and destination address is
set to the address of the node that has the shortest path. Interval is
set to zero for the highest possible data rate of one sample per second.

\section{Extending TinyDiffusion}
\label{extend}
Now that you have experienced basic capabilities of TinyDiffusion, you
are ready to build you own applications. Before you start building
your application, you need to learn how TinyDiffusion code is structured.

\subsection{Code Structure}
Most of the TinyDiffusion logic and application code is located in
DIFFNODE.c. This file includes message handlers and timer event
handler. These handlers heavily relay on libraries that implement
diffusion data structure construction and access methods. The
libraries are located in DiffNodeInc directory.

TinyDiffusion data structures include \emph{interest table},
\emph{gradient table}, \emph{location table} and \emph{data
cache}\footnote{By reusing common attributes as much as possible we
save RAM.}. Type information is stored in interest table, location
information is stored, respectively, in the location table. The
gradient table refers to interest and location tables to specify
gradient attributes. In addition to that, the gradient table contains
rate and expiration information for each gradient. Each interest entry
caches the highest rate gradient to avoid extensive lookup on every
timer event.  Attribute information in the location and the interest
tables is matched for incoming messages.

The data cache is used to store incoming data messages. Note that
reinforcement policy is directly related to the data caching
policy. In order to reinforce a path TinyDiffusion node simply finds a
data with matching attributes and sends reinforcement to the
sender of the data.

Some configuration parameters of TinyDiffusion as contained in
DIFFNODE.inc. Table sizes are defined in DiffTable.inc and
DataCache.inc\footnote{TinOS make scripts remove  *.h files. That is
why we used *.inc for header files.}.

\subsection{How to Build Your Own Application}
In this section you will find general guidelines on building your
application on top of TinyDiffusion; study the code to get more
information.

It is very easy to extend TinyDiffusion to use new data types. In
DIFFNODE.c do the following:
\begin{enumerate}
	\item add a new variable to the frame (ex: supersensorData)
	\item modify ClockTickTask, add sampling of your sensor
	\item modify produceData function
	\item add an ADC callback (see DIFFNODE\_SUB\_ADC\_PHOTO\_DATA). This
		  is the biggest change, because it requires changing
		  DIFFNODE.comp and diffnode.desc files.
\end{enumerate}

If you want to modify reinforcement policy, read notes in the source
code for data message handler(DIFFNODE\_RX\_DATA\_MSG). For more complex
modification you need to study source code. \emph{Note that
we a planning to make some large scale changes to TinyDiffusion, see
section~\ref{futurework} for more}

\section{Future Work}
\label{futurework}
In the future we plan to develop a simplified publish/subscribe API
that would enable building applications on top of the TinyDiffusion. We
believe that this will make TinyDiffusion easier to use. TinyDiffusion
would be a layer in the protocol stack. Applications will still be
able to bypass this layer. This arrangement will also allow to use
different routing mechanisms, such as geographic and/or energy-aware
routing, to be decoupled from application logic.

The main penalty for publish/subscribe implementation of TinyDiffusion
will be the increase in the code and RAM footprint. Since we also intend
to switch to an XML style\footnote{Very, very simplified!} data
representation, this will increase message size.

%-------------- Bibliography ----------------

\begin{thebibliography}{9}
\bibitem{diffpaper}
    C. Intanagonwiwat, R. Govindan and D. Estrin, "Directed Diffusion:
    A Scalable and Robust Communication Paradigm for Sensor
    Networks," ACM/IEEE International Conference on Mobile Computing
    and Networks (MobiCom 2000), August 2000, Boston, Massachusetts.
\bibitem{tinyoswebsite}
	TinyOS: An Operating System for Networked Sensors\\
	http://tinyos.millennium.berkeley.edu

\bibitem{scadds}
    SCADDS CVS repository.\\
    http://hey.isi.edu/viewcvs

\bibitem{tinyos}
    TinyOS Sourceforge repository\\
    http://sourceforge.net/projects/tinyos	

\bibitem{fabiocode}
    Linux-based implementation of Directed Diffusion.\\
    Source code: http://www.isi.edu/scadds/testbeds.html

\end{thebibliography}


	

\end{document}

