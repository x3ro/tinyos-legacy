\section{Appendix B: \name Reference}

%% someday maybe this will be autogenerated from the javadoc... not too likely

\subsection{simcore package}

The {\tt simcore} package is the base interface between scripts and
\name. It consists of the following reflected Java classes:

\begin{raggedright}

\vspace{12pt}
\begin{tabular}{|ll|l|}\hline
\multicolumn{3}{|l|}{\textbf{class Command}}\\
\hline
void 			       && \\
sendRadioMessage&(short moteID,	& Send a fabricated Radio {\tt Message} instance to the \\
&long time,                     & given {\tt moteID} at the specified time. \\
&Message msg)                   & \\
\hline
void 			       && \\
sendUARTMessage&(short moteID, 	& Send a fabricated UART {\tt Message} instance to the \\
&long time,                     & given {\tt moteID} at the specified time. \\
&Message msg)                   & \\
\hline
void 			       && \\
turnMoteOn&(short moteID,	& Turns the specified {\tt moteID} on at the given time\\
&long time) 			& \\
\hline
void 			       && \\
turnMoteOff&(short moteID,	& Turns the specified {\tt moteID} off at the given time\\
&long time) 			& \\
\hline
void 			       && \\
setADCValue&(short moteID,	& Sets the value of the ADC sensor at the given {\tt port} \\
&		long time,	& of the specified {\tt moteID} to a new {\tt value} \\
&		byte port,	& \\
&		short value)	& \\
\hline
void 			       && \\
setSimRate&(double rate) 	& Adjusts the rate of simulation. Analogous to the {\tt -l}  \\
			       && TOSSIM command line option. \\
\hline
void 					&& \\
setLinkBitErrorProbability&(short src,	& Sets the radio link bit error rate from {\tt src} to {\tt dest} \\
&			long time,	& to the new value of {\tt loss}. Note that {\tt} loss is a bit error\\
&			short dest,	& probability, not a packet loss probability. \\
&			double loss)	& \\
\hline
double					&& \\
packetLossToBitError&(double packetLoss) & As per above, converts a desired packet loss rate into\\
				       && the appropriate bit error rate. \\
\hline
double 				       && \\
distanceToPacketLoss&(double distance) & Returns the packet loss for a given distance from the \\
				       && current radio model. \\
\hline
void 			       && \\
pauseInFuture&(long time,	& Schedules a {\tt SimulationPausedEvent} to be delivered \\
&		int pauseID)	& at the specified {\tt time}.\\
\hline
VariableResolveEvent 	       && \\
resolveVariable&(short moteID,	& Returns an event with the address and length of the variable\\
&		String name)	& specified by {\tt name}, or \{0, -1\} if it cannot be resolved.\\
\hline
VariableValueEvent 		&& \\
requestVariable&(long addr,	& Returns {\tt length} bytes from TOSSIM memory at {\tt addr}.\\
&		short length)	& \\
\hline
void 			       && \\
setDBG&(long dbg) 		& Enables the specified {\tt dbg} flag.\\
\hline
int 			       && \\
getPauseID()& 			& Returns a unique identifier for use in {\tt pauseInFuture}.\\
\hline
\end{tabular}

\vspace{12pt}
\begin{tabular}{|ll|l|}\hline
\multicolumn{3}{|l|}{\textbf{class Interp}}\\
\hline
int					&& \\
addEventHandler&(PyFunction callback) 	& Register the specified {\tt callback} to be called with events.\\
					&& Returns a unique identifier for the event handler. \\
\hline
int					&& \\
addEventHandler&(PyFunction callback, 	& Same as above, but only deliver events of the specified\\ 
&PyJavaClass eventclass)		& Java class {\tt eventclass}\\
\hline
int					&& \\
removeEventHandler&(int id)		& Remove a registered event handler with identifier {\tt id}.\\
\hline
\end{tabular}


\vspace{12pt}
\begin{tabular}{|ll|l|}\hline
\multicolumn{3}{|l|}{\textbf{class Mote}}\\
\hline
int				&& \\
getID()&			& Return the mote id.\\
\hline
String				&& \\
getCoord()&			& Return a string of the mote location in the form ``(x, y)''.\\
\hline
double				&& \\
getXCoord()&			& Return the X coordinate of the mote.\\
\hline
double				&& \\
getYCoord()&			& Return the Y coordinate of the mote.\\
\hline
String				&& \\
toString()&			& Return a string depiction of the mote state.\\
\hline
double				&& \\
getDistance&(int moteID)	& Return the distance to another mote, specified by {\tt moteID}.\\
\hline
double				&& \\
getDistance&(double x, 		& Return the distance to the specified location.\\
&double y)			& \\
\hline
void				&& \\
turnOn()&			& Power up the mote.\\
\hline
void				&& Power down the mote.\\
turnOff()&			& \\
\hline
boolean				&& \\
isOn()&				& Return the state of whether or not the mote is on.\\
\hline
void				&& \\
setLabel&(String label, 	& Sets a descriptive string {\tt label} to be displayed if the TinyViz\\
&int xoff, 			& GUI is running. {\tt xoff} and {\tt yoff} specify the coordinate offsets\\
&int yoff)			& from the mote's position that the string should be displayed.\\
\hline
void				&& \\
move&(double dx,		& Move the mote by the given {\tt dx} and {\tt dy} offsets.\\
&double dy)			& \\
\hline
void				&& \\
moveTo&(double x,		& Move the mote to the absolute position given by {\tt x}, {\tt y}\\
&double y)			& \\
\hline
byte[]				&& \\
getBytes&(String var)		& Return the mote variable named by {\tt var} as a byte array.\\
\hline
long				&& \\
getLong&(String var)		& Return the mote variable named by {\tt var} as a long integer.\\
\hline
int				&& \\
getInt&(String var)		& Return the mote variable named by {\tt var} as an integer.\\
\hline
short				&& \\
getShort&(String var)		& Return the mote variable named by {\tt var} as a short integer.\\
\hline
byte				&& \\
getByte&(String var)		& Return the mote variable named by {\tt var} as a byte.\\
\hline
\end{tabular}


\vspace{12pt}
\begin{tabular}{|ll|l|}\hline
\multicolumn{3}{|l|}{\textbf{class Radio}}\\
\hline
String				&& \\
getCurModel()&			& Return the name of the current radio model.\\
\hline
void				&& \\
setCurModel&(String modelname)	& Set the current radio model.\\
\hline
void					&& \\
setScalingFactor&(double scalingFactor)	& Set the distance scaling factor.\\
\hline
double				&& \\
getLossRate&(int senderID, 	& Get the packet loss rate between the specified motes.\\
&int receiverID)		& \\
\hline
void				&& \\
setLossRate&(int senderID, 	& Get the packet loss rate between the specified motes.\\
&int receiverID, 		& \\
&double prob)			& \\
\hline
void				&& \\
printLossRates()&		& Dump out the current loss rate graph.\\
\hline
void					&& \\
setAutoPublish&(boolean autoPublish)	& Set a flag indicating whether or not the radio should \\
					&& automatically propagate model changes due to motes moving\\ 
					&& to TOSSIM.\\ 
\hline
void				&& \\
updateModel()&			& Force an update of the internal radio model connectivity graph.\\
\hline
void				&& \\
publishModel()&			& Publish the current connectivity graph to TOSSIM.\\
\hline
\end{tabular}

\vspace{12pt}
\begin{tabular}{|ll|l|}\hline
\multicolumn{3}{|l|}{\textbf{class Sim}}\\
\hline
void				&& \\
pause()&			& Pause the simulation.\\
\hline
void				&& \\
resume()&			& Resume the simulation.\\
\hline
void				&& \\
stop()&				& Stop the simulation.\\
\hline
long				&& \\
getTossimTime()&		& Return the ``current'' time in TOSSIM time units. This is actually \\
				&& the time that the last TOSSIM event was received by the system. \\
\hline
void				&& \\
exit&(int errcode)		& Exit the process.\\
\hline
void				&& \\
setSimDelay&(long delay\_ms)	& Set the delay parameter. \\
\hline
void				&& \\
dumpDBG&(String filename)	& Start spooling TOSSIM DBG messages to the given {\tt filename}\\
\hline
void				&& \\
stopDBGDump()&			& Stop dumping out DBG messages.\\  
\hline
\end{tabular}

\end{raggedright}

\newpage
\subsection{simutil package}

The {\tt simutil} package is a python package that implements a some
richer functional additions to much of the {\tt simutil}
functionality. It provides the following python classes and functions:

\begin{raggedright}

\vspace{12pt}
\begin{tabular}{|ll|l|}\hline
\multicolumn{3}{|l|}{\textbf{Functions}}\\
\hline
call\_at&(when,			& \\
&callback,			& Schedules an event to call {\tt callback} with optional {\tt args}\\
&args = None)			& at time {\tt when} in the future. Returns the event ID.\\
\hline
call\_in&(delay,		& \\
&callback,			& Same as {\tt call\_at}, except the call takes place in {\tt delay}\\
&args = None)			& ticks past the current time. Retuns the event ID.\\
\hline
call\_cancel&(id)		& Cancels the previously scheduled call.\\
\hline
\end{tabular}

\vspace{12pt}
\begin{tabular}{|ll|l|}\hline
\multicolumn{3}{|l|}{\textbf{class Periodic}}\\
\hline
Periodic&(interval,		& \\
&callback,			& The class constructor interacts with the event system\\
&args = None,			& to schedule {\tt callback} to be called with optional {\tt args} \\
&call\_immediate = 1)		& repeatedly every {\tt interval} ticks. The {\tt call\_immediate} option \\
				&& controls whether the first call is at time 0 or waits for an interval.\\
\hline
void &&\\
stop()&				& Stops the periodic call. \\
\hline
boolean &&\\
is\_stopped()&			& Returns whether or not the object is stopped. \\
\hline
\end{tabular}

\vspace{12pt}
\begin{tabular}{|ll|l|}\hline
\multicolumn{3}{|l|}{\textbf{class MoteMover}}\\
\hline
MoteMover()&			& Creates a {\tt MoteMover} instance.\\
\hline
void &&\\
moveTo&(mote,			& Move the given {\tt mote} object in increments to the ({\tt x, y}) location. \\
&step, 				& Call the {\tt callback} (if any) when it gets there. The {\tt step} \\
&x, y, 				& parameter controls how far to move on each interval, The {\tt rate} parameter\\
&callback = None, 		& controls the frequency of movement, -1 means use the class default. \\
&rate = -1)			& \\
\hline
void &&\\
randomWalk&(mote,		& Similar to {\tt moveTo}, though move the mote in a random direction \\
&step, 				& at each tick. The {\tt step} and {\tt rate} parameters have the same \\
&rate = -1)			& meaning. \\
\hline
void &&\\
stop&(mote)			& Stop the given mote's movement. \\
\hline
boolean &&\\
isMoving&(mote)			& Return whether or not the given mote is moving. \\
\hline
void &&\\
setDefaultRate&(rate)		& Set the default movement rate. \\
\hline
\end{tabular}

\end{raggedright}
