\documentclass[10pt]{article}

\usepackage{graphicx}
\usepackage{graphics}
\usepackage{multicol}
\usepackage{epsfig,amsmath,amsfonts}
\usepackage{xspace}

\makeatletter                                   % Make '@' accessible. 
\oddsidemargin=0in                              % Left margin minus 1 inch.
\evensidemargin=0in                             % Same for even-numbered pages.
\marginparsep=0pt                               % Space between margin & text

\renewcommand{\baselinestretch}{1}              % Double spacing

\textwidth=6.5in                                % Text width (8.5in - margins).
\textheight=9in                                 % Body height (incl. footnotes)

\topmargin=0in                                  % Top margin minus 1 inch.
\headsep=0.0in                                  % Distance from header to body.
\skip\footins=4ex                               % Space above first footnote.
\hbadness=10000                                 % No "underfull hbox" messages.
\makeatother                                    % Make '@' special again.

\newcommand{\name}{{Tython}\xspace}
\newcommand{\driver}{{\tt SimDriver}\xspace}

\title{\name: Scripting TOSSIM\\{\small \bf Work in Progress: Do not cite or distribute, comments welcome}}
\author{Michael Demmer and Philip Levis}
\date{}

\begin{document}

\maketitle
\vspace{2in}
\begin{center}
Version 0.4\\
Jan 9, 2004
\end{center}

\fontfamily{cmr}                                % Make text Roman (serif).
\fontseries{m}                                  % Medium spacing.
\fontshape{n}                                   % Normal: not bold, etc.
\fontsize{10}{10}                               % 10pt font, 10pt line spacing


\thispagestyle{empty}
\newpage
\tableofcontents
\newpage

\section{Introduction}

Experimenting with and testing sensor networks is hard. TOSSIM, a
TinyOS simulator, allows users to run and test algorithms, protocols,
and applications in a controlled, reproducible environment. However,
by itself a TOSSIM simulation is static. Instead of modeling behaviors
such as motion or changing sensor readings, TOSSIM provides a
socket-based command API for other programs to do so. On one hand, this
keeps TOSSIM simple and efficient; on the other, it puts the burden
of writing complex real-world models on the user.

One solution to this problem is TinyViz, a GUI that communicates with
TOSSIM over the socket API. With TinyViz, users can interact with a
simulation through a GUI panel, by dragging motes and setting options.
These actions can be difficult to reproduce exactly (e.g., dragging a
mote). Additionally, TinyViz can (as its name suggests) visualize what
goes on in the network. Users can write TinyViz ``plugins'' in Java to
extend the GUI's functionality and issue commands to TOSSIM. However,
because users must precompile their plugins, this only allows limited
interactivity.

\name (or, Tinython) complements TinyViz's vizualization by adding a
scripting interace to TOSSIM. Users can interact with a running
simulation through TinyViz, a \name console, or both simultaneously.
\name is based on Jython, a Java implementation of the Python
language. In addition to being a complete scripting language (with
plenty of documentation, literature, tutorials, etc.), Jython makes it
very easy to import and use Java classes within Python. This allows
users to access the entire TinyOS Java tool chain, including packet
sources, MIG-generated messages, and TinyViz.

This document explains how to get started with \name, and explains the
library functions that can be used to access TOSSIM or TinyViz. It
does not contain an in-depth Python tutorial: many of those can be
found at {\tt www.python.org}.

\section{Getting Started}

TinyViz and \name sit on top of \driver, a java application that
manages interactions with TOSSIM. The simplest way to use \name is to
start a TOSSIM simulation with the {\tt -gui} option, then run \driver
with the {\tt -console} option:

\begin{verbatim}
java net.tinyos.sim.SimDriver -console
\end{verbatim}

The Jython environment may take a little while to come up the first
time; it scans all of the TinyOS jar files, but caches this
information for later invocations. After booting, \name will give you
a script prompt:

\begin{verbatim}
>>>
\end{verbatim}

\subsection{Pausing and Resuming}
\label{sec:pause}

If you used the {\tt -gui} option when invoking TOSSIM (or used the
{\tt -run} option), then the simulation will be paused when \name
boots. The {\tt sim} object allows you to pause and resume TOSSIM.
Additionally, hitting the escape key and hitting return will pause a
running simulation. For example:

\begin{verbatim}

>>> sim.resume()
>>> sim.pause()

\end{verbatim}

You can either type in scripts (useful for examining data) at the
prompt, or execute script files with the {\tt execfile} command,
specifying the file name:

\begin{verbatim}

>>> execfile("script.py")

\end{verbatim}

\subsection{Invoking SimDriver}

\driver can be invoked with a variety of options. For example, {\tt
  -console} has \driver start a scripting console, while {\tt -gui}
has it start a GUI interface (TinyViz). The {\tt -run} option will
automatically invoke a TOSSIM simulation, while {\tt -args} allows you
to pass arguments to the TOSSIM invocation. For example,

\begin{verbatim}
# java net.tinyos.sim.SimDriver -console -gui -args "-l=1.0 -b=1" -run main.exe 20
\end{verbatim}

Will start TinyViz, provide a console, and start a TOSSIM simulation
(in this case, {\tt main.exe}) of 20 motes. The TOSSIM arguments
specify that it should run in real time if it can ({\tt -l=1.0}), and
the motes should boot in the first second ({\tt -b=1}). The {\tt -h}
flag option will proint the full set of \driver command line options.
For TOSSIM options, refer to the TOSSIM reference manual.

When \driver invokes TOSSIM, it always enables the TOSSIM lossy radio
model.

\subsection{Some Basic Commands}
\label{sec:commands}

In addition to a basic Jython environment, \name has several objects
that provide functions to interact with
TOSSIM. Section~\ref{sec:pause} showed how the {\tt sim} object can be
used to pause and resume a simulation. 

For all commands, specifying a time in the past will cause the command
to be executed as soon as possible (i.e., at current time in TOSSIM).
An easy way to have something happen immediately is to specify a time
of 0. Time is measured in simulator ticks, which are at 4MHz. So, a
time of one minute is 240,000,000, or 60 seconds of four million ticks
each. Some basic commands are shown in
Table~\ref{table:basic_commands}. Appendix B contains a reference
description of the \name object model.

\begin{table}

\centering

\begin{tabular}{|l|l|}\hline
Name & Effect \\ \hline
sim.pause()  & Pause TOSSIM\\
sim.resume() & Resume TOSSIM\\
motes[i].turnOn(time)  & Turn mote {\it i} on at time {\it time}\\
motes[i].turnOff(time) & Turn mote {\it i} off at time {\it time}\\
motes[i].moveTo(x, y)  & Move mote {\it i} to {\it x,y}\\
motes[i].move(x,y)     & Move mote {\it i} {\it x,y} from its current position\\
comm.waitUntil(time)   & Make the script block until {\it time} in TOSSIM\\
& Note script will hang if TOSSIM is paused.\\
comm.setSimRate(rate)  & Set the simulator rate to {\it rate}\\
                       & This is identical to TOSSIM's {\tt -l} option.\\
comm.sendRadioMessage(mote, time, message) & Deliver message to mote over radio\\
comm.sendUARTMessage(mote, time, message)  & Deliver message to mote over UART\\ \hline
\end{tabular}
\caption{Basic \name Commands}
\label{table:basic_commands}
\end{table}


The send commands have a message as a parameter; this message must be an
instance of a sublcass of {\tt net.tinyos.message.Message}. In the common
case, this is a message class created with the MIG tool. For example, to
send a basic AM packet to mote 2, with an address field of 2, AM type of 12, 
and an empty payload:

\begin{verbatim}
from net.tinyos.message import TOSMsg
msg = TOSMsg()
msg.set_addr(2)
msg.set_type(12)
msg.set_length(0)
comm.sendRadioMessage(2, 10 * 4000000, msg)
\end{verbatim}

This packet is scheduled to arrive at mote 2, ten seconds into
simulation (each second has four million ticks).


\subsection{Exploring Objects}

The \name environment has a few predefined Python objects specific to
a TOSSIM simulation. For example, the {\tt sim} object has functions for
pausing and resuming simulation. Use the {\tt dir()} function to see
the set of functions an object provides. First, type the name of the object:

\begin{verbatim}

>>> sim
net.tinyos.sim.script.reflect.Sim@5f3a5f53

\end{verbatim}

This tells you that the {\tt sim} object is actually an instance of the
{\tt net.tinyos.sim.script.reflect.Sim} Java class. To see the functions a
{\tt Sim} object provides, type {\tt dir(Sim)}.

\begin{verbatim}

>>> dir(Sim)
['dumpDBG', 'exit', 'pause', 'resume', 'setSimDelay', 'simDelay', 'stop', 'stopDBGDump']

\end{verbatim}

{\tt dir()} can be used on both Java and Python objects. For example,
{\tt dir(Sim)} returns a Python list of strings, with each element being a
method of the Sim class. So, {\tt dir(dir(Sim))} lists the functions
a Python list object provides:

\begin{verbatim}

>>> dir(dir(sim))
['append', 'count', 'extend', 'index', 'insert', 'pop', 'remove', 'reverse', 'sort']

\end{verbatim}

Note that {\tt dir()}'s semantics are slightly different for Python and
Java. One can call {\tt dir()} on a Python object to learn its functions,
but for Java, one must call it on the class. Specifically:


\begin{verbatim}

>>> sim
net.tinyos.sim.script.reflect.Sim@5f3a5f53
>>> dir(sim)
[]
>>> dir(Sim)
['dumpDBG', 'exit', 'pause', 'resume', 'setSimDelay', 'simDelay', 'stop', 'stopDBGDump']
>>> m = dir(Sim)
>>> dir(m) 
['append', 'count', 'extend', 'index', 'insert', 'pop', 'remove', 'reverse', 'sort']

\end{verbatim}

Calling {\tt dir()} with no parameters lists the set of variables
bound in the Jython environment:

\begin{verbatim}

>>> dir()
['Commands', 'Interp', 'Location', 'Mote', 'Motes', 'PacketType', 'Packets', 'Radio', 'Sim', 
'SimBindings', 'SimReflect', '__doc__', '__driver', '__name__', 'comm', 'interp', 'location', 
'motes', 'packets', 'radio', 'sim', 'simcore', 'sys']
\end{verbatim}

\subsection{Importing and Instantiating Java}

Jython allows you to import Java classes with Python's import syntax.
For example, to import {\tt java.util.Enumeration}, or the entire
{\tt java.util} package:


\begin{verbatim}

>>> from java.util import Enumeration
>>> from java.util import *

\end{verbatim}

Python is a dynamically typed language. Therefore, variables do not
have type specifiers. While instantiating a {\tt StringBuffer} in
Java looks like this\footnote{From now on, \name code will be shown as if it were in a file, that is, without a command line prompt. Each line could be just as easily typed into a console.},

\begin{verbatim}
StringBuffer buffer = new StringBuffer();
\end{verbatim}

in Jython it looks like this,

\begin{verbatim}
buffer = StringBuffer()
\end{verbatim}

Note, however, that Jython does not automatically import {\tt java.lang.*}.

By importing Java classes, one can access the TinyOS Java toolchain:

\begin{verbatim}
from net.tinyos.message import *
from net.tinyos.packet import *

moteIF = MoteIF()
msg = BaseTOSMsg()
msg.set_addr(0)
msg.set_type(5)
moteIF.send(0, msg)
\end{verbatim}

The above code will send a message of AM type 5 to mote 0 over the default
MoteIF interface.

Access to the toolchain includes GUI-based programs. For example, one
can instatiate a SerialForwarder:

\begin{verbatim}
from net.tinyos.sf import *
s = SerialForwarder([])
\end{verbatim}

Note, however, that some classes (SerialForwader's GUI among them)
expect to be stand-alone applications. This means that when you tell
them to quit, with the quit button, for example, they may cause the
entire Jython environment to quit.

\section{The Caffeinated Snake}

Jython's Java support allows a user to basically use it as a 
Java scripting language. However, being dynamically typed, it loses
many of the correctness benefits of the Java compilation process. 
Essentially, there is a lot of redundancy between Java and Python,
and moving back and forth between the two can be a little difficult at times.

For objects such as strings, lists, and maps, it's often easier to use
Python's primitives instead of Java's, as they are syntactically built
into the language. For example, compare the use of a Python map and a
Java hashtable:

\begin{verbatim}
# Java
from java.util import *
h = new Hashtable()
h.put("Seed", getPacket())
print(h.get("Seed"))

# Python
h = {}
h["Seed"] = getPacket()
print h["Seed"]
\end{verbatim}

Similarly, one can use either Java or Python to write a file:


\begin{verbatim}
# Java
from java.io import *
file = FileWriter("output.txt")
file.write(packets[0])

# Python
file = open("output.txt", "w")
file.write(packets[0])
\end{verbatim}

We've found that, for data collection and processing, Python is much
easier to use than Java. Among other things, its data structure
primitives end up being a lot simpler and less error prone. However,
Java allows you to interact with all of the existing structure a
TinyOS distribution provides, such as packet sources, message class
generation, and interaction with TinyViz.

\section{\name, SimDriver, plugins, and the radio}

The core of SimDriver is an event bus. Java plugins can connect to this
event bus, receiving TOSSIM events and sending TOSSIM commands; many
of the \name abstractions are built on top of SimDriver plugins. TinyViz,
the visualization environment, also uses plugins, so users can write new
visualizations. Correspondingly, there are two classes of SimDriver plugins:
GUI and non-GUI.

Examples of GUI plugins include TinyViz's radio quality visualization, 
and its communication visualization. Examples of non-GUI plugins include 
the radio model plugin and the packet logger plugin. GUI plugins are only
loadable through TinyViz, while non-GUI plugins are accessible through
\name, and are used to actuate TOSSIM in response to network changes.
SimDriver maintains a list of mote objects, which store state pertinent
to the scripting environment, such as position and LED values.

For example, calling mote.moveTo() (see Table~\ref{table:basic_commands}) 
modifies that mote's position. This can change radio connectivity. The
mote object uses the radio model plugin to calculate bit error rates
based on new distances. The plugin will then automatically communicate the
new link qualities to TOSSIM, which is unaware of mote position.

\subsection{Position-based Radio Models}

\name and TinyViz provide two position-based radio models: disc and
empirical. The first models the network using "perfect disc" connectivity,
at three different disc radii ("disc10", "disc100", and "disc1000"). 
Connectivity is perfect
in that links are either error-free or not present, depending on
whether the target is in or outside of the disc. Although
links are perfect, collisions can still occur, which will corrupt a packet.

The second model, empirical, uses an error distribution based on
empirical data. Essentially, for any particular distance there is a 
distribution of possible bit error rates: two links of identical length
can have different rates. These bit errors can cause a packet to be
corrupted so that a receiver does not recover it properly. Links are
directed: the error rate from A to B is distinct from the rate
from B to A. Roughly speaking, motes closer than 16 units will have
good connectivity, motes between 16 and 28 units will have highly variable
and possibly asymmetric connectivity, and motes 30 units and higher away
will have poor connectivity. 

These position models boil down to a simple function: given a distance
between two motes, produce a bit error rate. For disc-based models,
this is a simple cutoff, which results in symmetric and perfect links. For
the empirical model, the function returns a sample from the
distribution corresponding to the distance.

Scripts can also explicitly specify connectivity qualities with the
{\tt radio} object. {\tt setLossRate(src, dest, prob)} sets
the bit error rate from ID {\tt src} to ID {\tt dest} to be 
probability {\tt prob}. {\tt comm.packetLossToBitError(prob)}  calculates 
bit error rates from a packet loss rate~\footnote{This is actually a 
non-trivial calculation. It makes several assumptions about the data link 
encoding (the mica stack, with SecDed based encoding and a 9 bit start
symbol) and packet length. Specifically, the packet error rate represents
a 36-byte packet. As packets grow longer, their loss rate increases,
as there is a set bit error rate.}. Calling this function does not update
TOSSIM; that must be done with {\tt radio.publishModel()}. Calling
this updates the entire graph to TOSSIM, which is $O(n^{2})$. Therefore,
if a large number of links are updated, it's best to only publish
the model once, after all of the updates.

\subsection{GUI Labels}

When using \name with TinyViz, scripts can add annotations to 
mote visualizations. The {\tt setLabel()} method of the mote class
takes three parameters: a string, an X offset, and a Y offset.
For example,

\begin{verbatim}

motes[2].setLabel("queue depth: " + depth, 10, 0)

\end{verbatim}

will put a small label by mote 2 in the TinyViz visualization panel,
stating what its queue depth is (assuming the variable {\tt depth} has
been set to this value previously).

\section{\name Data Structures}

Section~\ref{sec:commands} described a few of the commands that
\name provides to interact with TOSSIM. In addition to these functions,
\name provides data structures to access simulator events. For
example, the {\tt packets} variable is a dictionary of every packet
that has been sent over the radio, indexed by mote ID.

\begin{verbatim}
print packets[3]
\end{verbatim}

will print all of the packets mote 3 has sent. Section~\ref{sec:commands}
mentioned the {\tt motes} variable, which is a list of motes. 

\begin{verbatim}
print motes[3]
\end{verbatim}

will display the current state of mote 3.

\subsection{Packets}
\label{sec:packets}

The {\tt packets} variable allows scripts to access all of the packets
that motes have transmitted. The variable is a dictionary of lists, keyed
by mote ID. Each list contains all of the packets a given mote has sent, 
in temporal order (the first packet is the first element of the list). 
Each element is a Java object, an instance of 
{\tt net.tinyos.sim.event.RadioMsgSentEvent}. This class is used because
it stores not only the message sent, but also metadata, such as time. 
The time represents when the packet send {\it completed}, not when it 
began. 

{\tt RadioMsgSentEvent} has three basic accessor functions: {\tt getMessage()},
which returns a {\tt net.tinyos.message.TOSMsg}, 
{\tt getMoteID()}, which returns the sender's
ID, and {\tt getTime()}, which returns the send time.

{\tt packets} is an immutable object; unlike normal dictionaries, a
script cannot assign to it. If a script needs to transform the packet 
dictionary, it can make a copy with {\tt packets.copy()}, which is
a normal Python dictionary and can be modified accordingly.

In addition to normal dictionary operations, {\tt packets} has an extra
function to support TinyOS messages, {\tt addPacketType(Message message)}.
This function can create packet type specific reflections of {\tt packets}.
For example,

\begin{verbatim}
from net.tinyos.message import IntMsg
packets.addPacketType(IntMsg())
\end{verbatim}

will create a new variable in the /name environment, {\tt IntMsgs}, which
contains all of the messages in {\tt packets} that are of type {\tt IntMsg},
as defined by active message type. This variable, {\tt IntMsgs}, can
be indexed and accessed just as {\tt packets} is.

Note, however, that each element in the {\tt packets} lists
contains a {\tt TOSMsg};
this means that in the above example, {\tt IntMsgs} also has {\tt TOSMsgs};
you cannot easily manipulate the elements as if they were actually instances
of {\tt IntMsg}, by accessing their message type specific fields, etc. 

In order to transform the {\tt IntMsgs} into a dictionary of
{\tt net.tinyos.message.IntMsg}, you must call {\tt IntMsgs.downCast()}.
For example:

\begin{verbatim}
from net.tinyos.msg import IntMsg
packets.addPacketType(IntMsg())

# Get a dictionary of IntMsgs, instead of TOSMsgs
myMsgs = IntMsgs.downCast()
# Print the value in the first IntMsg that mote 0 transmitted
print myMsgs[0].pop().get_val()
\end{verbatim}

Using a {\tt downCast()} dictionary allows you to easily access
message fields. However, it discards the additional metadata (such as
transmit time) that a {\tt RadioMsgSentEvent} contains.

The variable defined by {\tt addPacketType()} will update as more
motes send packets; it represents a filter on top of {\tt packets}. 
The dictionary returned by {\tt downCast()}, however, is static;
it does not update as more motes send packets.

\subsection{Motes}
\label{sec:motes}

The {\tt motes} variable is a list of {\tt Mote} objects, which
provide the following methods:

\begin{table}
\centering
\begin{tabular}{|l|l|} \hline
Name                                & Effect \\ \hline
int getID()                         & Returns the mote's ID\\
String getCoords()                  & A string representation of its x,y position\\
double getYCoord()                  & The x coordinate of its position\\
double getXCoord()                  & The y coordinate of its position\\
double getDistance(int otherMoteID) & The euclidean distance from another mote\\
void turnOn()                       & Turn this mote on\\
void turnOff()                      & Turn this mote off\\
boolean isOn()                      & Is this mote on?\\
void move(double x, double y)       & Mote mote (x,y) from its current position\\
void moveTo(double x, double y)     & Move mote to (x,y)\\
byte[] getBytes(String variable)    & Get a variable\\
long getLong(String variable)       & Get a variable\\
int getInt(String variable)         & Get a variable\\
short getShort(String variable)     & Get a variable\\ \hline
\end{tabular}
\caption{Mote Methods}
\label{table:motes}
\end{table}

The final four functions ({\tt getBytes()}, {\tt getLong()}, {\tt
getInt()}, and {\tt getShort()}) allow you to request variable values
from TOSSIM. The string parameter is the variable's C name. For nesC
components, this takes the form of $<component name>$\$$<variable
name>$. For example, this code fetches some variables from the TimerM
component of mote 2,:

\begin{verbatim}

tail = motes[2].getByte("TimerM$queue_tail")
head = motes[2].getByte("TimerM$queue_head")
print "head: ", head,  "tail: ", tail

\end{verbatim}

The {\tt getBytes()} function can be used to fetch entire
structures. Currently, the variable parameter does not support
accessing structure fields, pointer traversals, or array
elements. Therefore, while {\tt getBytes("TimerM\$mTimerList")} will
return all of the timer structures as an array of bytes, \\
{\tt getBytes("TimerM\$mTimerList[1]")} and
{\tt getBytes("TimerM\$mTimerList[1].ticks")} do not work.


\section{Complex Scripting}

The previous sections described how to control and monitor TOSSIM through
\name. However, the examples given were fairly simple. By using some
of Python's expressive power, one can write much more intricate and complex
scripts.

\subsection{Event Handlers}
\label{sec:handlers}

Many \name primitives depend on being able to handle events sent by
TOSSIM, which update the known state about the simulation. For
example, every time a mote transmits a packet, TOSSIM sends an event
to {\tt SimDriver}.\name subscribes to packet event notifications and
uses them to update the {\tt packets} variable (discussed in
Section~\ref{sec:packets}).

Scripts can also register event handlers through the {\tt interp}
object. Event handlers must be functions with a single parameter, the
event handled. Scripts register handlers with {\tt
interp.addEventHandler()}. The first parameter is the event handling
function, the second, optional parameter is an instance of an event
type. Passing an event type will register the handler for only events
of that type; omitting it will have it handle all event types. For example:

\begin{verbatim}

def print_handler(event):
  print event

# Print all events
interp.addEventHandler(print_handler)

# Print only LED events
from net.tinyos.sim.event import LedEvent
interp.addEventHandler(print_handler, LedEvent())

\end{verbatim}

The {\tt net.tinyos.sim.event} package contains all of the events that
TOSSIM generates (as well as a few which TinyViz uses internally). The
TOSSIM events are:

\begin{table}
\centering
\begin{tabular} {|l|l|} \hline
Name                    & When? \\ \hline
ADCDataReadyEvent       & When a mote receives an ADC reading\\
DebugMsgEvent           & Each enabled {\tt dbg} statement\\
RadioMsgSentEvent       & When a mote finishes sending a radio packet\\
SimulationPausedEvent   & In response to a pause command; refer to Section~\ref{sec:pausing} \\
TossimInitEvent         & When \name connects to TOSSIM \\
UARTMsgSentEvent        & When a mote finishes sending a UART packet \\
VariableResolveEvent    & In response to a variable resolve command \\
VariableValueEvent      & In response to a variable request command \\ \hline 
\end{tabular}
\caption{TOSSIM Events}
\label{table:events}
\end{table}

Event handlers can be useful for waiting for a condition, or for
logging information. For example, if you want to examine why a mote
sends a specific (perhaps malformed) packet, you can register a
handler for send events, and when the handler detects the packet,
pause TOSSIM.

\subsection{Future Actions}
\label{sec:pausing}

Most of the events described in Table~\ref{table:events} are issued
when a mote performs a certain action. TOSSIM also provides an event
for external programs to receive a callback in the future, the
SimulationPausedEvent. TOSSIM issues a SimulationPausedEvent in
response to a SimulationPauseCommand, which one can issue with {\tt
comm.pauseInFuture()}. This function takes two parameters; the first
is time (in 4MHz simulator ticks), while the second is an integer ID.
The integer ID allows a program to distinguish pause events when it
has issued several concurrently. Scripts can obtain globally unique
pause IDs with {\tt comm.getPauseID()}.

Here is an example use of these pause events, where a user wants to
print out mote state one minute into simulation:

\begin{verbatim}
from net.tinyos.sim.event import SimulationPausedEvent

id = comm.getPauseID()

def time_handler(event):
  global id
  if event.get_id() == id:
    print "Printing motes 1-20"
    for i in range(0, 20):
      print motes[i]

interp.addEventHandler(time_handler, SimulationPausedEvent())
comm.pauseInFuture(60 * 4000000, id)
\end{verbatim}

To help manage some of the event and id bookkeeping, the {\tt simutil}
package provides a simpler python function interface to accomplish the
same task:

\begin{verbatim}
import simutil

def time_handler():
  print "Printing motes 1-20"
  for i in range(0, 20):
    print motes[i]

simutil.call_in(60 * 4000000, time_handler);

\end{verbatim}

Note that all the pause and event identifiers are managed by the
wrapper function. In addition, note that the first parameter to {\tt
simutil.call\_in} is an offset from the current time, so the above
call will occur 60 seconds in the future. This is in contrast to both
{\tt comm.pauseInFuture} and the simple wrapper {\tt simutil.call\_at}
which both take a absolute time value.

\subsection{Periodic Actions}
\label{sec:periodic}

SimulationPausedEvents allow a script to take actions in the future
without having to block using {\tt comm.waitUntil()}; this way, they
can easily be doing multiple things simultaneously. Event handlers
can be made to reschedule themselves, for periodic actions:

\begin{verbatim}
def time_handler(event):
  global id
  if event.get_id() == id:
    print "Printing motes 1-20"
    for i in range(0, 20):
      print motes[i]
    # Reschedule to happen again in one minute
    comm.pauseInFuture(event.getTime() + 60 * 4000000, id)
\end{verbatim}

This can get the job done, but is clunky, verbose and difficult.
Changing the period from once a minute to twice a minute requires a
new function. In order to just print out mote state every thirty
seconds, a user has to write a function, request a pause ID, register
a handler, and schedule an event to go off. The {\tt simutil} package
provides a convenience class, {\tt Periodic} for exactly this reason:

\begin{verbatim}
from simutil import Periodic

def time_handler(event):
  print "Printing motes 1-20"
  for i in range(0, 20):
    print motes[i]

p = Periodic(60 * 4000000, time_handler)

\end{verbatim}

Because Python functions are first class values, we can use closures
to dynamically generate functions with the parameters a user needs.
Thus telling \name to print out a range of mote states periodically
can be done with a single function call:

\begin{verbatim}
from simutil import Periodic

def print_state(low_mote, high_mote, period):
  def time_handler(event):
    print "Printing motes ", low_mote, " to ", high_mote
    for i in range(low_mote, high_mote):
      print motes[i];
  return Periodic(period, time_handler);

p = print_state(3, 5, 60 * 4000000);

\end{verbatim}

{\tt print\_state} returns the newly allocated {\tt Periodic} object
instance, whose {\tt stop} method can be used to stop the recurring
action. Once {\tt print\_state} is called, the \name environment will
print out the state of motes {\tt low\_mote} to {\tt high\_mote},
every {\tt period} ticks.

Closures can also be used for periodic commands to TOSSIM. For
example, a routing protocol may need periodic beacons from a PC to a
mote over the UART:

\begin{verbatim}
from simutil import *
def uart_beacon(mote, period, msg):
  def sendMessage(event):
    comm.sendUARTMessage(mote, 0, msg)
  return Periodic(period, sendMessage)
\end{verbatim}

\subsection{Behavior Objects}

Using closures as described in Section~\ref{sec:periodic} allows
scripts to start periodic actions in a concise and simple manner. One
use of this is to implement mote behaviors. For example, to make a
mote move in a random walk, all one needs is a handler that moves a
mote a small random step, then schedule this event to go off
periodically.

One problem that emerges is conflicting behaviors. For example, one
could call {\tt random\_walk()} to make a mote move randomly, and also
call {\tt move\_to()} to make it move towards a specific destination.
By registering two event handlers, the script has specified two
conflicting movement behaviors, which will interleave and come into
conflict.

A clean solution to solve this problem is through the use of a Python
object. The object has functions for each of the different behaviors,
and stores state on what behavior a mote is currently following. 

From an abbreviated example adapted from the {\tt simutil} package:

\begin{verbatim}
class MoteMover:
  handlers = {}

  def moveTo(self, moteID, step, x, y, arrivedCallback = None):
    if (self.handlers.has_key(moteID)):
      raise IndexError, "Mote ID %d already on the move" % moteID 

    dx = x - mote.getXCoord();
    dy = y - mote.getYCoord();
    distance = mote.getDistance(x, y);
    nsteps = distance / step;

    def callback(pauseEvent):
      distance = mote.getDistance(x, y);
      if (distance < step):
        mote.moveTo(x, y);
        self.stop(mote); # clear handlers, cancel event, etc
        if (arrivedCallback != None):
          arrivedCallback(mote)
      else:
        mote.move(dx / nsteps, dy / nsteps);

    periodic = Periodic(rate, callback);
    self.handlers[moteID] = (periodic, 'move_to');

  def randomWalk(self, mote, step):
    moteID = mote.getID();

    if (self.handlers.has_key(moteID)):
      raise IndexError, "Mote ID " + moteID + " already on the move"

    r = Random();
    
    def callback(pauseEvent):
      x = r.nextDouble() * step * 2.0 - step;
      y = r.nextDouble() * step * 2.0 - step;
      motes[moteID].move(x, y)
      
    periodic = Periodic(rate, callback);
    self.handlers[moteID] = (periodic, 'random_walk')

  def stop(self, mote):
    moteID = mote.getID()
    if (self.handlers.has_key(moteID) == False):
      raise IndexError, "Mote ID % not moving" % moteID

    (periodic, what) = self.handlers.get(moteID);
    periodic.stop();
    del self.handlers[moteID]

\end{verbatim}

The class variable {\tt handlers} is used to store the notion of
whether or not a particular mote is moving. Conflicting movement
patterns are therefore detected and an error is reported. Additional
movement patterns can be easily added to the  object as new functions
that follow the same pattern to avoid conflict.

\section{Conclusions}

\name is a new system that adds a powerful new tool to the sensor
network developer's portfolio. Through both predefined scripts and
interactive console sessions, \name aids the tasks of developing,
testing, and evaluating a new algorithm or application. The core
architecture is extensible, allowing developers to write new python
modules and {\tt SimDriver} plugins to add new forms of interaction
and manipulation.

\newpage
\section{Appendix A: Design Ideology}

The primary goal of \name is to offer the sensor network developer a
simulation environment with dynamic interactivity, enabling both
unattended simulation experiments, as well as interactive debuggging
and simulation control. The confluence of these two goals informs the
major design decisions of our project, as well as the particular
interfaces exposed by the \name commands and classes.

In some senses, the first order design question relates to the value
of a dynamic simulation environment. Indeed, TOSSIM alone is a
valuable tool for aiding sensor network development, offering a
scalable simulation of a network of sensor motes with a fairly
realistic radio model and a rich debugging capability (gdb). Yet
TOSSIM alone essentially just simulates tossing a set of motes into a
field and letting them go, assuming a constant radio topology. On the
other hand, the real world is a dynamic place; objects and motes can
move, radio connectivitity changes, motes can fail. An important tool
in the developer's toolbox, therefore, is the ability to simulate
these dynamic interactions and thereby engineer a program that can
cope with these situations.

Rather than integrating dynamically controlled interfaces to TOSSIM,
the simulator instead implements a network protocol that allows
interactive applications to connect to and control a running
simulation. This separation allows the TOSSIM code base to remain
insulated and relatively lean (maintaining performance capabilities),
while more complicated calculations and interactivity can be
implemented in a separate process.

TinyViz is an existing tool that enables developers to dynamically
manipulate a simulation. The protocol between TOSSIM and TinyViz
enables the GUI to introduce dynamics into a test application's
execution. However, GUI elements do not address all the needs of a
dynamic interaction. In general, a user's interaction with a GUI is
non-deterministic, making repeated executions of a test case difficult
if not impossible to reproduce. Furthermore, to run a set of
experiments using manual manipulation of the simulation state is an
extremely cumbersome task, limiting the developer to a simple cursory
exploration of the potential interactivity parameter space.

The TinyViz plugin system is an available avenue to manipulate or
visualize the parameters of an application. The plugin API allows a
Java object to be loaded into the TinyViz GUI environment and to
interact with the TOSSIM protocol. Through custom plugins, an
application writer could fully express control over the dynamics of a
particular experiment. In point of fact, much of the core \name
functionality is implemented using the TinyViz plugin system. However,
writing custom plugins is an insufficient solution due not only to the
cumbersome nature of writing experiments in a compiled language, but
more importantly that it does not enable interactive code execution.
In addition, being able to run an unattended simulation is a valuable
convenience to developers.

Through a scripting environment, developers are able to control
experiments through repeatable interactions. The Python and reflected
Java commands/objects expose the key control elements of the
simulation environment. These hooks can be used both in a controlled
experiment framework and through console interaction with a running
simulation. The interative session is both useful for dynamic
debugging and investigation of an experiment, but also as a
prototyping arena for code that may become part of a longer experiment
framework. The dynamic console functionality not only eases the burden
of writing simulation scripts, but also enables interactive
experimentation with the simulation itself. A developer can pause a
simulation at a given time, use the variable resolution features to
probe around (and potentially alter) the simulation state, then
continue the simulation to observe the effects of the actions.

\section{Appendix B: \name Reference}

%% someday maybe this will be autogenerated from the javadoc... not too likely

\subsection{simcore package}

The {\tt simcore} package is the base interface between scripts and
\name. It consists of the following reflected Java classes:

\begin{raggedright}

\vspace{12pt}
\begin{tabular}{|ll|l|}\hline
\multicolumn{3}{|l|}{\textbf{class Command}}\\
\hline
void 			       && \\
sendRadioMessage&(short moteID,	& Send a fabricated Radio {\tt Message} instance to the \\
&long time,                     & given {\tt moteID} at the specified time. \\
&Message msg)                   & \\
\hline
void 			       && \\
sendUARTMessage&(short moteID, 	& Send a fabricated UART {\tt Message} instance to the \\
&long time,                     & given {\tt moteID} at the specified time. \\
&Message msg)                   & \\
\hline
void 			       && \\
turnMoteOn&(short moteID,	& Turns the specified {\tt moteID} on at the given time\\
&long time) 			& \\
\hline
void 			       && \\
turnMoteOff&(short moteID,	& Turns the specified {\tt moteID} off at the given time\\
&long time) 			& \\
\hline
void 			       && \\
setADCValue&(short moteID,	& Sets the value of the ADC sensor at the given {\tt port} \\
&		long time,	& of the specified {\tt moteID} to a new {\tt value} \\
&		byte port,	& \\
&		short value)	& \\
\hline
void 			       && \\
setSimRate&(double rate) 	& Adjusts the rate of simulation. Analogous to the {\tt -l}  \\
			       && TOSSIM command line option. \\
\hline
void 					&& \\
setLinkBitErrorProbability&(short src,	& Sets the radio link bit error rate from {\tt src} to {\tt dest} \\
&			long time,	& to the new value of {\tt loss}. Note that {\tt} loss is a bit error\\
&			short dest,	& probability, not a packet loss probability. \\
&			double loss)	& \\
\hline
double					&& \\
packetLossToBitError&(double packetLoss) & As per above, converts a desired packet loss rate into\\
				       && the appropriate bit error rate. \\
\hline
double 				       && \\
distanceToPacketLoss&(double distance) & Returns the packet loss for a given distance from the \\
				       && current radio model. \\
\hline
void 			       && \\
pauseInFuture&(long time,	& Schedules a {\tt SimulationPausedEvent} to be delivered \\
&		int pauseID)	& at the specified {\tt time}.\\
\hline
VariableResolveEvent 	       && \\
resolveVariable&(short moteID,	& Returns an event with the address and length of the variable\\
&		String name)	& specified by {\tt name}, or \{0, -1\} if it cannot be resolved.\\
\hline
VariableValueEvent 		&& \\
requestVariable&(long addr,	& Returns {\tt length} bytes from TOSSIM memory at {\tt addr}.\\
&		short length)	& \\
\hline
void 			       && \\
setDBG&(long dbg) 		& Enables the specified {\tt dbg} flag.\\
\hline
int 			       && \\
getPauseID()& 			& Returns a unique identifier for use in {\tt pauseInFuture}.\\
\hline
\end{tabular}

\vspace{12pt}
\begin{tabular}{|ll|l|}\hline
\multicolumn{3}{|l|}{\textbf{class Interp}}\\
\hline
int					&& \\
addEventHandler&(PyFunction callback) 	& Register the specified {\tt callback} to be called with events.\\
					&& Returns a unique identifier for the event handler. \\
\hline
int					&& \\
addEventHandler&(PyFunction callback, 	& Same as above, but only deliver events of the specified\\ 
&PyJavaClass eventclass)		& Java class {\tt eventclass}\\
\hline
int					&& \\
removeEventHandler&(int id)		& Remove a registered event handler with identifier {\tt id}.\\
\hline
\end{tabular}


\vspace{12pt}
\begin{tabular}{|ll|l|}\hline
\multicolumn{3}{|l|}{\textbf{class Mote}}\\
\hline
int				&& \\
getID()&			& Return the mote id.\\
\hline
String				&& \\
getCoord()&			& Return a string of the mote location in the form ``(x, y)''.\\
\hline
double				&& \\
getXCoord()&			& Return the X coordinate of the mote.\\
\hline
double				&& \\
getYCoord()&			& Return the Y coordinate of the mote.\\
\hline
String				&& \\
toString()&			& Return a string depiction of the mote state.\\
\hline
double				&& \\
getDistance&(int moteID)	& Return the distance to another mote, specified by {\tt moteID}.\\
\hline
double				&& \\
getDistance&(double x, 		& Return the distance to the specified location.\\
&double y)			& \\
\hline
void				&& \\
turnOn()&			& Power up the mote.\\
\hline
void				&& Power down the mote.\\
turnOff()&			& \\
\hline
boolean				&& \\
isOn()&				& Return the state of whether or not the mote is on.\\
\hline
void				&& \\
setLabel&(String label, 	& Sets a descriptive string {\tt label} to be displayed if the TinyViz\\
&int xoff, 			& GUI is running. {\tt xoff} and {\tt yoff} specify the coordinate offsets\\
&int yoff)			& from the mote's position that the string should be displayed.\\
\hline
void				&& \\
move&(double dx,		& Move the mote by the given {\tt dx} and {\tt dy} offsets.\\
&double dy)			& \\
\hline
void				&& \\
moveTo&(double x,		& Move the mote to the absolute position given by {\tt x}, {\tt y}\\
&double y)			& \\
\hline
byte[]				&& \\
getBytes&(String var)		& Return the mote variable named by {\tt var} as a byte array.\\
\hline
long				&& \\
getLong&(String var)		& Return the mote variable named by {\tt var} as a long integer.\\
\hline
int				&& \\
getInt&(String var)		& Return the mote variable named by {\tt var} as an integer.\\
\hline
short				&& \\
getShort&(String var)		& Return the mote variable named by {\tt var} as a short integer.\\
\hline
byte				&& \\
getByte&(String var)		& Return the mote variable named by {\tt var} as a byte.\\
\hline
\end{tabular}


\vspace{12pt}
\begin{tabular}{|ll|l|}\hline
\multicolumn{3}{|l|}{\textbf{class Radio}}\\
\hline
String				&& \\
getCurModel()&			& Return the name of the current radio model.\\
\hline
void				&& \\
setCurModel&(String modelname)	& Set the current radio model.\\
\hline
void					&& \\
setScalingFactor&(double scalingFactor)	& Set the distance scaling factor.\\
\hline
double				&& \\
getLossRate&(int senderID, 	& Get the packet loss rate between the specified motes.\\
&int receiverID)		& \\
\hline
void				&& \\
setLossRate&(int senderID, 	& Get the packet loss rate between the specified motes.\\
&int receiverID, 		& \\
&double prob)			& \\
\hline
void				&& \\
printLossRates()&		& Dump out the current loss rate graph.\\
\hline
void					&& \\
setAutoPublish&(boolean autoPublish)	& Set a flag indicating whether or not the radio should \\
					&& automatically propagate model changes due to motes moving\\ 
					&& to TOSSIM.\\ 
\hline
void				&& \\
updateModel()&			& Force an update of the internal radio model connectivity graph.\\
\hline
void				&& \\
publishModel()&			& Publish the current connectivity graph to TOSSIM.\\
\hline
\end{tabular}

\vspace{12pt}
\begin{tabular}{|ll|l|}\hline
\multicolumn{3}{|l|}{\textbf{class Sim}}\\
\hline
void				&& \\
pause()&			& Pause the simulation.\\
\hline
void				&& \\
resume()&			& Resume the simulation.\\
\hline
void				&& \\
stop()&				& Stop the simulation.\\
\hline
long				&& \\
getTossimTime()&		& Return the ``current'' time in TOSSIM time units. This is actually \\
				&& the time that the last TOSSIM event was received by the system. \\
\hline
void				&& \\
exit&(int errcode)		& Exit the process.\\
\hline
void				&& \\
setSimDelay&(long delay\_ms)	& Set the delay parameter. \\
\hline
void				&& \\
dumpDBG&(String filename)	& Start spooling TOSSIM DBG messages to the given {\tt filename}\\
\hline
void				&& \\
stopDBGDump()&			& Stop dumping out DBG messages.\\  
\hline
\end{tabular}

\end{raggedright}

\newpage
\subsection{simutil package}

The {\tt simutil} package is a python package that implements a some
richer functional additions to much of the {\tt simutil}
functionality. It provides the following python classes and functions:

\begin{raggedright}

\vspace{12pt}
\begin{tabular}{|ll|l|}\hline
\multicolumn{3}{|l|}{\textbf{Functions}}\\
\hline
call\_at&(when,			& \\
&callback,			& Schedules an event to call {\tt callback} with optional {\tt args}\\
&args = None)			& at time {\tt when} in the future. Returns the event ID.\\
\hline
call\_in&(delay,		& \\
&callback,			& Same as {\tt call\_at}, except the call takes place in {\tt delay}\\
&args = None)			& ticks past the current time. Retuns the event ID.\\
\hline
call\_cancel&(id)		& Cancels the previously scheduled call.\\
\hline
\end{tabular}

\vspace{12pt}
\begin{tabular}{|ll|l|}\hline
\multicolumn{3}{|l|}{\textbf{class Periodic}}\\
\hline
Periodic&(interval,		& \\
&callback,			& The class constructor interacts with the event system\\
&args = None,			& to schedule {\tt callback} to be called with optional {\tt args} \\
&call\_immediate = 1)		& repeatedly every {\tt interval} ticks. The {\tt call\_immediate} option \\
				&& controls whether the first call is at time 0 or waits for an interval.\\
\hline
void &&\\
stop()&				& Stops the periodic call. \\
\hline
boolean &&\\
is\_stopped()&			& Returns whether or not the object is stopped. \\
\hline
\end{tabular}

\vspace{12pt}
\begin{tabular}{|ll|l|}\hline
\multicolumn{3}{|l|}{\textbf{class MoteMover}}\\
\hline
MoteMover()&			& Creates a {\tt MoteMover} instance.\\
\hline
void &&\\
moveTo&(mote,			& Move the given {\tt mote} object in increments to the ({\tt x, y}) location. \\
&step, 				& Call the {\tt callback} (if any) when it gets there. The {\tt step} \\
&x, y, 				& parameter controls how far to move on each interval, The {\tt rate} parameter\\
&callback = None, 		& controls the frequency of movement, -1 means use the class default. \\
&rate = -1)			& \\
\hline
void &&\\
randomWalk&(mote,		& Similar to {\tt moveTo}, though move the mote in a random direction \\
&step, 				& at each tick. The {\tt step} and {\tt rate} parameters have the same \\
&rate = -1)			& meaning. \\
\hline
void &&\\
stop&(mote)			& Stop the given mote's movement. \\
\hline
boolean &&\\
isMoving&(mote)			& Return whether or not the given mote is moving. \\
\hline
void &&\\
setDefaultRate&(rate)		& Set the default movement rate. \\
\hline
\end{tabular}

\end{raggedright}


\end{document}

