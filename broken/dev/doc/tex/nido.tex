\documentclass[10pt]{article}

\usepackage{graphicx}
\usepackage{graphics}
\usepackage{multicol}
\usepackage{epsfig,amsmath,amsfonts}

\makeatletter                                   % Make '@' accessible. 
\oddsidemargin=0in                              % Left margin minus 1 inch.
\evensidemargin=0in                             % Same for even-numbered pages.
\marginparsep=0pt                               % Space between margin & text

\renewcommand{\baselinestretch}{1}              % Double spacing

\textwidth=6.5in                                % Text width (8.5in - margins).
\textheight=9in                                 % Body height (incl. footnotes)

\topmargin=0in                                  % Top margin minus 1 inch.
\headsep=0.0in                                  % Distance from header to body.
\skip\footins=4ex                               % Space above first footnote.
\hbadness=10000                                 % No "underfull hbox" messages.
\makeatother                                    % Make '@' special again.


\begin{document}

\fontfamily{cmss}                               % Make text sans serif.
\fontseries{m}                                  % Medium spacing.
\fontshape{n}                                   % Normal: not bold, etc.
\fontsize{10}{10}                               % 10pt font, 10pt line spacing 

\title{Nido System Description}
\author{Philip Levis, Nelson Lee}
\maketitle

\fontfamily{cmr}                                % Make text Roman (serif).
\fontseries{m}                                  % Medium spacing.
\fontshape{n}                                   % Normal: not bold, etc.
\fontsize{10}{10}                               % 10pt font, 10pt line spacing

\section*{System Overview}

Nido\footnote{Spanish; nest (as in an insect nest)} is a discrete
event simulator that allows users to run and debug TinyOS applications
on a PC. Experimental results show it to scale well up to networks one
thousand motes. An external communication system allows other tools to
monitor network traffic and inject packets into the simulated
network. Nido also allows finely grained configuration of debugging
output at runtime.

Current general purpose network simulators are inadequate for
TinyOS. Traditional network simulators are usually focused on protocol
simulation for large area networks in which there is a large
connectivity disparity in regions of the network (e.g. backbones
vs. LANs). For example, {\tt ns-2} simulates a network at packet
granularity and has detailed node and link specification. In contrast,
Nido is intended for simulating TinyOS and its behavior; the
inacessibility of sensor motes means some form of debugging is
invaluable. Additionally, Nido simulates the TinyOS networking stack
at bit granularity, allowing the testing of protocols at layer 2 and
above.

Nido itself provides run-time information in the form of configurable
text output. By setting the {\tt DBG} environment variable, a user can
enable or disable classes of output messages. There are also several
tools that communicate with Nido to inject packets and display network
traffic.

\section*{Getting Started}

Nido is automatically built when you compile an
application. Applications are compiled by entering an application
directory (e.g. {\tt /apps/Blink}) and typing {\tt
make}. Alternatively, when in an application directory, you can type
{\tt make pc}, which will only compile a simulation of the application.

The Nido executable is named {\tt main.exe}, and resides in {\tt
build/pc}. It has the following usage:


{\tt main.exe [-h|--help] [-r $<$static|simple|space$>$] [-ro
=$<$portnumber$>$] [-ri =$<$portnumber$>$] [-e $<$file$>$] [-p $<$sec$>$] <num\_nodes>}

The {\tt -h} or {\tt --help} options print out a more verbose usage
message, specifying which dbg modes are enabled in the simulator, etc.

The {\tt -r} option specifies the radio model to use. None of the
current radio models take into account interference or noise; signals
are perfectly transmitted and received. The {\tt simple} model places
all of the motes in a single radio cell; every mote can hear every
other mote. The {\tt static} model reads in a file at startup ({\tt
cells.txt}) that specifies which motes can hear each-other. The
format of this file is defined at the end of this document. The {\tt
space} model is a rudimentary model of 2D space. Motes are randomly
placed at startup, then move in a random, Brownian motion-like
fashion. Signal reception is based on distance between motes, with
potentiometer settings mapping to a distance table that has no basis
in reality. The default model is {\tt simple}.

The {\tt -ro} option pauses the simulator at startup, waiting for an
output connection on the port specified {\tt portnumber} or by default
port 10577. This is primarily used for SimGUI, so that SimGUI
can connect to the simulation before it begins and log
every message sent over the network.

The {\tt -ri} option pauses the simulator at startup, waiting for an
input connection on the port specified {\tt portnumber} or by default
port 10576. This allows for pre-startup injection of packets with SimGUI. 

The {\tt -p} option allows simulation text output to be more easily
understood. If it is specified, then the simulation pauses at every
system clock (not radio clock) event for the specified number of
seconds. Since system clock events are often major instigators of mote
behavior, this allows a user to make the simulation output be broken
up enough to be read, instead of the standard (far too fast to read)
constant stream.

The {\tt -e} option is for named EEPROM files. If {\tt -e} isn't
specified, the logger component stores and reads data, but this data
is not persistent across simulator invocations: it uses an anonymous
file. If {\tt -e} is specified, the simulator uses the specified
file. This allows logger data to be persistent across invocations of
the simulator, and also allows algorithms based on logger data to be
easily compared. The file grows as needed, and is initially a sparse
file (if mote 1000 writes a single byte to its last EEPROM line in an
otherwise empty EEPROM, only one disk block is allocated). The Nido
logger component resembles the mica hardware; it is 512KB in size.

The {\tt num\_nodes} option specifies how many nodes should be
simulated. A compile-time upper bound is specified in {\tt
/apps/Makerules}. The standard TinyOS distribution sets this value to be
1000.

\section*{dbg}

Nido provides configuration of debugging output at run-time. We have
added debugging statements to most TinyOS system components. Every
debugging statement is accompanied by one or more modal flags. When
the simulator starts, it reads in the DBG environment variable to determine
which modes should be enabled. Modes are stored and processed as
entries in a bit-mask, so a single output can be enabled for multiple
modes, and a user can specify multiple modes to be displayed. For
example, the statement


\begin{verbatim}
dbg(DBG_CRC, ("crc check failed: \%x, \%x\n", crc, mcrc));
\end{verbatim}


will print out the failed CRC check message if the user has enabled
CRC debug messages, while


\begin{verbatim}
dbg(DBG_BOOT, ("AM Module initialized\n"));
\end{verbatim}


will be printed to standard out if boot messages are enabled. DBG
flags can also be combined, as in this example:


\begin{verbatim}
dbg(DBG_ROUTE|DBG_ERROR, ("Received control message: \
                           lose our network name!.\n"));
\end{verbatim}


This will print out if either route or error messages are enabled.

There are a set of bindings from ASCII strings to debug modes. For
example, setting DBG to {\tt boot} will enable boot message (by
setting the appropriate bit in the mask). Setting DBG to {\tt
boot,route,am} will enable boot, routing and active message output.

This system allows a user to reconfigure the type of output displayed
on each run of the simulator without needing to modify source files
and recompile. There are several modes reserved for applications or
temporary testing use, which can be safely used when writing new
components. There are also modes that provide output on the internals
of the simulator, such as memory allocation or the event queue. When
TinyOS is compiled for mote hardware, all of the debug statements are
removed through redefinition of the macro; we have verified that they
add no instructions to the TinyOS code image.

\section*{The GUI}

The Nido visualization GUI is included in the default TinyOS release
as a Java application. Currently, the simulation only supports radio
packet messages (incoming and outgoing). It resides in the {\tt
tools/java/} directory of the default TinyOS release, and can be run
by typing {\tt java net.tinyos.sim.SimGUI} when in that
directory. It has two optional parameters, {\tt -r}, for specifying a
file of radio messages to introduce to the simulation, and a {\tt -p}
option which specifies which port to connect to and read packets from.

\subsection*{File Format}

SimGUI allows a file of radio packets to be specified at the
command line. These packets are read in and then sent over port
10576. The file has the following format:

\vspace{0.1in}
$<$time (decimal long long)$>$ $<$mote (decimal short) $<$data0 (hex byte)$>$ ... $<$data36 (hex byte)$>$.
\vspace{0.1in}

Each value is separated from the other by whitespace. All whitespace
is insignificant. A sample file exists at {\tt tools/radio.txt}.

\section*{The NetworkInjector}

This GUI provides a convenient mechanism for dynamically injecting
packets into the simulation. It resides in the {\tt tools/java/}
directory. NetworkInjector connects to Nido using port 10579 by
default if no port number is specified.

\section*{Nido Internals}

Currently, Nido compiles by using alternative implementations of
some core system components (e.g. {\tt Main.td}, {\tt HPL.td}) and
incorporating a few extra files (e.g. {\tt event\_queue.c}). 

The basic Nido structure definitions exist in {\tt
platform/pc/sim.h}. These include the global simulation state
(event queue, time, etc.) and individual node state. Hardware settings
that affect multiple components (such as the potentiometer setting)
currently must be modeled as global node state. {\tt sim.h} also
includes a few useful macros, such as {\tt NODE\_NUM} and {\tt
THIS\_NODE}; these should be used whenever possible, so minimal code
modifications will be necessary if the data representation
changes. 

{\tt Nido.td} is where the Nido global state is defined. The {\tt
main()} function parses the command line options, initializes the
Nido global state, initializes the external communication
functionality (defined in {\tt external\_comm.c, PCRadio.h}), then initializes the
state of each of the simulated motes with {\tt call StdControl.init()} and {\tt
call StdControl.start()}. It then starts executing a small loop that handles
events and schedules tasks. Currently, tasks are un-interruptible and
execute instantaneously (in simulation time). We hope to model tasks
more accurately soon.

Nido catches SIGINT (control-C) to exit cleanly. This is useful when
profiling code.

\subsection*{Nido Architecture}

Nido replaces a small number of TinyOS components, the components
that handle interrupts and the {\tt Main} component. Interrupts are
modeled as discrete events. Normally, the core TinyOS loop that runs
on motes is this:


\begin{verbatim}
while(1){
  TOSH_run_task();
}
\end{verbatim}

{\tt TOSH\_run\_task} runs tasks until the task queue is empty, at which
point it puts the CPU to sleep. An interrupt will wake the mote. If
the interrupt has caused a task to be scheduled, then that task will
be run. While that task is running, interrupts can be handled.

The core Nido loop is slightly different:

\begin{verbatim}
for (;;) {
  while(TOSH_run_next_task()) {}
  if (!queue_is_empty(&(tos_state.queue))) {	
    tos_state.tos_time =
       queue_peek_event_time(&(tos_state.queue));
    queue_handle_next_event(&(tos_state.queue));
}
\end{verbatim}



A notion of virtual time (stored as a 64-bit integer) is kept in the
simulator (stored in {\tt tos\_state.tos\_time}), and every event is
associated with a specific mote. Most events are emulations of
hardware interrupts. For example, when the clock in {\tt hpl.c} is
initialized to certain counter values, Nido's implementation
enqueues a clock interrupt event, whose handler calls the function
normally registered as the clock interrupt handler; from this point,
normal TinyOS code takes over (in terms of events being signaled,
etc.) In addition, the event enqueues another clock event for the
future, with its time (for the event queue) being the current time
plus the time between interrupts.

 
When compiling for Nido, the nesC compiler modifies code generation
for components such that fields declared in components result in
arrays of each field.  The maximum number of motes that can be
simulated at once is set at compile time by the size of this array;
the default value is 1000 and is specified in {\tt /apps/Makerules}
with the {\tt -fnesc-tossim-tosnodes=} flag. Since every event is
associated with a specific mote, the {\tt tos\_state.current\_node}
field is set just before the event handler function is called, so that
all modifications to component fields affect the correct mote's
state. Since all TinyOS tasks enqueued by an event are run to
completion before the next event is handled, tasks all execute with
the same node number and access the enqueuing mote's state.

\subsection*{RFM}

Instead of requiring a certain level of radio simulation accuracy,
Nido uses an extensible radio model. By doing so, the simulation can
be run with any level of accuracy a user wants (and is willing to
write). While simple protocol correctness tests might be satisfied
with a model that assumes perfect signal transmission, protocol
performance tuning tests might want to introduce bit error and
transient interference.

The interface to the radio model is defined in {\tt rfm\_model.h}:

\begin{verbatim}
typedef struct {
  void(*init)();
  void(*transmit)(int, char);
  void(*stop\_transmit)(int);
  char(*hears)(int);
} rfm_model;
\end{verbatim}

The data pointer is provided so that radio models that require a large
amount of state can only allocate the memory if they're being used;
otherwise, if the state is comprised of global variables in source
files, the process uses up space for every model's bookkeeping.

The {\tt int} arguments are mote IDs, while the {\tt char} arguments
are bits (0 or 1).

{\tt init} is called at simulation startup. {\tt transmit} is called
whenever a mote calls the transmit command on the lowest level radio
abstraction. {\tt stop\_transmit} is called on every radio interrupt
to clear status if the mote was previously transmitting. {\tt hears}
is called on every radio interrupt when the radio is in receive state,
and returns the value that is heard. The {\tt data} pointer is
provided so that large data structures can be allocated dynamically at
{\tt init}; otherwise, if the simulation includes many models, a lot
of memory could be wasted.

We have implemented several radio models; we will describe a simple
one that we call {\bf static}. When the {\tt init} method of the
static model is called, the simulator tries to open a file named {\tt
cells.txt} that defines an undirected network connectivity graph in
terms of mote IDs. {\tt init} reads and builds this graph. If no file
is found, every mote is made adjacent to every other. The model stores
two pieces of state for every mote: whether it is currently {\bf
transmitting }a one, and a {\bf radio\_active} value. Both values are
initialized to zero in {\tt init}.

When a mote transmits a 1, its {\bf transmitting} is set to true, and
{\bf radio\_active} of every mote adjacent to it in the graph is
incremented. If {\bf transmitting} is true when {\tt stop\_transmit}
is called, {\bf radio\_active} of every adjacent mote is
decremented. When a mote {\tt hears}, it tests {\bf radio\_active}; if
greater than zero, the mote hears a bit. This assumes that two motes
transmitting at the same time will never cancel each-other's
signals. The increment/decrement element is necessary to determine
when the last of the transmitters a mote can hear has finished.

The static model has many shortcomings; it assumes perfect signal
propagation (there are no bit errors) and the connectivity graph
doesn't change (hence the name). We have also implemented a simplistic
spatial model that handles mote movement and transmit distances; motes
begin at random positions and move in a random walk
manner. Connectivity for a mote is recomputed at each radio clock
event. Our intention is not to provide realistic world models that
account for the complexities of RF propagation, instead, to provide a
framework for users to develop models of complexity as fine or coarse
as their work needs; we believe the models we have developed show the
feasibility of more realistic models.

Nido simulates the radio using the mica 40Kb stack, {\tt
MicaHighSpeedRadioM}. {\tt MicaHighSpeedRadioM} distributes
functionality across several components. Thus more components had to
be replaced, as opposed to the old radio radio stack where only {\tt
RFM} had to be replaced. Hooks into higher level components for the
external communication system were added, but their logic remained
unchanged.

\section*{ADC and Spatial Models}

Nido also has extensible ADC and spatial models. The provided
implementations of these models return identical data on every call;
every ADC port has a value of zero, and all motes are always at the
same coordinates. This is the interface to a spatial model:

\begin{verbatim}
typedef struct {
  double xCoordinate;
  double yCoordinate;
  double zCoordinate;
} point3D;

typedef struct {
  void(*init)();
  void (*get_position)(int, long long, point3D*);   // int moteID,
                                                    // long long time
                                                    // point3D* storage
} spatial_model;
\end{verbatim}

This is the interface to an ADC model:

\begin{verbatim}
typedef struct {
  void(*init)();
  short (*read)(int, char, long long);  // int mote, char port, long long time
} adc_model;
\end{verbatim}

The global {\tt tos\_state} structure holds a pointer to all of the
three models: rfm, spatial and adc.

The spatial model is separate from the other two so that they can
easily share state and can provide correlated readings.

\section*{External Communication}

IO is a major limitation in debugging sensor networks. If a network
has a complex topology (e.g. motes can be several network hops from
one another), gathering real-time information on network traffic can
be difficult. For example, a single mote might be producing packets
with corrupted application data due to a bug. These corrupted packets
could then have a ripple effect through the network. Neither of the
two options to deal with this kind of bug are very appealing.

One can incorporate network traffic logging on the motes, storing
packets for later retrieval. The difficulty arises when correlating
the traffic observed by multiple motes; doing so requires global time
synchronization, which is not incorporated into the motes by default,
and adding it might make the bug disappear, just as adding logging might. 

Alternatively, one could add a number of motes that solely listen and
log traffic. These monitor motes could be connected to PCs, allowing
real-time observation of traffic. However, if the network is composed
of many transmit cells (many hops), one must use a large number of
monitor motes, or at least move them around.

Instead, Nido allows a global view of the network that is distinct
from what the motes hear. It sends data on network and UART traffic
over TCP sockets, including AM-level packet transmissions as well as
bit transmissions. The former is useful for application debugging,
while the latter can be used to debug data link level encoding and
decoding. When Nido starts, it attempts to connect to ports on the
local machine, each of which is associated with a certain kind of IO
data. If a connection attempts fails, Nido does not retry to connect
and doesn't output that data during its run.

Nido also opens a server socket that runs in a separate
thread. Connections made to this socket allow a user to dynamically
inject packets into the network; each packet has a time stamp and a
mote ID; that mote will receive the packet at that time, bypassing all
of the protocol stack below the AM component (an event is enqueued
whose handler signals the event that tells AM a packet has been
received). If the timestamp is set for a point in the past, the packet
is received at the current time in the simulator (the time of the next
event in the queue).

Nido waits for a TCP connection request on default or specified ports
for message injection and output before it commences the mote boot
sequence -- this synchronous behavior ensures that initial packets
sent will not be missed and that the simulations involving injected
packets can be reproduced.

Nido also includes functionality for introducing and reading network
traffic real-timne. This is also achieved through the use of TCP
sockets that are opened before the simulation begins. Nido initiates a
connection to localhost on a series of ports. Each of these ports,
their use, and their communication format are detailed below. In
addition, Nido always opens two server (passive) port and waits for
external connections; client sockets connecting to one port can inject
messages into the network, while the client sockets connecting to the
other port will receive packets sent in the network.

\subsection*{UART}

\subsubsection*{Reading: Port 10580}

The UART communication channel allows an external process to
communicate with the motes over their UARTs. At startup, the Nido
process reads from the socket until the server side closes it. Data
sent over the socket must have the following format:

\vspace{0.1in}
\begin{tabular}{|c|c|c|}\hline
\hspace{4in} & \hspace{1in} & \hspace{0.5in} \\ \hline
Time (long long)& Mote ID (short) & Data(char) \\ \hline
\end{tabular}
\vspace{0.1in}

These messages cannot be generated dynamically; they are all read in
before the simulation starts. Once the server process has sent all of
its messages, it closes the connection to Nido.

{\bf UART communication is currently not implemented.}

\subsubsection*{Writing: Port 10581}

UART messages are written using the same format with which they are
read. On startup, the simulator opens a connection to port 22577 on
localhost and send messages over this channel.

{\bf UART communication is currently not implemented.}

\subsection*{Radio}
\subsubsection*{Reading: Port 10576}

The radio communication channel allows an external process to
communicate with the motes over the radio. At startup, Nido opens a
server connection on port 10576 waiting for SimGUI to connect to
it (this functionality specified by the {\tt -ri} option when running
Nido). Once connected, Nido reads from the socket until no more data is
read. Data sent over the socket must have the following format:

\vspace{0.1in}
\begin{tabular}{|c|c|c|}\hline
\hspace{2in} & \hspace{2in} & \hspace{0.5in} \\ \hline
Time (long long)& Mote ID(short) & TOS\_Msg (36 bytes) \\ \hline
\end{tabular}
\vspace{0.1in}

These messages cannot be generated dynamically; they are all read in
before the simulation starts. The simulation stores them as
events. Sending messages in this manner bypasses the radio layers in
the mote, directly calling {\tt Nido\_received(\&(data-$>$msg))}, which
signals {\tt ReceiveMsg.receive(packet)} in {\tt AMStandard.td}.

Once the client has sent all of its messages, it closes the connection
to Nido.

This mechanism allows for the initialization of base stations at
simulation startup. Since all motes must run the same code image, base
station logic must be incorporated into the build of TinyOS. A message
handler can be used to receive messages that switch a mote from
standard to base-station mode, and messages of this type can be
introduced into the simulator through this mechanism.

\subsubsection*{Writing: Port 10577}

This socket notifies an external program whenever a mote starts to
transmit a packet. This notification occurs when the mote completes
the transmission; if there is a lot of radio traffic, this might be
significantly after it attempts to start transmitting. The simulator pauses before mote boot sequence for a client to
connect (this of course is specified with the {\tt -ro} option when running Nido). The packet is sent along the socket in this format:

\vspace{0.1in}
\begin{tabular}{|c|c|c|}\hline
\hspace{2in} & \hspace{2in} & \hspace{0.5in} \\ \hline
Time (long long)& Mote ID(short) & Data (36 bytes) \\ \hline
\end{tabular}
\vspace{0.1in}

SimGUI is the primary application that uses this feature of the simulator.


\subsubsection*{Real-time Reading: Port 10579}

Nido spawns a thread at startup that opens a server socket and waits
for connections on port 10579. Clients connecting to this port can
inject messages dynamically into the network. Data sent over this
socket should have the following format:

\vspace{0.1in}
\begin{tabular}{|c|c|c|}\hline
\hspace{2in} & \hspace{2in} & \hspace{0.5in} \\ \hline
Time (long long)& Mote ID(short) & TOS\_Msg (36 bytes) \\ \hline
\end{tabular}
\vspace{0.1in}

If the time field specifies a time that has already passed in the
simulation, the message is scheduled to be received immediately. If
the time field specifies a time in the future, the message is
scheduled to be received at the specified time.

If the client closes the connection, Nido will wait for another
connection request.

A GUI for injecting packets is included in the TinyOS release: {\tt
net.tinyos.tossim.NetworkInjector}. It uses the {\tt
net.tinyos.packet} package to provide a packet type specific GUI. If
you want to add a new packet type, merely write a new packet class and
include it in the {\tt main} of {\tt NetworkInjector}. 

\subsubsection*{Real-time Writing: Port 10584}
Nido spawns a thread at startup that opens a server socket and waits
for connection on prot 10584. Clients connecting to the port will
receive packets sent in the following format:

\vspace{0.1in}
\begin{tabular}{|c|c|c|}\hline
\hspace{2in} & \hspace{2in} & \hspace{0.5in} \\ \hline
Time (long long)& Mote ID(short) & Data (36 bytes) \\ \hline
\end{tabular}
\vspace{0.1in}

If the client closes the connection, Nido will wait for another
connection request.

SimGUI is the primary application that uses this feature of the simulator.

\subsubsection*{Bit Writing: Port 10578}

Radio messages are written to a connection made to this port. The
messages have the following format:

\vspace{0.1in}
\begin{tabular}{|c|c|c|}\hline
\hspace{4in} & \hspace{1in} & \hspace{0.5in} \\ \hline
Time (long long)& Mote ID(short) & Bit (byte) \\ \hline
\end{tabular}
\vspace{0.1in}

The bit states whether the mote transmits a 1 or doesn't transmit
(0). The mote is considered to be producing the same value until
another message saying otherwise arrives.

This functionality is useful if debugging activity below the packet
layer, such as encodings/start symbols.

\section*{cells.txt}

There are currently two network connectivity models in Nido. The simple model places every mote within a single cell; all of the motes can hear one another. This is the default network model. The second network model, the static model, uses a network connectivity graph that is determined at run-time before the simulation starts. The graph is specified by a file named {\tt cells.txt} which must be in the directory where the program is executed. The file format is very simple.

\begin{verbatim}

0:1 1:2 4:5 9:1

\end{verbatim}

defines a graph in which motes 0 and 1 can communicate, 2 and 1 can communicate, 4 and 5 can communicate, and 9 and 1 can communicate. Connectivity is bidirectional: {\tt 0:1} is the same as {\tt 1:0}. A sample file can be found in the root TinyOS directory.

\section*{GDB}
One of the most useful purposes of Nido, if not the most useful, is
that the binary executable produced by typing {\tt make pc} in any app
directory can be debugged using GDB. Developers of TinyOS code can now step through
programs they have written, thereby debugging deterministic, logical
aspects of their code before loading their programs onto motes.

When accessing variables and functions in Nido under GDB, specific
prefixes precede them. For instance, if two components had a {\tt
uint8\_t count} variable, how would one using GDB print or set each value?
To answer this question, it is first necessary to explain how the nesC
compiler works. In essence, it takes each source file, combines it
into one {\tt C} file, and compiles that single file into an
executable. The nesC compiler takes each component's fields and functions
and renames them with unique identifiers. This renaming for fields of
a component is done by
taking the field name, and preceding it by the component name and a
{\tt \$} sign. Functions are renamed by taking the name of the
function and preceding it by the name of the component it is defined
in, followed by a {\tt \$} sign, followed by the interface
name, followed by another {\tt \$} sign.

At compile time, the maximum number of nodes that can be simulated is
specified.  This value, as mentioned in section
{\tt Nido Architecture} specifies the size of the array for each component's fields during code
generation of the compilation process. When compiling for the {\tt
pc}, each declaration of a component field is followed by {\tt [}, the
value of {\tt tossim\_num\_nodes}, and {\tt ]}, where tossim\_num\_nodes
is specified by the flag {\tt -fnesc-tossim\_tosnodes=} in {\tt
Makerules}, or by default, set
to 1000.  This is the way Nido simulates multiple
motes. Nido keeps an array for each variable of state, and references
each mote's corresponding state when running code for that mote.

Furthermore, during code generation for the {\tt pc}, all references to component
fields are followed by {\tt [}, {\tt tos\_state.current\_node} and {\tt
]}, where {\tt tos\_state.current\_node} refers to the moteID of
the mote Nido is currently executing code for.

Using the application {\tt Blink} as an example, type {\tt make pc} in
the {\tt /apps/Blink/} directory to compile the application for Nido. Run {\tt gdb} on the executable
produced by typing {\tt gdb build/pc/main.exe}. Examining {\tt
BlinkM.td}, to break at the function Clock.fire(), type {\tt b
*BlinkM\$Clock\$fire}. {\tt b} specifies a breakpoint in GDB, and the
{\tt *} tells the GDB command line parser to examine the entire block of text
following it as a single expression. It is currently necessary to
specify the {\tt *} before the function name for GDB to parse it
properly. {\tt BlinkM} denotes the component the function is defined
in, and {\tt Clock} specifies the interface the function {\tt fire}
belongs to.

To run {\tt Blink} with one mote, type {\tt r 1}.  {\tt r} is the GDB
command to run the current executable, and all values after the {\tt
r} are passed as arguments to the executable. Therefore the {\tt 1}
specifies the number of motes to simulate to Nido. MoteIDs are
assigned starting at 0.  In this current example the moteID of the
single mote being simulated is 0.  

Upon hitting the breakpoint at {\tt fire}, to print the value of {\tt
BlinkM}'s {\tt state}, type \\{\tt p
BlinkM\$state[tos\_state.current\_node]}. {\tt p} is the print command
in GDB. {\tt BlinkM} corresponds to the component the field {\tt
state} belongs to. Since the application was compiled for the pc,
it is necessary to specify which mote's {\tt BlinkM\$state} to
access, as {\tt [tos\_state.current\_node]} following the field name does. 


\end{document}





