\documentstyle[11pt,a4]{article}
\begin{document}

\begin{titlepage}
SimX: Simulation eXtensions for Tossim/T2.

Paul Stickney // 2008-2009

\end{titlepage}

\section{Overview}

SimX is a collection of Simulation Extensions for Tossim/T2.
(Tossim/T2). Tossim/T2 is the defacto TinyOS 2.x discrete-event
simulation environment. The tools SimX provides are designed to be
minimally evasive to the Tossim (sim) and Tossim-Live (sim-sf) TinyOS
build targets: no modifications are required of either Tossim or
Tossim-Live in order to use Simx.

\begin{itemize}
\item Minumum invasion
\item Discrete components
\end{itemize}

\subsection{Disclaimer: Run Real Tests}

Simulators, including Tossim, can only be as accurate as the model
that they are based on. As Chad Metcalf demonstrated, in his
Tossim-Live paper, Tossim can render an accurate representation of a
real environment if it is constructed from data collected from the
original WSN environment. However, using tools like the USC link-gain
generator (or DynTopo), may lead to idealized models which do not
adequately represent real-world sitations. {\em All simulation testing
  should be backed by tests conducted in a real WSN.}

\subsection{Separation: Don't Cheat}

The ideology behind SimX is to provide useful extensions while
maintaining a separation of concern between simulation layers: the
simulation (run in Tossim kernel), the simulation environment (SimX,
code that creates and maintains the environment), and external
applications (such as MViz). Each component strives to only expose the
data available in a real environment.

{\em In no case should motes or ``real world'' applications be given
  more information than an be expected in a real scenario.}

While SimX tries to establish informal barriers that decrease the
chance of ``leaking information'', the levels of seperation must be
maintained by the users of SimX to avoid pollution of the simulation
environment.

During the course of this document the following terms will be used:
simulation, simulation environment (or just environment), and external
applications (or applications).

\paragraph{Simulation}
The simulation engine, Tossim, runs TinyOS code using an event-driven
simulation environment with virtualized hardware interfaces. The
simulation operates under the parameters and configuration provided by
the simulation environment.

\paragraph{Simulation Environment}
The simulation environment is composed of a set of modules, including
SimX, and test framework code that is responsible for defining the
parameters of the simulation environment. This includes link-gain
information, noise models, sensor input, and environmental
interaction, if any.

\paragraph{External Applications}
External applications may communicate with motes, but only insofar as
provided by the motes themselves, such as when one or motes function
as a gateway through a serial connection. For testing purposes,
external applications may also passively monitor the
environment. However, {\em any interaction with the environment is
  discouraged} as the relationship is only provided through the
simulation environment and is not idicative of a real WSN.

\section{Modular Design}

The base SimX modules can be used directly in a Tossim
environment. The modules provided increased control on the simulation
environment and are written in a combination of Python, nesC and C++,
Java, and Scala, according to need. While there are dependencies
between some of the base modules, {\em most base modules can be used
  as stand-alone entities or entity-groups in Tossim TinyOS projects}.

Base modules with nesC components are designed to be easy to link into
existing Tossim projects and require only minimal changes to build
configurations. Normally, each nesC-SimX module builds its own shared
library (such as \_SimxPushbackmodule.so) and links it against the
compiled Tossim module (\_TOSSIM.so). Using this approach enables a
high degree of customization, allows efficient caching of compiled
code, and does not require modifications to existing sim/sim-sf TinyOS
targets.

Since conversion of an application to support Tossim may require some
modifications, the minimal changes are acceptable. Furthermore, guards
can be used to maintain non-simulated and simulated code in the same
codebase.

For example, to add Sensor support to a Tossim project one simply has
to copy in the Makefile.simx file, modify if needed, and use that as
the compilation target.

\subsection{SimxBase}

Dependencies: (none)
nesC: no
Python: yes

SimxBase is a Python wrapper around the Tossim Python model. It
provides a mechanism to cleanly decorate nodes with extensions,
performs some small caching, and adds useful utility
functions. Because it is a wrapper for Tossim, it exports all the
initial functionality of the Tossim object.

The features provided by SimxBase are used as a consistent foundation
for other support, such as sensor readings and a model with dynamic
gain calculation.

Using tossim base is very simple and consistent.

``
from TOSSIM import Tossim
from simx.tossim import TossimBase
base = TossimBase(Tossim[])
``

``
Fetching a single node:

# prefered method
node0 = base[0]

# just as in Tossim
node0 = base.getNode(0)

# Tossim.methodCalls have a base.method_call alias for consistency
node0 = base.get_node(0)

# TossimBase can also fetch multiple nodes
node2, node3, node4 = base[2:5] # start-index=2, end-index=5

node0, node1 = base[xrange(0, 2)]
node42, node17 = base[[1,2,3]]
nodes = base.get_nodes([1,2,3])

Turning on nodes is just like in Tossim:

node.turn_on() # or node.turnOn()

# A node can be turned on and booted at the specified time
# This is a short-hand for node.turn_on(); node.bootAtTime(time);
node.turn_on(time)

node.turn_off() # or node.turnOff()
``

Because of the internal caching semantics of TossimBase nodes should
be connected using direct access.

``
Connecting nodes

node0.connect(node1, gain)
node0.connect_both(node1, gain)
node0.disconnect(node1, gain)
node0.disconnect_both(node1, gain)

# List radio connections [Node, Gain]
node0.connections()
# Returns list of neighbor nodes
node0.neighbors()
``

The radio method of TossimBase returns a radio proxy that will
correctly dispatch to the correct node.

Notes: Since TossimBase performs caching it is important that only the
Node objects returned by TossimBase are used. Methods should never be
called directly on the original Tossim object or radio.


\subsection{Probe}

The Probe component allows efficient read and write access to Tossim
variables using zero-copy mapped buffers. It uses a small C/SWIG
extension to create the PyBuffer objects. Data marshalling, buffer
caching, and structure parsing/manipulation is performed in
Python. The Probe module replaces TossimApp, TossimNescDecls and
Tossim.Variable for variable-access in the Tossim environment.

Probe is primarily a Python module which reads in the nesC-generated
``app.xml'' file. After the declarations are read and the Tossim
environment is initialized, Tossim variables can be read from and and
written to. In addition to supporting transparent access through
Python data structures, Probe also has the ability to dynamically
re-define data-types based on composition of existing types. This
allows for operations such as treating a ``message\_t'' buffer as both
a raw TOS message data-structure and a application-specific payload at
the same time.

Probe is also designed to support other Simx components, most notably
Simx/Watch. Thus, Probe provides a method of quickly determining which
variable data may have changed which can be used to significantly
reduce the amount of expression checks required.

Steps to using Probe.
\begin{itemize}
\item Enable building the Probe module from the TinyOS 2 application
\item Load Tossim and the Probe module
\item Load the app.xml file with Probe
\item Lookup a Tossim variableitem
\item Bind the variable to one or more motes
\end{itemize}

For efficiency purposes, resolved Probe definitions (returned from
ProbeLoader.lookup()) and bound Probe objects (returned from bind())
should be kept until no longer needed. Resolving and binding Probes
can be rather costly operations.

Example \#1:

import probe.parser.ProbeLoader
\# Load definitions
loader = ProbeLoader(``app.xml'')
\# Bind the variable from Mote \#1
var1 = loader.lookup(``OscilloscopeC\$buffer'').bind(1)
\# Set the first byte of the header of the message\_t type
var1[``header''][0] = 1

Example \#2 (combining structural types): 

Access payload of message\_t as a specific kind of structure (an
oscilloscope\_t structure in this case).

loader = ProbeLoader(``app.xml'')
def = loader.lookup(``OscilloscopeC\$buffer'',
                    (``[data]'', ``oscilloscope\_t''))
var = def.bind(1)
print var[``payload''][``version'']


Advantages over Tossim.Variable:

\begin{enumerate}
\item Can modify Tossim variables. This could also be considered a
  disadvantage.
\item Easier to use. The entire structure of a variable, including any
  nested structures (structs or arrays), is exposed in a simple
  manner. Nested primitives are accessible by direct access/assignment
  of item slots. No additional marshalling is required to access the
  contents of a variable.
\item Can dynamically re-define structure formats to meet application
  requirements through composition of existing types.
\item Memory-backed buffers avoid making copies of data and
  significantly reduce the amount of C/SWIG calls required. This also
  reduces the complexity of the C code required to marshall the data
  and moves more control into the Python layer.
\item Data marshalling handled by standard ``struct'' Python module.
\item Integrated support for variable watching. A single call to
  C/SWIG can scan a list of Probe objects to identify which
  variables have changed.
\end{enumerate}

Advantages over TossimApp/TossimNescDecls:

Probe was written with variable access from within Tossim exclusively
in mind. It strives to focus on easy of use from within a Tossim
environment.
\begin{enumerate}
\item Streamlined interface. Only ``essential'' support/access
  provide. Some extra features found in TossimNescDecls have been
  dropped entirely.
\item Faster parsing. Simple tests show a reduction from 2.3 to 1.3
  seconds.
\item Supports ``combination'' of structure types.
\end{enumerate}

Limitations/missing features/disadvantages:

\begin{enumerate}
  \item Only variables which are wrapped by Tossim can be accessed.
    These are limited to global variables located in TinyOS components
    which have not been removed by optimization. This limitation also
    applies to Tossim.Variable and is imposed by how Tossim operates.
  \item While the structure of variables and data-types is kept, all
    non-primitive type information, including structure names, is
    dropped during the decomposition. Probe is designed to work with
    raw structures, not simply definitions of
    structures. TossimNescDecls keeps much more information.
  \item Only variables and associated structures are supported. No
    constants, enums, functions, interfaces or components are handled
    directly.
  \item There is no support for pointers or zero-length arrays.
  \item There is no support for unions (this may be added).
  \item Probe and Probedef objects may use significantly more memory
    due to the structural decomposition model employed.
  \item Buffers and shadow buffers are cached (avoids duplication) but
    never released.
\end{enumerate}


\subsection{Pushback: Uniform callbacks}

The Pushback component is a nesC/SWIG/Python module which provides a
uniform mechanism for establishing callbacks between compiled nesC
code in Tossim and the Python/C++ simulation environment.

While not normally used directly, this mechanism is used as the
underlying transport for such modules as SimxSensor and SimxTxPower
modules. By employing a dynamic dispatch, Pushback is flexible,
reduces code duplication, and can easily support future callback-based
extensions.

The Pushback component is not dependent upon any other SimX modules.

How it works:

The Tossim environment sets up callbacks. Each callback is given a
particular name. When the node code is executed it may invoke a
callback by using the SimxPushback nesC module.

\subsection{Sensor: Custom sensor readings}

SimxSensor supplies the SimxSensorC nesC component and supporting
Python classes. SensorC supports the {\em Read} interface and can
address 256 independent read channels. Each channel can be configured
to read in 8, 16, or 32 bit sizes. There is also a per-channel
read-delay that may be specified as a fixed value, a sampled value, or
the result of a function.

A Python module, SensorControl, is also provided simplify channel
management.

Included Python helper functions are available to stream data from
plain-text files, fixed-width binary files, and SAC (Seismic Activity
Code) files. In addition, a fully customizable sine-wave generator is
provided to feed sensor input with easily verifiable test data.

The following nesC instantiates two sensors. SensorA is a 32-bit sensor on channel 0 and SensorB is a 16-bit-wide sensor on channel 2.
``
components
  new SimxSensorC(uint32_t, 0) as SensorA,
  new SimxSensorC(uint16_t, 2) as SensorB;
``

Using the sensor module is equally as easy.

``
from simx.sensor import SensorControl, SensorExtension

# base has already been setup as a TossimBase wrapper object
base.register_extension(SensorExtension(SensorControl(SimxSensor())))

node0 = base[0]
node0.connect_sensor(0, funcA, 1)
node0.connect_sensor(0, funcB, 1)

# disconnect a sensor (future readings will result in failures)
node0.disconnect_sensor(0)
``

Both the read functions and delay functions take three
arguments... the delay function is invoked as soon as Read.read of the
SimxSensor component is requested and the read function is request
after the delay and is immediately used as a value to
SimxSensor.readDone (Read).

In addition the sensor Python modules provide a simple way for connect
some common sources of channels and some function generators.

The Sensor module relies on the Pushback module.


\subsection{Sync: High-fidelity time control}

Sync, or Tossim sync, allows a simulation to be run in ``world
time''. That is, the simulation can be throttled to run no faster than
the real WSN would operate. Or, optionally, the ratio of simulation
time to real-world time may be configured.

Sync provides an advanced version of the time-throttle support
provided by the Tossim Live extensions. Since Sync actively examines
the forward event queue it has a higher time accuracy than the Tossim
Live extension. Peeking forward allows for a much more accurate
synchronization approach, especially as the frequency of events
decreases.

Additionally, TSync correctly manages the Python GIL (Global
Interpreter Lock) to support multi-threaded Python applications. Some
SimX features, such as Inject, require multi-threading support.


\subsubsection{Event: Run at precise times}

The Event module is a pure-Python component that provides convenience
methods and helpers to run events at a precise simulation time. This
mechanism can be used to model external events and event propagations.

The Event module also supports configuration of code to execute
immediately after the node is booted. This provides a safe and easy
method of configuring sensors and other per-node information.


\subsection{Inject: Create SF streams}

{\em The standard Tossim-Live Throttle will not work with Inject.}
Inject requires threading support to work. Use Sync as an advanced
alternative to the standard Tossim-Live Throttle.

The Inject module provides a thread-safe mechanism to create
additional Serial Forward (SF) streams. The primary purpose is to
allow communication between the Act and React modules but it may also
be used as any generalized SF transport stream. Inject SF streams can
also act as a ``piggyback'' by bridging an existing SF connection.


\subsection{DynTopo: Dynamic Topology Manager}

\emph{While a dynamic topology can be used to create an interative
  environment, it is crucial to realize that this environment is only
  as ``real'' as the contrived forumals used to generate it.}

DynTopo is a pure Python extension that allows for the creation and
maintenance of topologies with dynamically adjusting link-gain
models. It achieves this by maintaining a virtual node location and
applying the USC link-gain formula. Noise generation is not currently
supported.

DynTopo also provides functions to create topologies (grids, ellipses,
lines, random scatters, etc.) as well as to save and load generated
topologies and link-gain models.


\subsubsection{TxPower: Simulate TX Power}

{\em TxPower depends on DynTopo}

TxPower is a nesC/Python module that allows motes to change their
transmission power. Since Tossim itself only works off of a supplied
link-gain model, this module requires a dynamic topology manager, such
as DynTopo, to rebuild the simulation model after a power change.

The TxPower module relies on the Pushback module.


\section{Act/React: Tossim Visualization Environment}

\subsection{Overview}

The Act and React modules provided a visualization environment for
Tossim.  In this way, Act/React, along with the base SimX components,
serve as a replacement for TinyViz in TinyOS 1.x. Act/React are highly
dependant upon the base SimX components to operate. The exact
requirement varies based on configuration and required functionality.


\subsection{Separation: Again}

As with the rest of the SimX environment, Act/React strive to keep the
simulation, simulation environment, and application layers
separate. However, {\em It is up to Act/React users to ensure that
  cross-layer contamination does not occur.}


\subsection{Basic Operation}

Act is a server that runs the Tossim simulation and provides the
necessary bindings for remote access and is responsible for {\em
  executing normal python commands in the running environment}.

React is a Java-compatible client (the source is written in Scala)
that communicates with the Act server to inspect and modify the
simulation environment. The primary React interface is the Console
which allows execution of remote commands on an Act server. Additional
visualization components are also provided.


\subsection{Visual Components}

React strives to provide a nice graphical user interface to interact
with a running Act server and comes with several standard interfaces
and Act extension modules. The ability for each React visual component
to perform is dependent upon the corresponding Act module to be
configured. A minimal Act configuration may not support all React
components.

{\em All SimX base components can be used directly within Act/React --
  React simply provides a nice interface.} That is, simulation sensors
can be connected, TX power and topologies can be simulated and
generated, event call-backs can be added, etc. The use of React on top
is simply to provide a nice interaction environment and aid in
simulation interaction.


\subsubsection{Console}

The console allows direct execution of arbitrary Python commands
inside the Act environment. The output area displays results, errors
and debugging information. In addition, a history is provided to
recall previously sent commands.


\subsubsection{Template Manager}

Templates allow commands and operations to be saved for future
execution.  The templates can be directly executed or inserted into
the current Console buffer.  Templates can also be easily distributed
between different React clients.

Default templates are included which show a wide range of operations
such as node manipulation.  These templates can be used, modified, and
combined as needed to perform a wide range of tasks.  Additionally,
since the templates are merely python code, they may be used directly
inside the test harness.


\subsubsection{TimeControl}

The time control component uses the TSync module to run the simulation
``in real time'' as well as providing support to pause and resume the
simulation. It shows an approximation of the current simulation time,
the execution status, and allows the simulation to be stopped,
started, or run until a specific simulation time (absolute-time or
delta-time).

{\em TimeControl requires the use of TSync and Event}


\subsubsection{Topology}

The topology component provides a zoom-able user interface which
visualizes nodes and links in a synthetic network created by the
DynTopo component. The topology control also allows a limited form of
direct mote interaction.

{\em Topology visualization and control requires DynTopo.}


\subsubsection{Watch Variables}

{\em This is obsolete. The newer Probe interface is not complete yet.}


\subsection{Details}

\subsubsection{Client/Server}

Act/React use a client-server model to interact. This is significantly
different approach than TinyViz for TinyOS 1.x and has both advantages
and disadvantages.

\begin{enumerate}
\item Distributed environment. Act and React can run on separate
  machines as long as a SF TCP connection can be established. However,
  performance depends upon the speed of the connection.
\item Compatible with plain Tossim/T2 C++ bindings. Unlike TinyViz, no
  additional Jython / Java bridge is required.
\item Can connect multiple clients. A React server can connect to
  control an environment while an external application can connect to
  collect comparison data.
\item Separation of contexts. By design, the client/server approach
  force clear protocols to be followed. This makes it more difficult,
  but not impossible, to cross-pollute some layers.
\item Increase in overhead and additional transport layer. The
  communication protocol is more expensive than the Jython Java
  bindings and requires an explicit API including protocol
  maintenance.
\end{enumerate}

\subsubsection{Communication}

{\em This is changing!}  The old SF-based socket with the MIG
interface has become hard to maintain and, while flexible, suffers
from some internal problems.  The newer design is based to work off of
Google Protocol Buffers, but these to do not have sufficient library
support for the IO component.

All communication is done over SF (Serial-Forward) compatible
sockets. Messages between Act and React are based on MIG-generated
structures. Some messages utilize a variable-length encoding. The
maximum length of message data is 64*1204 bytes in size. The messages
are further encoded into valid SF message frames with an AM type
specific to Act. Each encoded message frame has no more than 240 bytes
of payload. A single Act or React message may take multiple SF
messages to complete.

Communication with the Act server occurs through Command
messages. These messages contain Python code which are executed within
the context of the Act server and loaded modules. Upon completion of
the command, the Act server sends the appropriate reply messages to
React. The Act server may also send unsolicited messages, such as when
internal information has been updated. A general subscribe/publish
model is used.

\end{document}
