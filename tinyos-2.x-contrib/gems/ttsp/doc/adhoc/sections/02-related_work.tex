\section{Available Approaches to Time Synchronization}
\label{relwork}

% keywords: Traditional Clock Synchronization Protocols, Available Time Synchronization Protocols for WSNs
Traditional clock synchronization protocols commonly found in wired networks, are not suitable for WSNs. Due to the different set of challenges/constraints that are found on these networks, typically, limitations on energy, bandwidth and hardware, moreover, network connections may be unstable over time. Thus, a traditional wired clock synchronization protocol would not achieve the same performance as it does on wired networks, mainly due to the non-determinism in transmission time delays which happen at the MAC layer. Due to these reasons there was the need to develop specific protocols to achieve time synchronization in WSNs. This development, gave birth to a series of protocols, from which the most successful and representative of these are Reference-Broadcast Synchronization (RBS) \cite{Elson02-RBS}, Timing-Sync Protocol for Sensor Networks (TPSN) \cite{Ganeriwal03:TPSN} and FTSP \cite{Maroti04:FTSP}.\\
 
% keywords: RBS
RBS, protocol uses a receiver-to-receiver synchronization algorithm for pair-wise synchronization. RBS uses one sensor node to act as a beacon by broadcasting a reference packet. All receivers record the packet arrival time. The receiver sensor nodes then exchange their recorded timestamps and estimate their relative phase offsets. RBS also estimates the clock skew by using a least-squares linear regression. One of the interesting features about using a receiver-to-receiver approach is that all timing uncertainties (including MAC medium access time) on the transmitter's side are eliminated. For a network-wide synchronization, RBS uses the concept of domain clusters. A domain cluster can be seen as cluster of sensor nodes locally synchronized within a beacons range. In order for two domain clusters to communicate between themselves, the sensor nodes that belong to both of the domain clusters act as gateways between these two domains. The gateway reconciles timestamps while forwarding  messages between domains, according to their next-hop destination and its time difference to the current node.\\

% keywords: TPSN
TPSN, uses a sender-to-receiver synchronization algorithm for pair-wise synchronization. TPSN reduces the uncertainties by using timestamps at the medium access control (MAC) layer. This eliminates the uncertainties introduced by the MAC layer (e.g., retransmissions, back-offs, medium access). For network-wide synchronization, TPSN first establishes a hierarchical structure in the network and then the pair-wise synchronization is performed along the edges of this structure. This structure can be seen as a spanning tree, in which the root of the tree is the reference sensor node, to which all other sensor nodes shall synchronize with.
TPSN has two main phases. A level discovery phase, starts after the network is deployed, with the root node broadcasting its level 0 and every other immediate neighbors assigning themselves as level 1, one greater than the level they have received. This process is continued and eventually every node in the network is assigned a level, thus constructing a spanning tree. In the synchronization phase, the pair-wise synchronization algorithm is performed along the edges of the hierarchical structure earlier established, eventually synchronizing every sensor node with the root.\\

% keywords: FTSP
FTSP is tailored for applications requiring stringent time precision. FTSP uses a sender-to-receiver pair-wise synchronization algorithm by broadcasting timestamps to neighboring sensor nodes. The intrinsic delays in a sender-to-receiver algorithm are minimized by applying timestamps at the MAC layer, thus achieving a high precision performance. By using comprehensive error compensation including skew estimation, its possible to minimize the frequency that synchronization is done on the network. For a network-wide synchronization, FTSP uses an algorithm for selecting a master among the sensor nodes. This master starts to broadcast timestamps to its neighbors. Once those neighbors get synchronized with the master, they will also start to broadcast timestamps to its neighbors, thus initiating a controlled and periodic flooding over the network. If the master stops broadcasting after a certain period of time, the algorithm for selecting a master will be again initiated, thus achieving robustness over topology changes.\\

% keywords: Adaptive Time Synchronization
Regardless of the architecture and the effectiveness achieved by each of these protocols, none of these take into account
mechanisms of controlling the required precision needed by a given application/MAC layer protocol. Most of these were tailored to deliver the maximum accuracy and precision they could get on the network. The average precision error achievable by most of these protocols is quite remarkable. FTSP by itself, manages to achieve an average precision error of 0,5 $\mu$s per hop. One important aspect to consider here is, once these sensor nodes get synchronized, it is necessary to repeat the synchronization procedure in a near future, or else, this average precision error will start to raise. Mainly due to external factors that can influence the shift of the frequency that a oscillator works, like temperature, pressure and the quality of the component itself. The frequency of a synchronization round must be very high to maintain such precision errors. Eventually the needed resources to maintain such synchronization round profile are quite high. Most of the authors of these protocols are not able to justify the use for these precision errors. Most of these protocols, do not even consider the precision requirements needed by the application or the MAC layer protocol, nor their authors quantify the impact on resource consumption while trying to achieve such precision error. Clearly, forgetting about the intrinsic resource constraints that a WSN is limited to, mainly, energy constraints.