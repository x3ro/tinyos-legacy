

The whole point of using HPLPowerManagement is that you don't actually *use*
it.  You call HPLPowerManagement.enable() and then the mote will
automatically sleep when the following conditions are met:
  * the radio is off
  * all high speed clock output compare interrupts are disabled
  * spi interrupt disabled
  * task queue is empty

There is no code to "use" power management, instead each service (Timer,
Radio, SpiByteFifo, etc) calls HPLPowerManagement.adjustPower() when the
component has shut down to notify the PowerManagement component that it
should check the state of the above items again and reset the sleep
register.  Thus applications need not worry about explicitly telling the
mote to go to sleep (THIS is the main difference between HPLPowerManagement
and the *deprecated* Snooze).

The main advantage of this approach is it allows services to clean up, save
state, and shut down before the cpu is halted.  In snooze, the cpu is stolen
out from under running services without any notification to those services.
For HPLPowerManagement to work, an application must call
CC1000RadioC.StdControl.stop().  After this call, HPLPowerManagement takes
over (once the radio is done shutting down), and sets the appropriate
registers to put the mote to sleep after the task queue empties.  Note that
if using a low power radio listening state (such as
CC1000RadioIntM.setListeningMode(i)), the CC1000 radio module will take care
of notifying HPLPowerManagement to say it is okay to halt the CPU.

The mote will wake up on the next 32kHz Timer interrupt.

The long winded answer to your question, Mark, is that *only* services need
to call adjustPower() when they have finished shutting down or starting up.
It is invalid semantics for an application to call adjustPower(), however
you are not penalized by it in the current implementation.  Thus, your
application should call HPLPowerManagement.enable() in its
StdControl.init().  Power Management is disabled by default.

I believe HPLPowerManagement is documented somewhere in the TOS 1.1 release.

GDI and TASK/TinyDB both enable HPLPowerManagement.  The GDI code is in
contrib/ucb/apps/TestLabApp.  GDI uses low power radio sampling (part of the
default CC1000 radio stack) and achieves 30uA sleep state.  TinyDB uses its
own TDMA scheme for low power communications and also achieves the same low
power state.