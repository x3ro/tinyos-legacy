\section{\tinyos Concepts}

Before one can understand how to customize components or place 
components together to form an application, some central TinyOS concepts  
must first be explained.\footnote{The 
information found in this section is covered in detail in \cite{jhill-thesis}.} 

\subsection{The Execution Model}

Most modern operating systems support a multi-tasking model where the
processing is switched between tasks either because a task is waiting 
on a resource or because the task has used its allocated time quota. This
common model would not be ideal for the network sensor regime because
of the severe hardware restrictions and the near realtime requirements
of the sensor hardware. For example, managing separate stacks for
multiple threads of execution would waste precious precious memory and
time. Additionally, mote applications usually need to support fine-grained
concurrency that would exacerbate the expense of context switching. For
example, a mote application would like to quickly sample input from a
sensor, read a bit from the radio, and return to sampling input or
perhaps storing sensor data in onboard EEPROM.

\tinyos has a two-pronged execution model better suited to this 
workload: a high-priority event queue with a lower-priority task queue
allowed to run whenever the event queue is empty. \emph{Event handlers} 
are short snippets of code that run in response to an \emph{event} or 

\subsubsection{\nesc support for the execution model}

async / sync

\subsection{The Component Model}

\subsubsection{\nesc support for the component model}

\subsection{The Hardware Presentation Layer (HPL)}


