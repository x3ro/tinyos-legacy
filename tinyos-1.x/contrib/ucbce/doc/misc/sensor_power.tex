\documentclass[10pt]{article}


%%%%%%%%%%%%%%%%%%%%%%%%%%%%%%%%%%%%

\setlength{\topmargin}{0.in}
\setlength{\textheight}{8.5in}  % Use for proofreading.
\setlength{\textwidth}{6.5in}
\setlength{\oddsidemargin}{0.in}


\renewcommand{\textfraction}{.1}
\renewcommand{\bottomfraction}{.3}
\setlength{\textfloatsep}{13pt}
\setlength{\abovedisplayskip}{8pt}
\setlength{\belowdisplayskip}{8pt}

%\usepackage{endfloat}
%\usepackage{doublespace}
%%%  Use for initial submission, remove for final.
\usepackage{fancyhdr}


\lhead{D. M. Doolin}
\chead{}
\rhead{Sensor power}
\lfoot{\today}
\cfoot{}
\rfoot{\thepage}
\renewcommand{\headrulewidth}{0.4pt}
\renewcommand{\footrulewidth}{0.4pt}
\pagestyle{fancy}


\input {comment}
\usepackage{chicago}
\usepackage{draftcopy}
\usepackage{subfigure}
\usepackage{psfig}
\usepackage{graphicx}
\usepackage{amssymb}
\usepackage{amstext,amsmath}
\usepackage{latexsym}
\usepackage{mathrsfs}  %  command: \mathscr{#}
\usepackage{psfrag}
\usepackage{comment}
\usepackage{url}
\usepackage{algorithm}
\usepackage{algorithmic}
\usepackage{lscape}
\usepackage{color}


\newcommand{\eq}{Eq.}
\newcommand{\eqs}{Eqs.}
\newcommand{\tab}{Tab.}
\newcommand{\tabs}{Tabs.}
\newcommand{\fig}{Fig.}
\newcommand{\figs}{Figs.}
\newcommand{\Fig}{Fig.}
\newcommand{\Figs}{Figs.}
\newcommand{\alg}{Alg.}
\newcommand{\page}{page}


\newcommand{\dphidn}{\frac{\partial\phi}{\partial\mathbf{n}}}
%\newcommand{\pdiff}[2]{\frac{\partial{#1}}{\partial{#2}}}
\newcommand{\ab}[1]{{\ensuremath{\langle #1\rangle}}}

\newcommand{\Iota}{\ensuremath{\mathrm{I}}}
\newcommand{\gapfunc}{\ensuremath{\mathbf{g}}}
\newcommand{\lagrange}{\ensuremath{\mathcal{L}}}
\newcommand{\action}{\ensuremath{\mathcal{A}}}
\newcommand{\VE}{vertex-edge}


\newtheorem{thm}{Theorem}[section]
\newtheorem{cor}{Corollary}[section] % No numbers....  (?)
\newtheorem{dfn}{Definition}[section]
\newtheorem{example}{Example}[section]
\newtheorem{proposition}{Prop.}[section]

\newcommand{\nth}{\ensuremath{n^{\text{th}}}}
\newcommand{\mth}{\ensuremath{m^{\text{th}}}}
\newcommand{\ith}{\ensuremath{i^{\text{th}}}}
\newcommand{\jth}{\ensuremath{j^{\text{th}}}}

\newcommand{\ee}[1]{\ensuremath{\times 10^{#1}}}


\newcommand{\Rone}{\ensuremath{\mathbb{R}^1}}
\newcommand{\goesto}{\ensuremath{\to}}
\newcommand{\imples}{\ensuremath{\ \Rightarrow\ }}


%  A couple of macros...
\def\matlab{{\tt Matlab}}
\def\maple{{\tt Maple}}


%Give a little extra space in the equation arrays because
%of all the sums and fractions.
\setlength{\jot}{.2in}



%%%%%  FIXME: Move this up:
\def\smallfig{55mm}


\newcommand{\Comment}[1]{\textcolor{red}{#1}}


\pagestyle{plain}

\begin{document}


\title{Powering sensors}
%\author{D. M. Doolin\thanks{%
%Post-doctoral researcher, University of California,
%Civil and Environmental Engineering,
%2108 Shattuck Ave., Berkeley, CA 94720-1716}}
\date{\today}
\maketitle


%% I am calling this thing a sensor because it
%% sits on the same hardware bus as the others.

%\section*{Power interface}

The general case for powering sensors on a sensor
board requires handling $1,\ldots,n$ distinct
power states.  For example, the 
LeadTek 9546 GPS ``sensor'' has 4 power states,
which may be succinctly described as 
Off, Cold, Warm and Hot, which influence the 
data aquisition rate of the sensor.
There are several ways of dealing with this issue,
in particular.

% should a separate interface
%for power be defined.  


%\paragraph{What Me, Worry?}
\paragraph{Do nothing}
One way is to ignore the issue and use the StdControl 
interface Start command to power 
the sensor into its default power state.
Works fine for ON/OFF sensors.
Simplicity is the obvious advantage, rigidity
the obvious limitation.
This is what the current driver code for the 
LeadTek GPS does.  It works, sort of:  cold
start data aquisition (GPS location) requires
45 seconds, an unpleasantly long time on a device
trying to operate in the $\mu$sec to msec time
frame.

%\paragraph{The End Run}
\paragraph{Embed in specialized component}
Another way is to define an interface corresponding
directly to the hardware in question bypassing 
the definition of a power interface altogether.  
Using the LeadTek again, this would result in commands
\begin{verbatim}
interface LeadTek9546 {
   ...
   command result_t powerColdStart();
   command result_t powerWarmStart();
   command result_t powerHotStart();
   command result_t powerOff();
   // Possible power*Done events defined here
   ...
}
\end{verbatim}
The advantage to the application programmer is 
reasonably fine-grained control over the behavior
of the sensor, at the cost of learning 
he intricacies of slightly different APIs for  different 
sensors.  A disadvantage of this technique is 
overspecialization; there isn't any abstraction 
here, just a definition for a device.  (One 
could probably argue that is also an advantage,
since presumably, once drivers are written for 
some sensor of year/model, they need not 
be written again. For an analogy, consider 
CC1000Control)


%\paragraph{The (Other) End Run}
\paragraph{Power level as argument}
For certain, now historical reasons, the driver code
for the Crossbow, Inc. MTS420CA ``fireboard'' was
decomposed into it's constituent sensor parts,
each in their own directory, each with their 
own header file.  This has several advantages,
one of which allows a defining a power command
as part of a sensor interface abstraction, and 
enumerating states as arguments to that 
command.\footnote{This has actually worked fairly
well.  It allows application developers to mess 
around with data collection algorithms without 
messing around with the sensor driver code.}
For example:
\begin{verbatim}
interface Sensor {
   ...
   command result_t power(uint8_t power_state);
   event result_t powerDone();
   ...
}
\end{verbatim}
In the header file {\tt foo\_sensor.h}, 
for a sensor {\tt FOO}, one finds
\begin{verbatim}
enum {FOO_SENSOR_POWER_OFF, FOO_SENSOR_COLD_START, ...};
\end{verbatim}
An error event can be signaled if something 
unexpected transpires.  The advantage is that the 
sensor API only needs to be learned once, the 
differences being only in the arguments to the 
power command.

Likely, most sensors will have two states, on and off.
Note that this technique has not yet been, but probably will
be, implemented in the driver code for this board.
Just as soon as the proprietary byte string for 
controlling the LeadTek device is reverse-engineered.
See note following on problems with defining the ``off''
state in this kind of call.

%\paragraph{\Large{$\aleph_0$}}
\paragraph{Analogy with CC1000 radio power}
As stated, the general case requires handling
$1,\ldots,n$ distinct power states for each sensor.
$n$ is probably 2, could be up to 5 or 6, but 
possibly very many more.  An analogy could be 
drawn with the CC1000 power states:  how could 
RF power be handled with an interface, given that 
each radio has multiple, but differing number of 
power states?


\paragraph{Compromise}
Given that 
\begin{enumerate}
\item there will be a standardized Sensor interface
defined for the TinyOS core, 
\item $[$Std/Split$]$Control with semantics of 
init, start and stop will be adopted, and
\item there is no way to know beforehand how many 
power states will be needed in the future,
\end{enumerate}
the following compromise exploits the fact that 
most existing sensors driven by TinyOS code 
have simply on and off power states:

\begin{itemize}
\item Define a power(uint8\_t power\_state) command
in the Sensor interface.  
\item Define an enum (or preprocessor value) in 
a header file, perhaps sensorboard.h, perhaps 
foo\_sensor.h, for each
power state for each sensor.  
%%%%%  This seemed like a good idea when I wrote, but 
%%%%%  seems like a bad idea now.  Comments?
%\item For ``on-off'' sensors,
%map Sensor.power(ON) to Sensor.start() and 
%Sensor.power(OFF) state to Sensor.stop() at compile time.
\end{itemize}



\subsection*{Some comments from David Gay}

\begin{verbatim}

> 
> tex and pdf file outlining various schemes
> for controlling power to sensors committed
> to cvs: contrib/ucbce/doc/misc/sensor_power.[tex,pdf]

I like the discussion of power levels, current problems, etc. I question the 
assumption that we're going to stick with SplitControl - I'm going to propose to 
replace it with:

interface PowerControl
{
   /**
    * Initialize the component and its subcomponents.
    * @return Whether initialization was successful.
    */
   command result_t init();

   /**
    * Change the component's power level, the component is responsible for
    *   changing its subcomponent's power levels as appropriate
    * @param state New power state for this component. Possible values are
    *   POWER_ON: switch to default power on mode
    *   <component specific values>: some components may define additional
    *     power levels
    * @return Whether the power change request was accepted.
    */
   command result_t power(uint8_t state);

   /**
    * Notify components that the power level has been changed
    */
   event result_t powerDone();

   /**
    * Switch the component's power off, the component is responsible for
    *   switching its subcomponent's power off as appropriate
    * @return Whether power off request was accepted
    */
   command result_t stop();

   /**
    * Notify components that the power level has been changed
    */
   event result_t stopDone();
}


Rationale:
- split-phase init is not really useful - calling split-phase commands in other 
components in init is not possible as start hasn't happened yet. Better to keep 
init for internal state initilisation - start/power can do the hardware/slow 
initialisation

- mostly keep the power levels from David Doolin's proposal, define a standard 
one (on)

- powerOff (aka stop) is defined separately - this should simplify the logic in 
powerDone/stopDone event handlers

David Gay
\end{verbatim}

\paragraph{From Dave Doolin}  Gay's proposal makes a fair bit
of sense, although using ``power'' and ``stop'' instead 
of ``powerOn'' and ``powerOff''
needlessly breaks symmetry.  That is, ``stop'' what, exactly?
And who will write and maintain the documentation explaining 
that ``stop'' means turning off the power.
Using ``powerOn'' and ``powerOff'' is plain english clear.


Condensing powerOn/Off into a single call has the advantage 
of being terse and elegant, with a possible very large 
major drawback: handling completion of powerOn is different 
than handling completion of powerOff.  So rolling in the 
powerOff mode into a single power function would require 
switching code to differentiate powering on and off.


\subsection{Response to DG from David Culler}

\begin{verbatim}
The problem is that if the hardware takes a long period to initialize,
what does the processor do in the meantime?  This is a natural case for
one shots.  Kick the hardware initialization process.  Wake me up again
some time in the future.  Trigger the upper software layer know when
things are all ready.

It is not that you always need split phase, but that split-phase is the
more general case.  It means that if you replace the component with a
very different initialization requirement, the rest of the system will
already be able to deal with it.
\end{verbatim}

\paragraph{Doolin and Gay respond to Culler}

\begin{verbatim}
David Doolin wrote:
> On Wed, 25 Feb 2004, David Culler wrote:
>>The problem is that if the hardware takes a long period to initialize,
>>what does the processor do in the meantime?  This is a natural case for
>>one shots.  Kick the hardware initialization process.  Wake me up again
>>some time in the future.  Trigger the upper software layer know when
>>things are all ready.
>
>
> One way to sidestep the hardware issue is to limit
> init() of a component and subcomponents to the
> software side, using power/powerDone for hardare
> initialization.

I think that's the right way too.

David Gay
\end{verbatim}

\end{document}
