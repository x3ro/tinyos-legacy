\documentclass{article}
\usepackage{fullpage}

\parindent 0in
\parskip 0.5em

\begin{document}

\title{Nucleus Network Management}
\author{Gilman Tolle}

\maketitle

\tableofcontents
\newpage

\section{Introduction}

The purpose of a network management system for you, the developer, is
to make it easy to expose information about the functioning of your
networked applications. The purpose of a network management system for
you, the administrator, is to make it easy to gather information about
the functioning of the networked application you are monitoring. 

In your previous applications, you have probably written your own
message types, generated {\tt mig} classes, and written client-side
Java applications just to get data out of your motes. The Nucleus
system is a matched set of TinyOS components and Java tools that make
exposing and collecting network information easier.

You can use Nucleus to expose information in three different ways:

\begin{itemize}

\item Attribute Queries: Let's say that you are writing a multihop
routing component, and want to get a snapshot of the current state of
the routing graph. You can expose the address and cost of each node's
next hop and the list of potential next hops as Nucleus Attributes.
You can then inject a Nucleus Query into the network to gather the
current contents of these attributes. 

More generally, Nucleus allows the developer to associate a
human-readable name with an item of data. The item of data must have a
known constant length. An attribute can also be a list of data items
each with known constant length, and individual items can be accessed
by index. Nucleus then assigns a small integer identifier to the
attribute name, so that a compact query can be constructed. The query
can be submitted over the network or the serial connection, and the
results can be directed to the network or the serial connection.

\item RAM Queries: Without modifying your program at all, Nucleus
allows you to remotely retrieve the contents of symbols in RAM. If you
want to expose an attribute without much extra work, you can just
create a variable to store it, and use the Nucleus RAM Query to
retrieve its contents.

For Nucleus, the name of a RAM symbol is the name of the variable,
prefixed by the component name if it is a component-local
variable. The identifier of a RAM symbol is the actual location in
RAM. The length of the RAM symbol is the actual size of the memory
referenced by the symbol. The query processor just reads the data
directly from RAM when the symbol is requested.

\item Log Events: Now let's say that your routing protocol is
adaptive, and you want to study its adaptation pattern. You can expose
a Nucleus Event that contains a message like ``Node \%d changed its
parent to Node \%d, with cost \%d'', and the data necessary to make
this event meaningful. When the component changes parents you can
signal the event, which can be stored locally or sent off the node
immediately.

\end{itemize}

The Nucleus framework can save you the time and effort you would have
spent on writing your own custom messages for collecting this
data. The RAM and code overhead of Nucleus is small enough that you
can leave the data collection framework in your program even after you
have finished debugging it. Then, when your application begins to
perform strangely in the field, you will be able to gather more data
about what's going on.

\newpage

\section{Installation}

This section shows how to prepare your development machine to begin
using Nucleus.

\subsection{Obtaining the Software}

Software to get:

\begin{enumerate}
\item A working TinyOS development environment. You should already
  have this, but if you don't, go to {\tt http://www.tinyos.net} and
  follow the documentation there.
\item Cygwin, if you are using Windows.
\item The {\tt ncftp} package if you are running Cygwin. The installation of {\tt XML::Simple} depends on it.
\item {\bf nesC 1.2}: Nucleus
makes heavy use of the nesC 1.2 attribute tags and generic
components. 
\begin{enumerate}
\item Download {\tt nesc-1.2alpha??.tar.gz} from {\tt http://sourceforge.net/projects/nescc}
\item Unpack it into {\tt /opt} and change to the {\tt nesc-1.2alpha??} directory.
\item Run {\tt ./configure}
\item Run {\tt make}
\item Run {\tt make install}
\end{enumerate}
%\item {\bf nesC 1.2alpha5 Frontend}: nesC 1.2alpha5 requires
%installing some utilities from the main TinyOS source tree as well.
%\begin{enumerate}
%\item Change to the {\tt tinyos-1.x/tools/src} directory.
%\item Run {\tt cvs update -d}
%\item Change to the {\tt ncc} subdirectory.
%\item Run {\tt ./Bootstrap}
%\item Run {\tt ./configure TOSDIR=<cygwin path>/tinyos-1.x/tos}
%\item Run {\tt make}
%\item Run {\tt make install}
%\end{enumerate}
\item {\bf Perl XML::Simple}: Nucleus also depends on the {\tt
XML::Simple} Perl module. 
\begin{enumerate}
\item Run {\tt cpan install XML::Simple}
\item Press the enter key through all the prompts
\item Unblock the {\tt perl} program if Windows asks
\item When you can't press enter anymore, you have to select your
continent, country, and a few servers to access. This part is only necessary if this is the first time you have run {\tt cpan}.
\item Keep pressing enter until the rest is done.
\end{enumerate}
\end{enumerate}

Check out the {\tt tinyos-1.x} tree from CVS. 

If you already have the tree, then run {\tt cvs update -d} in these
directories:

\begin{itemize}

\item {\tt tinyos-1.x/tools/make/}, which is needed by the Nucleus
build process.

\item {\tt tinyos-1.x/beta/Drip/}, which contains the TinyOS components
and Java tools for the Drip dissemination layer.

\item {\tt tinyos-1.x/beta/Drain/}, which contains the TinyOS
components and Java tools for the Drain collection layer.

\item {\tt tinyos-1.x/contrib/nucleus/}, which contains the TinyOS
components, Java tools, and build scripts for the Nucleus system.

\end{itemize}

If you don't have these directories, run {\tt cvs update -d} in the
{\tt tinyos-1.x/beta} directory and in the {\tt tinyos-1.x/contrib}
directory.

Enable the new Make system by adding this line to your {\tt ~/.bashrc}:

\begin{verbatim}
export MAKERULES=$TOSDIR/../tools/make/Makerules
\end{verbatim}

\subsection{Compiling the Java Tools}

Once you have updated your CVS tree, you can compile the Nucleus Java
tools.

First, you need to add the Java directories to your {\tt \$CLASSPATH}
environment variable. Here is how to do this if you are running the
{\tt bash} shell. Your mileage may vary.

Insert these lines into your {\tt \$(HOME)/.bashrc}. 

\begin{verbatim}
export CLASSPATH="$CLASSPATH;<full path>/tinyos-1.x/beta/Drip/tools/java/"
export CLASSPATH="$CLASSPATH;<full path>/tinyos-1.x/beta/Drain/tools/java/"
export CLASSPATH="$CLASSPATH;<full path>/tinyos-1.x/contrib/nucleus/tools/java/"
\end{verbatim}

NOTE: You must replace the {\tt <full path>} in those lines with the
full path to your TinyOS tree. If you are running in Cygwin, you MUST
include the Windows drive specifier, like this: {\tt C:/Program
Files/UCB/cygwin/opt}. Forward slashes are preferred.

Once you have edited {\tt .bashrc}, type {\tt source \$HOME/.bashrc},
or close and re-open your terminal window.

Then, you can compile the Java tools. Run {\tt make} in each of the following
directories:

\begin{enumerate}
\item {\tt tinyos-1.x/beta/Drip/tools/java/net/tinyos/drip}
\item {\tt tinyos-1.x/beta/Drain/tools/java/net/tinyos/drain}
\item {\tt tinyos-1.x/contrib/nucleus/tools/java/net/tinyos/nucleus}
\end{enumerate}

\subsection{Compiling a TinyOS Program with Nucleus}

Now, we are going to modify an TinyOS application to compile with
Nucleus. If you are doing this for the first time, try this on the
{\tt CntToLeds} application in {\tt tinyos-1.x/apps}. You will perform
these steps for every application that you want to make
Nucleus-enabled.

Add the following lines to your application's {\tt Makefile}:

\begin{verbatim}
TOSMAKE_PATH += $(TOSDIR)/../contrib/nucleus/scripts

CFLAGS += -I$(TOSDIR)/../beta/Drip
CFLAGS += -I$(TOSDIR)/../beta/Drain
CFLAGS += -I$(TOSDIR)/../contrib/nucleus/tos/lib/Nucleus
\end{verbatim}

Make sure to place them before the {\tt include ... Makerules}
line. Also, make sure you use CFLAGS, not PFLAGS. The Nucleus build
process needs to place an include before the includes specified here,
and it depends on setting PFLAGS for this.

Now, to compile your application with Nucleus, use this command:

\begin{verbatim}
make <platform> nucleus
\end{verbatim}

The application should compile normally, with some extra messages in
the build process. 

This procedure makes it possible to compile an application with
Nucleus. The next section will show you how to actually add the
Nucleus components to your application, and how to gather data from
the application.

\newpage
\section{RAM Queries}

The simplest use of Nucleus is to retrieve RAM symbols from your
running application. 

\subsection{Preparing your TinyOS Application}

Enable the Nucleus query components by adding the following lines to
your application's top-level configuration, usually located in {\tt
<your application name>.nc}:

\begin{verbatim}
components MgmtQueryC;
Main.StdControl -> MgmtQueryC;
\end{verbatim}

Including {\tt MgmtQueryC} will also include the rest of the
components needed by Nucleus.

Then, compile and install your application on a mote:

\begin{verbatim}
make <platform> nucleus install,<nodeid>
\end{verbatim}

\subsection{Using the Java Query Tools}
\label{sec:ram-tools}

Now that your application has become Nucleus-enabled and has been
installed, you can retrieve the values of RAM symbols.

First, attach a SerialForwarder to your mote. Use the following
command:

\begin{verbatim}
java net.tinyos.sf.SerialForwarder -comm <your MOTECOM> &
\end{verbatim}

For details on the {\tt MOTECOM}, see the TinyOS Tutorial.

Nucleus provides a command-line query tool that you can use to
retrieve RAM sybols (and attributes). For now, we will be running this
tool directly, but you could also consider running it from another
script as part of a larger data-gathering system.
 
The Nucleus Query tool is called {\tt
net.tinyos.nucleus.NucleusQuery}. To test that the query tool is
correctly set up, run the following command:

\begin{verbatim}
java net.tinyos.nucleus.NucleusQuery
\end{verbatim}

Now, make an alias for it by adding the following line to your {\tt
.bashrc}:

\begin{verbatim}
alias nquery='java net.tinyos.nucleus.NucleusQuery'
\end{verbatim}

Run {\tt source \$HOME/.bashrc} or re-open your terminal window.

The tool needs to access a file called {\tt nucleusSchema.xml} that
has been generated by the build process and placed in your TinyOS
application's {\tt build/<your platform>/} directory.

For now, make sure to run the Nucleus Query tool from the
application's {\tt build/<your platform>/} directory. If you would prefer
to run the query tools from the application's main directory, then you
can add this option to the {\tt nquery} command line: {\tt -f build/<your
platform>/nucleusSchema.xml}.

To get a list of the available RAM symbols, you can open the {\tt
nucleusSchema.xml} in a text editor. This file also contains lists of
attributes and events. For the list of RAM symbols, find the {\tt
<symbols>} section. Global symbols like {\tt TOS\_LOCAL\_ADDRESS} keep
their original names and symbols within components are prefixed with
the name of the component, like {\tt Counter.state}.

Choose the name of the RAM symbol you want to retrieve.

We are going to start by querying the mote that you just
installed. This mote is directly attached to the serial port.  By
default, the Nucleus Query tool disseminates queries to the entire
network using Drip, and retrieves responses over the Drain collection
tree. Because we want to query a directly-attached mote, you run {\tt
nquery} with a few extra flags, like this:

\begin{verbatim}
nquery -s link -d serial <RAM Symbol Name>
\end{verbatim}

The {\tt -s link} flag specifies that you are injecting the query to
the mote attached to the local link. The {\tt -d serial}
flag indicates that the query response should also be sent over the
serial port.

One second later, which is the default delay, you should see results
looking like this:

\begin{verbatim}
<nodeid>: <RAM Symbol Name> = <RAM Symbol Value>
\end{verbatim}

You can also submit a query for multiple RAM symbols at the same time:

\begin{verbatim}
nquery -s link -d serial <RAM Symbol A> <RAM Symbol B> ...
\end{verbatim}

\begin{verbatim}
<nodeid>: <RAM Symbol A> = <RAM Symbol A Value>
<nodeid>: <RAM Symbol B> = <RAM Symbol B Value>
\end{verbatim}

The lack of types for RAM symbols results in some important
limitations on Nucleus RAM Query:

\begin{itemize}
\item Every RAM symbol will be interpreted as an unsigned integer. 
\item RAM symbols representing structures and arrays will be returned
in their entirety, but will not be further parsed by Nucleus Query.
\item To display a RAM symbol as a sequence of hex bytes instead of as
an integer, add the {\tt -b} flag to the {\tt nquery} command line.
You can then use other tools to reconstruct the data.
\end{itemize}

\newpage
\section{Attribute Queries}

In addition to implicitly exposing your data through RAM symbols, you
can explicitly expose a piece of data as a Nucleus Attribute. Exposing
your data as a Nucleus Attribute gives you some additional advantages
over exposing it as a RAM symbol:

\begin{itemize}
\item Your data will be exposed with a type, which can be used by the
tools to parse the bytes into a meaningful structure.
\item You can expose a list of items as a single attribute, and
retrieve individual elements of the list. This allows you to collect
things like a neighbor table or a received messages counter for each
different Active Message type.
\end{itemize}

Nucleus attributes are not tied to a specific compilation of an
application. If you are using RAM symbols and you change your
application, the location of every RAM symbol may change. Then, if
some nodes are running version 1 of your application and other nodes
are running version 2 of the application, different data items will be
returned to the same query. 

Nucleus attributes are more abstract than RAM symbols. They may be
more appropriate when you want to separate the name of the attribute
from the name of the variable, or when you want different programs to
export the same attribute.

\subsection{Exposing an Attribute in TinyOS}

This section explains what you need to change in your application in
order to expose a Nucleus attribute. Nucleus Attributes are always
exported by nesC modules. The easiest way to export an attribute is to
add it to the module that contains the data. As an illustrative
example, consider a routing component that looks like this:

\begin{verbatim}
module RoutingM {
  [uses some communication components, etc]
}
implementation {
  uint16_t currentParent;

  [has some routing logic to change that parent]
}
\end{verbatim}

We'll expose the current parent as a Nucleus Attribute. These are the
basic steps:

\begin{enumerate}
\item Decide on a name for the attribute (i.e. {\tt RoutingParent}).
\item Expose the attribute.
\item Write code to give it out when it is asked for.
\end{enumerate}

There are three changes you have to make to your module {\tt .nc}
file:

\begin{enumerate}
\item Add this line to the top of the file:
\begin{verbatim}
includes Attrs;
\end{verbatim}

\item 

Add a line to the module definition, like this:

\begin{verbatim}
module RoutingM {
  provides interface Attr<uint16_t> as RoutingParent @nucleusAttr("RoutingParent");
}
\end{verbatim}

This line does four things:

\begin{itemize}
\item It provides a standard interface that can be used to access the
attribute. ({\tt Attr})
\item It associates a type with the attribute. ({\tt uint16\_t}) The
  {\tt< >} brackets are the exact syntax you must use, because this
  type is being used as an argument to a nesC 1.2 generic
  interface. In this example, we are creating an interface {\tt Attr}
  of type {\tt uint16\_t}.
\item It sets a local name for the {\tt Attr} interface by using the
  {\tt as} command.
\item It marks the provided interface as an exposed attribute with the
{\tt @nucleusAttr()} tag. Remove this tag, and the Nucleus system will
not expose the attribute. The argument to this tag becomes the
human-readable name of the attribute. ({\tt RoutingParent})
\end{itemize}

\item Then, add code to respond with the attribute when it is
requested.

\begin{verbatim}
implementation {
  uint16_t currentParent;

  [has some routing logic to change that parent]

  command result_t RoutingParent.get(uint16_t* buf) {             
    memcpy(buf, &currentParent, sizeof(uint16_t));
    signal RoutingParent.getDone(buf);
    return SUCCESS;
  }
}
\end{verbatim}

This piece of code does four things:

\begin{itemize}
\item It implements a command that will be called by the query system
when the attribute is requested.
\item It copies the attribute value from a module variable into the
given buffer pointer. Note that you must use {\tt memcpy} to copy the attribute. Using {\tt =}, like {\tt *buf = currentParent;} will not work due to pointer alignment restrictions.
\item It sends a signal to indicate that the attribute has been
copied.
\item It returns {\tt SUCCESS} to the querier to indicate that the
attribute has been successfully retrieved.
\end{itemize}

\end{enumerate}

And that's it. During compilation, the Nucleus scripts will:

\begin{enumerate}
\item find every
exposed attribute in your application by looking for the
{\tt@nucleusAttr()} tags
\item assign each attribute a unique numeric identifier
\item generate a nesC configuration that wires each of them into the
Attribute dispatching component.
\end{enumerate}

If your attribute is not wired to anything, the code will be
eliminated by the nesC compiler. When you compile without {\tt make
<platform> nucleus} the attributes will not be included. Future
releases will include the ability to select which attributes will be
included in the application.

Now, reinstall your application with {\tt make <platform> nucleus
install,<nodeid>}.

\subsection{Using the Java Query Tools}

Querying for Attributes is very similar to querying for RAM
symbols. After building your application, open the {\tt
nucleusSchema.xml} file in a text editor. The {\tt<attributes>}
section contains the list of available attributes, with names and
types as specified in the program. 

As described in section \ref{sec:ram-tools}, you must first run {\tt
SerialForwarder} and set up the {\tt nquery} alias.

\subsubsection{Retrieving Attributes from One Mote}

You can retrieve the value of an Attribute from your attached node
with this command:

\begin{verbatim}
nquery -s link -d serial <Attribute Name>
\end{verbatim}

You should see output like this:

\begin{verbatim}
<nodeid>: <Attribute Name> = <Attribute Value>
\end{verbatim}

You can submit queries for multiple Attributes at the same time:

\begin{verbatim}
nquery -s link -d serial <Attribute Name A> <Attribute Name B> ...
\end{verbatim}

You should see output like this:

\begin{verbatim}
<nodeid>: <Attribute Name A> = <Attribute Value A>
<nodeid>: <Attribute Name B> = <Attribute Value B>
\end{verbatim}
 
To keep Nucleus Query as simple as possible, the size of a Nucleus
Query response is limited to a single message. This results in the
following limitations:

\begin{itemize}
\item If the value of a single Attribute or RAM symbol is too large to
fit in a single message, it will not be returned.
\item If the combined size of the Attribute or RAM symbol values is
too large to fit in a single message, values at the end of the list
will not be returned.
\end{itemize}

\subsubsection{Retrieving Attributes from a Whole Network}

If you want to retrieve Attributes or RAM symbols from the entire
network, you must take the following steps:

\begin{enumerate}
\item Attach a mote running the {\tt TOSBase} application to your
PC. To query a whole network with Nucleus, you have to inject the
queries and retrieve the responses through a {\tt TOSBase}.
\item Build a Drain collection tree by running the following Java
command: 
\begin{verbatim}
java net.tinyos.drain.Drain
\end{verbatim}
Once this command has completed, the tree is built. It will take a few
seconds to run. This tree will then be used to collect values from
remote nodes. It will persist until you build another tree. If you
start to have trouble retrieving data from the network, the
environment may have shifted enough to justify building a new
tree. You may also want to rebuild the tree before each query, if you
query infrequently enough.
\item Run the Nucleus Query command without the {-s} and {-d} flags:
\begin{verbatim}
nquery <Attribute Name> ...
\end{verbatim}
\end{enumerate}

The {\tt nquery} tool waits for a specified amount of time to collect
responses before returning the results. Set this response delay with
the {\tt -t <delay in 10ths of a second>} flag. The default value is
{\tt 10}, or one second. 

This flag sets two different things: how long each node will wait
before responding, and how long the tool will wait before assuming
that each node has responded. Nodes respond at a random time less than
or equal to the delay. The tool waits an additional second te account
for routing delays. If your network has many nodes, response traffic
sent with the default delay may overwhelm the network. You may want to
increase the delay in these situations.

\subsection{More Ways to Expose Attributes}

\subsubsection{Exposing List Attributes}

To expose an attribute that represents a list of data items, such as a
neighbor table, you must use a slightly different interface in your
TinyOS component. This code shows you how to do it:

\begin{verbatim}
module RoutingM {
  <uses some communication components, etc>

  provides interface AttrList<uint16_t> as RoutingNeighbors @nucleusAttr("RoutingNeighbors");
}
\end{verbatim}

Note that you are now providing an {\tt AttrList} interface instead of
an {\tt Attr} interface. Every element in the list must be of the
specified type. The rest is unchanged.

\begin{verbatim}
implementation {
  uint16_t neighbors[ROUTING_MAX_NEIGHBORS];

  <has some logic to fill that table>

  command result_t RoutingNeighbors.get(uint16_t *buf, uint8_t pos) {
    if (pos >= ROUTING_MAX_NEIGHBORS) {
      return FAIL;
    }
    memcpy(buf, &neighbors[pos], sizeof(uint16_t));
    signal RoutingNeighbors.getDone(buf);
    return SUCCESS;
  }
}
\end{verbatim}

Note that the {\tt AttrList.get()} command provides a buffer pointer
as well as an index. This is the position that is being
requested. Check whether the position is valid, then copy the element
at that position into the buffer. The internal representation of your
list is your choice.

To retrieve an element of a list-structured attribute, use the
following {\tt nquery} command:

\begin{verbatim}
nquery <Attribute Name>.<list index integer>
\end{verbatim}

\subsubsection{Exposing Split-Phase Attributes}

If your attribute's value must be obtained through a split-phase
operation, like a sensor reading or the result of a complex
calculation, you can fill the attribute buffer in a split-phase
fashion. Here is some example code:

\begin{verbatim}
module TempSensorM {
  provides interface Attr<uint16_t> as Temperature @nucleusAttr("Temperature");
  uses interface ADC;
}
\end{verbatim}

Exposing an split-phase attribute uses the same interface as a
non-split-phase attribute.

\begin{verbatim}
implementation {

  uint16_t *savedBuf;
  uint16_t savedReading;

  command result_t Temperature.get(uint16_t *buf) {                  
    // make sure that only one request is outstanding
    if (savedBuf != NULL)
      return FAIL;

    savedBuf = buf;
    call ADC.getData();
    return SUCCESS;
  }
}
\end{verbatim}

You will have to save the buffer pointer given to the {\tt get()}
command, so that you can fill it when the data is ready.

\begin{verbatim}
  async event result_t ADC.dataReady(uint16_t data) {
    savedReading = data;
    post TemperatureTask();
    return SUCCESS;
  }

  task void TemperatureTask() {
    memcpy(savedBuf, &savedReading, sizeof(uint16_t));
    signal Temperature.getDone(savedBuf);
    savedBuf = NULL;
  }
}
\end{verbatim}

Once the data is ready, copy it into the buffer and signal the {\tt
  sendDone} event for the attribute.

\subsubsection{Exposing Globally-Unique Attributes}

Normally, the Nucleus scripts automatically assign a number to each
attribute. This number is only unique within the application, meaning
that in a heterogeneous network, a different attribute may be given
the same number as your attribute. Then, your query will retrieve
strange data.

To solve this problem, you can assign your own number to an
attribute. This is very similar to the current TinyOS practice of
assigning {\tt AM} types. Here is some example code:

\begin{verbatim}
[... in some .h file ...]

enum {
  ATTR_ROUTINGPARENT = <some 16-bit integer>
};
\end{verbatim}

Like {\tt AM}, you should make an enumerated type to represent that
number. The prefix doesn't really matter, but it may be useful to mark
this value as an attribute identifier.

\begin{verbatim}
[... in a module .nc file ...]

includes <the .h file that defines the number>

module RoutingM {
  <uses some communication components, etc>

  provides interface Attr<uint16_t> as RoutingParent 
    @nucleusAttr("RoutingParent", ATTR_ROUTINGPARENT);
}
\end{verbatim}

The only difference here is that the identifier has been passed as an
argument to the {\tt @nucleusAttr} tag. This enables the build scripts
to use your number instead of the default number.

By placing the number in a {\tt .h} file, you make it possible for any
component that wants to expose the {\tt ATTR\_ROUTINGPARENT} attribute
to give your number to the {\tt @nucleusAttr()} tag.

\newpage
\section{Event Logging}

This section will be filled in once the Event Logging system is
complete. But, the Query system works now!

\end{document}

