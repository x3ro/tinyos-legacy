\documentclass{article}
\usepackage{fullpage}

\newcommand{\fname}[1]{{\tt #1}}
\newcommand{\code}[1]{{\tt #1}}

\title{Standard TinyOS Sensorboard Interface V1.1}
\author{David Gay, Wei Hong, Phil Levis, and Joe Polastre \\
dgay@intel-research.net, pal@cs.berkeley.edu, jpolastre@cs.berkeley.edu}

\begin{document}

\maketitle

\section{Introduction}

This document defines the ``standard'' interface a sensor board is expected
to offer in TinyOS. As this is still relatively early days, there will
undoubtably be sensor boards which cannot conform to this specification,
but such boards should attempt to follow its spirit as closely as possible.

This standard assumes that sensors return uninterpreted 16-bit values, and,
optionally uninterpreted, arbitrary-size calibration data. Conversion of
sensor values to something with actual physical meaning is beyond the
(current) scope of this document.

This standard departs from current conventions. Sensor board components
and applications will need to be converted.

\section{Directory Organisation}

\begin{enumerate}

\item A sensor board should have a unique name, composed of letters, numbers and
underscores. Case is significant, but two sensor boards must differ in more
than case.\footnote{Necessary to support platforms where filename case 
differences are not significant.} We will use $S$ to denote the sensor board
name in the rest of this document.

\item Each sensor board should have its own directory named $S$; standard TinyOS
sensor boards will be placed in \fname{tinyos-1.x/tos/sensorboards}, but
sensor board directories can be placed anywhere as long as the nesC compiler
receives a {\tt -I} directive pointing to the sensor board's directory.
\footnote{This is supported in v1.1.1 and later of the nesC compiler}.

\item Each sensor board directory must contain a \fname{.sensor} file. This file
is a perl script which contains any additional compiler settings needed for
this sensor board (this file will be empty in many cases). 
%See Section~\ref{sec:dot-sensor} for more information.

\item If the sensor board wishes to define any C types or constants, it should
place these in a file named \fname{$S$.h} in the sensor board's directory.

\item The sensor board directory should contain \emph{sensor board components}
for accessing each sensor on the sensor board. The conventions for these
components are detailed in Section~\ref{sec:sensor-components}.

\item A sensor board may include additional components providing alternative or
higher-level interfaces to the sensors (e.g., for TinyDB). These components
are beyond the scope of this document. Future versions of this standard
are likely to discuss TinyDB attributes.

\item Finally, the sensor board can contain any number of components, interfaces,
C files, etc for internal use. To avoid name collisions, all externally
visible names (interface types, components, C constants and types) used for
internal purposes should be prefixed with $S$.

\end{enumerate}

A simple example: the basic sensor board is named {\tt basicsb}, it's
directory is \[ \fname{tinyos-1.x/tos/sensorboards/basicsb} \] It has no
\fname{basicsb.h} file and its \fname{.sensor} file is empty. It has two
components, {\tt Photo} and {\tt Temp} representing its two sensors.



\section{Sensor Board Components}
\label{sec:sensor-components}

We have not yet selected any naming conventions for sensor board
components. Please select reasonable names\ldots

A sensor board component must provide a \code{StdControl} or
\code{SplitControl} interface for initialisation and power management, some
set of \code{Sensor} or \code{QSensor} interfaces for sampling, and,
optionally, some set of \code{CalibrationData} interfaces for obtaining
calibration data. These interfaces are shown in
Figure~\ref{fig:interfaces}. A component can provide additional interfaces
for other purposes; these are beyond the scope of this document. A sensor
board component should be as lightweight as possible - it should just
provide basic access to the physical sensors and not attempt to do
calibration, signal processing, etc. If such functionality is desired, it
should be provided in separate components.

\begin{figure}
\begin{verbatim}
  interface Sensor {
    /* Sensor is for sensors which will not be sampled at high rates */

    /** Request sensor sample
     *  @return SUCCESS if request accepted, FAIL if it is refused
     *    dataReady or error will be signaled if SUCCESS is returned
     */
    command result_t getData();

    /** Return sensor value
     * @param data Sensor value
     * @return Ignored
     */
    event result_t dataReady(uint16_t data);

    /** Signal that the sensor failed to get data
     * @param info error information, sensor board specific
     * @return Ignored
     */
    event result_t error(uint16_t info);
  }

  interface QSensor {
    /* QSensor is for sensors which need to be sampled at high rates or
       at precise times.
       The only difference with Sensor is that the getData/dataReady
       functions can be called from interrupt handlers. */
    async command result_t getData();
    async event result_t dataReady(uint16_t data);
    event result_t error(uint16_t errorInformation);
  }

  interface CalibrationData {
    /* Collect uninterpreted calibration data from a sensor */

    /** Request calibration data
     *  @return SUCCESS if request accepted, FAIL if it is refused
     *    data error will be signaled if SUCCESS is returned
     */
    command result_t get();

    /** Returns calibration data
     * @param x Pointer to (uinterpreted) calibration data. This data
     *   must not be modified.
     * @param len Length of calibration data
     * @return Ignored.
     */
    event result_t data(const void *x, uint8_t len);
  }
\end{verbatim}
\caption{Sensorboard interfaces}
\label{fig:interfaces}
\end{figure}

If a \code{Sensor} or \code{QSensor} interface named \code{X} has a
corresponding calibration interface, that interface should be called
\code{XCalibration}.

The commands and events in the \code{QSensor} interface are marked
\code{async}, i.e., may execute as part of an interrupt handler. This is to
support applications that require, and sensors that provide, low-jitter
sampling. Figure~\ref{fig:easyfast} shows a component that converts a
\code{QSensor} interface into a regular \code{Sensor} interface.

The \code{Sensor} and \code{QSensor} interfaces return uinterpreted
16-bit data. This might represent an A/D conversion result, a counter,
etc. The optional calibration interface returns uninterpreted,
arbitrary-size data.

Some common setups for sensor board components are:
\begin{itemize}
\item A single \code{Sensor} (or \code{QSensor}) interface. This is
probably the most common case, where a single component corresponds to a
single physical sensor, e.g., for light, temperature, pressure and there is
no expectation of high sample rates.

\item Multiple \code{Sensor} (or \code{QSensor}) interfaces. Some sensors
might be strongly related, e.g., the axes of an accelerometer.  A single
component could then provide a sensor interface for each axis. For
instance, a 2-axis accelerometer which can be sampled at high speed, and which
has some calibration data might be declared with:
\begin{verbatim}
  configuration Accelerometer2D {
    provides {
      interface StdControl
      interface QSensor as AccelX;
      interface CalibrationData as AccelXCalibration;
      interface QSensor as AccelY;
      interface CalibrationData as AccelYCalibration;
    }
  }
\end{verbatim}

\item A parameterised \code{Sensor} (or \code{QSensor}) interface. If a
sensor board has multiple similar sensors, it may make sense to provide a
single component to access all of these, using a parameterised
\code{Sensor} interface. For instance, a general purpose sensor board with
multiple A/D channels might provide an \code{Sensor} interface
parameterised by the A/D channel id.
\end{itemize}

Sensor board components are expected to respect the following conventions
on the use of the \code{StdControl}, \code{SplitControl} and \code{Sensor}
interfaces.  These are given assuming \code{StdControl} is used, but the
behaviour with \code{SplitControl} is identical except that \code{init},
\code{start} and \code{stop} are not considered complete until the
\code{initDone}, \code{startDone} and \code{stopDone} events are
signaled. The conventions are:

\begin{enumerate}
\item \code{StdControl.init}: must be called at mote boot time.

\item \code{StdControl.start}: ensure the sensor corresponding to this
component is ready for use. For instance, this should power-up the sensor
if necessary. The application can call \code{getData} once 
\code{StdControl.start} completes.

If a sensor takes a while to power-up, the sensor board implementer can
either use a \code{SplitControl} interface and signal \code{startDone} when
the sensor is ready for use, or delay \code{dataReady} events until the
sensor is ready. The former choice is preferable if the sensor is going to
be used for high-frequency sampling.

\item \code{StdControl.stop}: put the sensor in a low-power mode. 
\code{StdControl.start} must be called before any further readings 
are taken. The behaviour of calls to \code{StdControl.stop} during
sampling (i.e., when an \code{dataReady} event is going to be
signaled) is undefined.

\item \code{Sensor/QSensor.getData}: get a sample from a sensor. 

\item \code{Sensor/QSensor.dataReady(uint16\_t data)}: signals the
sample value to the application.

\item \code{Sensor/QSensor.error(uint16\_t info)}: reports a sensing 
problem to the application (not all sensors will report errors). The
values for \code{info} are sensor-board specific.

\end{enumerate}

\begin{figure}
\begin{verbatim}
  module QSensorAsSensor {
    provides interface Sensor;
    uses interface QSensor;
  } implementation {
    uint16_t temp;

    command result_t Sensor.getData() {
      return QSensor.getData();
    }

    task void ready() {
      signal Sensor.dataReady(temp);
    }

    async command result_t QSensor.dataReady(uint16_t data) {
      temp = data;
      post ready();
      return SUCCESS;
    }

    command result_t QSensor.error(uint16_t info) {
      return Sensor.error(info);
    }
  }
\end{verbatim}
\caption{Convert a \code{QSensor} to a \code{Sensor} interface}
\label{fig:easyfast}
\end{figure}

\section{\fname{.sensor} File}

This file is a perl script which gets executed as part of the \fname{ncc}
nesC compiler frontend. It can add or modify any compile-time options 
necessary for a particular sensor board. It can modify the following perl
variables:
\begin{itemize}
\item \code{@new\_args}: This is the array of arguments which will be
passed to \code{nescc}. For instance, you might add an include directive
to \code{@new\_args} with \[ \code{push\ @new\_args, '-Isomedir'} \]
\item \code{@commonboards}: This can be set to a list of sensor board names
which should be added to the include path list. These sensor boards must be
in \code{tinyos-1.x/tos/sensorboards}.
\end{itemize}

\section{Example: micasb}

The mica sensor board (micasb) has five sensors (and one actuator, the
sounder):

\vspace{0.2in}
\begin{tabular}{|l|l|l|l|}\hline
Name          & Component & Sensor Interfaces & Other Interfaces\\ \hline \hline
Accelerometer & Accel     & AccelX         & \\
              &           & AccelY         & \\ \hline
Magnetometer  & Mag       & MagX           & MagSetting\\
              &           & MagY           &  \\ \hline
Microphone    & Mic       & MicADC         & Mic \\
              &           &                & MicInterrupt \\ \hline
Light         & Photo     & PhotoADC       & \\ \hline
Temperature   & Temp      & TempADC        &\\ \hline
\end{tabular}
\vspace{0.2in}

Each physical sensor is represented by a separate component. Specific
sensors that have more than one axis of measurement (e.g., Accel and
Mag) provide more than one Sensor interface on a single component. Some
sensors, such as the magnetometer and microphone, have additional
functionality provided through sensor-specific interfaces.

Although light and temperature are represented by separate components,
in reality they share a single ADC pin. The two components Photo and
Temp sit on top of the PhotoTemp component, which controls access to
the shared pin, and orchestrates which sensor is currently connected
to it. From a programmer's perspective, they appear as individual
sensors, even though their underlying implementation is a bit more
complex.

The board's {\tt micasb.h} file contains private configuration data
(pin usage, ADC ports, etc).

The mica sensor board has an empty {\tt .sensor} file. For a more
interesting example, refer to the micawbdot sensor board.

\end{document}

% LocalWords:  TinyOS undoubtably Organisation nesC perl TinyDB basicsb init
% LocalWords:  StdControl SplitControl initialisation parameterised behaviour
% LocalWords:  initDone startDone stopDone getData getContinuousData dataReady
% LocalWords:  uint async
