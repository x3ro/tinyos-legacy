\documentclass[10pt,letterpaper]{article}

\usepackage{mathptmx}
\usepackage{courier}
\usepackage[dvips]{graphicx}
\usepackage{xspace}
\usepackage{color}
\usepackage{subfigure}
\usepackage[small]{caption}

% margin hacking
\setlength{\textwidth}{6.5in}
\setlength{\oddsidemargin}{0in}
%\setlength{\textheight}{9.125in}
\setlength{\textheight}{9in}
\setlength{\topmargin}{0in}
\setlength{\headheight}{0in}
\setlength{\headsep}{0in}

\pagestyle{plain}

%!PN, old LaTeX209 font commands, replaced by the new ones
% keeping just this one
%\def\tenrm{\fontsize{10}{12}\normalfont\rmfamily\selectfont}
%\def\BibTeX{{\rmfamily B\kern-.05em{\scshape i\kern-.025em b}\kern-.08em
%\TeX}}

\definecolor{gray}{rgb}{0.5,0.5,0.5}
\newcommand{\comment}[1]{{\color{gray}[\textsf{#1}]}}
\def\qed{\rule[.25mm]{2.5mm}{2.5mm}}
\newcommand{\ie}{\emph{i.e.}\xspace}
\newcommand{\eg}{\emph{e.g.}\xspace}
\newcommand{\etc}{\emph{etc.}\xspace}

%\renewcommand{\baselinestretch}{0.995}

\begin{document}

\title{Tiny Application Sensor Kit (TASK)}

\author{Wei Hong \\
Intel Research, Berkeley \\
whong@intel-research.net}

\date{}

\maketitle

Sensor network application development and deployment present daunting 
challenges to even sophisticated software developers.  Sensor network 
applications combine the complexities of both distributed and embedded 
systems design, and these are often amplified by unreliable network 
connections and extremely limited physical resources.  Moreover, many 
sensor network applications are expected to run unattended for months at a time.

Real users of sensor networks ranging from plant biologists monitoring
micro-climates in a giant redwood tree to facility managers monitoring
vibration signatures of their equipments are most likely not sophisticated
software developers.  We must reduce the complexity of sensor network
application development and deployment to ensure the success of sensor
network technology in the real world.

We believe that many of the complexities in sensor network application
development and deployment are caused by the current low-level
programming interfaces and the lack of tools.  At Intel Research
in Berkeley, we have been building a suite of tools called the Tiny Application
Sensor Kit (TASK) aiming to break down the barrier to entry for
non-sophisticated users to develop and deploy their own sensor network
applications.

TASK consists of the following components:
\begin{itemize}
\item {\bf TinyDB} 
based sensor network that allows traditional programs
  to interact the sensor network through a declarative SQL-like interface.
  See {\tt tinydb.pdf} for details.
\item {\bf TASK Server}, a server process running on a sensor network gateway
  that acts as a proxy for the sensor network on the internet.
\item {\bf TASK DBMS}, a relational database that stores sensor readings, sensor
  network health statistics, sensor locations and calibration coefficients, etc.
  Curently TASK only works with PostgreSQL (see http://www.postgresql.org) and
  has been tested on both 7.2 and 7.3 releases.
\item {\bf TASK client tools} including {\bf TASK Deployment Tool} that helps users record
  sensor node metadata, {\bf TASK Configuration Tool} that helps users choose
  data collection intervals and data filtering and aggregation criteria,
  and {\bf TASK Visualization Tool} that helps users monitor the network health
  and sensor readings.  See {\tt TASKVisualizer.pdf} for details.
\item {\bf TASK Field Tool} running on a PDA that help users diagnose
  and resolve problems in certain areas of the network in the field.
  See {\tt TASKFieldTool.pdf} for details.
\end{itemize}
TASK also integrates easily with most popular data analysis tools, e.g.,
MS Excel, Matlab, ArcGIS, etc through standard ODBC or JDBC interfaces.


The following is a quick-start guide for using TASK.
\begin{enumerate}
\item After the installation of TASK, the PostgreSQL database needs to be
initialized before TASK can be used.  First, {\tt cd tinyos-1.x/tools/java/net/tinyos/task/tasksvr}.  On Cygwin, simply  and run {\tt setup-task-db.sh}.  On Linux, do the following
\begin{enumerate}
\item as root: {\tt mkdir /pgdata; chown postgres.postgres /pgdata/}
\item change user to postgres
\item run {\tt initdb}
\item edit {\tt /pgdata/pg\_hba.conf} to uncomment ``local all all trust''
and ``host all all 127.0.0.1 255.255.255.255 trust'' and comment out ``local all all ident sameuser'' at the end of the file
\item run {\tt setup-task-db.sh}
\end{enumerate}
\item program TASK motes from {\tt tinyos-1.x/apps/TASKApp}.  You must program a
node $0$ for the basestation.
\item start the TASK server by {\tt cd tinyos-1.x/tools/java; java net.tinyos.task.tasksvr.TASKServer \&}
\item start TASK GUI by {\tt cd tinyos-1.x/tools/java; java net.tinyos.task.taskviz.TASKVisualizer localhost}
\item start TASK Field Tool by {\tt cd tinyos-1.x/tools/java/ne/tinyos/task/field; python config-gui.py; python tool.py}
\end{enumerate}

TASK bugs can be submitted at \\
{\tt https://sourceforge.net/tracker/?atid=551233\&group\_id=28656\&func=browse}

\end{document}
