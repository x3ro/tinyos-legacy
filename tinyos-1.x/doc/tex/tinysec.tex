\documentclass[11pt]{article}

\usepackage{geometry, graphicx, color, url}
\title{TinySec: User Manual}
\author{Chris Karlof \hspace{1cm} Naveen Sastry \hspace{1cm} David Wagner\\ \{ckarlof, nks,
  daw\}@cs.berkeley.edu}

\begin{document}
\maketitle
\section{Introduction}

We introduce TinySec, a link layer encryption mechanism which is meant to be
the first part in a suite of security solutions for tiny devices.  The core of
TinySec is an efficient block cipher and keying mechanism that is tightly
coupled with the Berkeley TinyOS radio stack.  TinySec currently utilizes a
single, symmetric key that is shared among a collection of sensor network
nodes.  Before transmitting a packet, each node first encrypts the data and
applies a Message Authentication Code (MAC), a cryptographically strong
unforgeable hash to protect data integrity.  The receiver verifies that the
packet was not modified in transit using the MAC and then deciphers the
message.

There are four main aims of TinySec:
\begin{itemize}
\item Access control. Only authorized nodes should be able to participate in
  the network.  Authorized nodes are designated as those nodes that possess
  the shared group key.
\item Integrity.  A message should only be accepted if it was not altered in
  transit. This prevents, for example, man-in-the-middle attacks where an
  adversary overhears, alters, and re-broadcasts messages.
\item Confidentiality. Unauthorized parties should not be able to infer the
  content of messages.
\item Ease of use. Finally, taking into account the diversity of sensor
  networks users, TinySec should not be difficult to use. We hope to provide a
  communication stack that provides the above three goals which is no more
  difficult to use than the traditional, non-security aware communication
  stack.
\end{itemize}

TinySec works both in the TOSSIM simulator as well as on the Mica and Mica2
motes.

\section{Installation}
TinySec is now included with the standard TinyOS distribution. 
Building TinySec enabled applications uses the {\tt mote-key} script. 
Make sure this script is in your {\tt PATH}. 

\subsection{Testing Your Installation}
The TestTinySec application can be used to check your TinySec installation. It
periodically sends a data packet using TinySec.  SecureTOSBase is the
TinySec-aware analogue to TOSBase and will toggle red when it receives
a valid packet. 

You should program two motes, one with TestTinySec and one
SecureTOSBase
\begin{verbatim}
     $
     $ cd nest/tinyos-1.x/apps/TestTinySec
     $ make mica2 install
     $ # Now install the second mote
     $ cd nest/tinyos-1.x/apps/SecureTOSBase
     $ make mica2 install
\end{verbatim}

TinySec is properly working if the SecureTOSBase mote toggles its red
LED periodically. This indicates that the packet was transmitted with
authentication enabled. 

If either application fails to build, then check that the ``mote-key'' script
is in your {\tt PATH}.  


\subsection{Writing Applications}
\label{sec:writing}
Enabling TinySec for your applications should be a straightforward process.  In
general, application code does not need to change\footnote{Except for
  applications using the group field. See Section \ref{sec:groups} about
  deprecation of the group field in the packet header.}.  If your Makefile
includes Makerules from the {\tt tinyos-1.x/apps} directory, you only need to
add {\tt TINYSEC=true} to your application's Makefile in order to enable
TinySec. This can also be be done from the command line when invoking make:
e.g., {\tt
  make mica2 TINYSEC=true}.

By default, TinySec will authenticate all messages, but encryption is turned
off. TinySec allows you to dynamically alter the combination of security
mechanisms for
your application with the {\tt TinySecMode} interface. The {\tt TinySecC}
component exports this interface. There are two commands in the {\tt
  TinySecMode} interface for setting the
transmit and receive mode:

\begin{verbatim}
     command result_t setTransmitMode(uint8_t mode); 
     command result_t setReceiveMode(uint8_t mode);  
\end{verbatim} 
{\tt setTransmitMode} takes one of three values as its argument: 
\begin{verbatim}
     TINYSEC_AUTH_ONLY 
     TINYSEC_ENCRYPT_AND_AUTH
     TINYSEC_DISABLED
\end{verbatim} 
Similarly, {\tt setReceiveMode} takes one of three values as its argument: 
\begin{verbatim}
     TINYSEC_RECEIVE_AUTHENTICATED
     TINYSEC_RECEIVE_CRC
     TINYSEC_RECEIVE_ANY
\end{verbatim}
The default is {\tt TINYSEC\_AUTH\_ONLY} for sending and {\tt
  TINYSEC\_RECEIVE\_AUTHENTICATED} for receiving. {\tt
  TINYSEC\_ENCRYPT\_AND\_AUTH} sends encrypted and authenticated messages, and
  {\tt TINYSEC\_DISABLED} sends messages with the standard TinyOS radio stack
  (CRC only, with no encryption or authentication). 

When the receiver is in 
 {\tt TINYSEC\_RECEIVE\_AUTHENTICATED} mode, it will only accept messages from a
 sender in {\tt TINYSEC\_AUTH\_ONLY} mode or {\tt TINYSEC\_ENCRYPT\_AND\_AUTH}
 mode. A receiver in 
 {\tt TINYSEC\_RECEIVE\_CRC} mode will only accept messages from a
 sender in {\tt TINYSEC\_DISABLED} mode, and a receiver in  \\ {\tt
 TINYSEC\_RECEIVE\_ANY} mode will accept messages from senders in any mode. 
The application layer can check which mode a message was received in by
inspecting the {\tt receiveSecurityMode} in the corresponding {\tt
 TOS\_Msg}. This interface is subject to race conditions.

\subsection{SecureTOSBase}

Use
the {\tt SecureTOSBase} application to interface your PC with a network of
motes enabled with TinySec. 

{\tt SecureTOSBase} will only accept messages that that have been sent
with a MAC; thus it will \emph{not} receive messages sent with the old radio
stack or with TinySec disabled.  You will need to modify the
{\tt SecureTOSBase} application if you would like to receive both
authenticated and unauthenticated messages.

\subsection{Key Management}        
\label{sec:key}

When using TinySec, you must be aware of keys and the key-file.  Each Mica mote
can only communicate with other motes that have been programmed with the same
key. The key is currently set in a given program at build time. Without any
extra arguments to the normal build process, the default key-file and default
key will be used. A default key-file will be created for you the first time
TinySec is used and stored in the file at {\tt  \~\//.tinyos\_keyfile}. 
It looks like:

\begin{verbatim}
     # TinySec Keyfile. By default, the first key will be used.
     # You can import other keys by appending them to the file.

     default 6D524D67F24F178B0A69933FDD6C6F7B
\end{verbatim}

Note that the your key value will not be the same as listed above. Each line
lists a key name and the key value. When you invoke {\tt make mica2}, the first
key in the default key-file will be installed. 

This means that by default, if you install a program onto one mote from your
laptop, and install a program onto another mote from your desktop, they will
not be able to interoperate. This is because they will be using different
keys. Thus, you'll need to perform one of the following:
\begin{itemize}
   \item Use the same key-file on both computers
   \item Copy the key-file from your laptop to the desktop, renaming the file
     to ``laptop-keyfile''. Then, when building on the desktop, use the new
     keyfile whenever you wish to create motes that interoperate with motes
     programmed from the laptop:
     \begin{verbatim} make mica2 KEYFILE=laptop-keyfile \end{verbatim}
  \item Copy the line from the keyfile which reads ``default 6D524\ldots'' from the laptop to the desktop
     keyfile (.keyfile). In addition, rename the key label from ``default'' to
    ``laptop''.  Then, when building on the desktop, use the new keyfile
     whenever you wish to create motes that interoperate with motes programmed
     from the laptop:
     \begin{verbatim} make mica2 KEYNAME=laptop \end{verbatim}
\end{itemize}

\subsection{Updating Keys}
The {\tt TinySecControl} interface exported by the {\tt TinySecC} component
enables you update TinySec keys and query and reset the initialization vector
(IV). The {\tt TinySecControl} interface has six commands:

\begin{verbatim}
    command result_t updateMACKey(uint8_t * MACKey);
    command result_t getMACKey(uint8_t * result);
    command result_t updateEncryptionKey(uint8_t * encryptionKey);
    command result_t getEncryptionKey(uint8_t * result);
    command result_t resetIV();
    command result_t getIV(uint8_t * result)
\end{verbatim}

{\tt updateMACKey()} and {\tt updateEncryptionKey()} update the corresponding
key, and take a pointer to an array of {\tt TINYSEC\_KEYSIZE} bytes. These
commands return {\tt SUCCESS} if the key update is successful. These commands
will return {\tt FAIL} if the key update is attempted while any crypto operation
is executing in TinySec. This is the simplest solution to ensure atomicity of
crypto operations with respect to the TinySec keys. The caller of these commands
must check their return values, and try again if they return {\tt FAIL}.

{\tt
  getMACKey()} and {\tt getEncryptionKey()} return the corresponding key into
the array referenced by the input parameter {\tt result}. {\tt result} must
reference a buffer of size at least {\tt TINYSEC\_KEYSIZE} bytes. 

{\tt resetIV()} resets the counter portion of the IV to 0. This will most likely
be used in conjunction with updating the encryption key. As in the key
update commands, {\tt resetIV()} will return {\tt FAIL} if it is called while
any crypto operation is executing in TinySec. {\tt getIV()} returns
the current value of the IV into the array referenced by the input parameter {\tt
  result}. {\tt result} must
reference a buffer of size at least {\tt TINYSEC\_IV\_LENGTH} bytes.
 
\subsection{Groups}
\label{sec:groups}
The concept of ``groups'' does not exist under TinySec. Groups are a
means to provide namespaces for destination addresses when sending
packets.  Each mote belongs to one of 255 different groups. The group byte is
transmitted as a part of every packet. A packet is accepted only if the sender
and receiver are in the same group.

This functionality is subsumed by using keys under TinySec. Using TinySec, a
message will only be accepted if the sender and receiver use the same
key. This is enforced cryptographically. However, if a received message passes
the MAC, the {\tt TOS\_Msg} received at the application layer will show the group
that the receiving mote was programmed with.  

\section{Under the Covers}
We've created replacements for the TinyOS radio stacks that incorporate
TinySec. These files are located in {\tt
  tinyos-1.x/tos/platform/\{mica2,mica,pc\}/TinySec}. The remaining TinySec
components are located in {\tt tinyos-1.x/tos/lib/TinySec}.
See \url{http://www.cs.berkeley.edu/~nks/tinysec} for more information.
\end{document}
