\documentclass{article}

\usepackage{fullpage}

\parindent 0em
\parskip 0.2cm

\newcommand{\kw}[1]{\mbox{\tt #1}}
\newcommand{\file}[1]{\mbox{\tt #1}}

\title{Design of Matchbox, the simple filing system for motes}
\author{David Gay \\
Version 1.0}

\begin{document}
  
\maketitle

\section{Overview}

The mote filing system is designed to provide a simple filing system for
mote-based applications. Its design goals and functionality are covered
in the matchbox document; this document describes Matchbox's implementation
for the mica family of motes.

This implementation is tailored to the Atmel AT45DB041B flash chip used
on the Micas. It should consider the following constraints of this 
particular flash chip:
\begin{itemize}
\item The flash is divided into sectors (mostly 128K, though some are smaller).
\item Each sector is divided into pages, each page is 264 bytes long. 
\item Pages can only be written as a whole.
\item Pages should be erased before being written (whatever ``should'' means).
\item After 10'000 writes to a sector, all its pages should have been
written at least once.
\end{itemize}

Note that these are different from ``usual'' flash rules. These typically say:
\begin{itemize}
\item Each page/byte/whatever can be written a limited number of times (e.g.,
100'000)
\item Erases are on a larger granularity than writes (e.g., 64k-erases, 
528-byte writes)
\end{itemize}

\section{Data structures}

\subsection{On flash} 

The 264-byte page is divided into 256 data bytes and 8 metadata bytes.
The metadata stores the following information:
\begin{itemize}
\item per-page CRC (data bytes only) (16 bits)
\item page-write counter (16 bits)
\item last-byte-used in page (9 bits)
\item next page pointer (a page number), or \kw{IFS\_EOF\_BLOCK} for
the last page
\item root page marker (3 bits)
\item metadata check byte (8 bits)
\end{itemize}

Pages are numbered from 0 to <file system capacity in pages>-1. The
expectation is that the file system will use some whole number of flash
sectors. Pages \kw{IFS\_FIRST\_PAGE} through \kw{IFS\_FIRST\_PAGE +
IFS\_NUM\_PAGES} are requested from the \kw{ByteEEPROM} component for
use by the filing system.

There are two types of pages: data pages (root page marker is 0), and
first-meta-data-page pages (root page marker is \kw{IFS\_ROOT\_MARKER}).

The metadata check byte detects inconsistent metadata. It is the bitwise
complement of the sum of the three bytes that store the last-byte-used in
page, the next page pointer and the root page marker.

Files and the metadata are both byte-streams. A byte-stream is a sequence
of bytes, and is identified by its initial page number. The pages of a
byte-stream are found by following the linked list formed by the next page
pointers. Data ends at the first page where last-byte-used is not 256. Note
that the linked list of pages may extend arbitrarily past the last page
containing data - this means that space has been reserved for future
expansion of the byte-stream.

The per-page CRC is always used for metadata, it is used for files if the
\kw{FS\_CRC\_FILES} constant (in \file{Matchbox.h}) is true.

\paragraph{Meta-data format (viewed as a byte-stream)}:
A header stores the meta-data version number, followed by an unsorted list
of file entries. Each file entry is a 14-byte null-terminated string
followed by the file's byte-stream's initial page number (2 bytes). The
first page of the meta-data bytestream has the first-meta-data-page
page-type. The other pages of the meta-data, and the file byte-stream pages
have the data page type.

Note that their is no fixed \emph{root} page: the initial page number of the
meta-data is found by scanning the flash looking for the page with type
first-meta-data-page which contains the highest meta-data version number.
This avoids continuously writing the \emph{root} page (a bad idea given the 
limited number of writes supported by flash). Scanning the flash to locate
this page at boot time is acceptable in Matchbox as the flash is small
(a few 100k) and page access relatively fast (approx. 1ms). 

There is also no explicit listing of the free pages, this can also be
reconstructed at boot time once the initial page of the meta-data is
found.

\subsection{In memory} 

We must keep the following information (each piece of information is
followed by the name of the component which owns it):
\begin{itemize}
\item Meta-data root page.
\item Meta-data version number.
\item Free page bitmap.
\item Last allocated page.
\item Number of free pages.
\item For each read file descriptor:
  \begin{itemize}
  \item initial page number
  \item current page
  \item next page
  \item current page offset
  \item last offset on this page
  \item use-crc-for-this-file flag
  \end{itemize}
\item For each write file descriptor:
  \begin{itemize}
  \item initial page number
  \item current page
  \item next page
  \item the current page's number (first page is 0, second is 1, etc)
  \item current page offset
  \item last offset on this page
  \item remaining reserved bytes
  \item use-crc-for-this-file flag
  \end{itemize}
\item various pieces of miscellaneous state for operations in progress
\end{itemize}

Matchbox uses one read and one write file descriptor internally (for the
metadata).

\section{Algorithms}

This section mentions a few basic principles, rather than cover the
(straightforward) algorithms for each operation:

Update of meta-data is always atomic: new meta-data is written by making a
copy of the old meta-data in new pages, with appropriate changes. The new
initial metadata page contains a higher version number than the current
metadata. The initial page is only marked as a first-meta-data-page once
all other metadata has been written. New meta-data is only written after the
information it refers to is known to be valid (e.g., on new file creation,
the file's first page has been successfully written).

At boot time, the initial metadata page is located by reading all of the
flash sequentially and finding the valid (by reling both on the metadata
check byte and the per-page CRC) first-meta-data-page page with the highest
version number.

Once the initial metadata page is located, the free page bitmap is built by
walking through the filing system. Inconsistencies in free pages will lead
to the filing system being marked readonly. A separate fsck-style tool
should be written to fix this (note that this should only happen as a
result of page corruption given the atomic metadata updates, hence should
be hopefully rare).

A corrupt file is one containg a page with a bad CRC. This could be due,
e.g., to a power-off during a file-page write. Currently, attempts to
read such a file will fail when the corrupt page is reached. Attempts
to open a corrupt file for writing will fail.

A free page is found by scanning the free page bitmap starting at the value
of \emph{Last allocated page} (and updating \emph{Last allocated
page}). This produces some amount of ``wear levelling'' (i.e., writing each
page an equal number of times). The boot-time value of \emph{Last allocated
page} is the meta-data initial page number.

A background task sequentially walks over the filing system's pages
and checks their write counters. Any page which is more than 9000
writes ``old'' is rewritten (and hence sees its write counter updated).
The write counter value is maintained inside the EEPROM component.
A walk rate of one page per minute should be sufficient (assuming
that write-rate is not high - may need further analysis and/or
adaptivity).

\section{Interaction with TinyOS and Current Status}

Matchbox will interoperate with direct access to the flash chip via the
\kw{ByteEEPROM} component. Matchbox reserves its area of the flash chip
with \kw{ByteEEPROM} at boot time. It is possible to change this area by
editing the \kw{IFS\_FIRST\_PAGE} and \kw{IFS\_NUM\_PAGES} constants in
\file{tos/lib/FS/IFS.h}.

The write counters and the background task which rewrites pages have not
yet been implemented.

\end{document}
