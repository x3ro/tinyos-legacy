\documentclass[10pt]{article}

\usepackage{fullpage}
\usepackage{verbatim}

\parindent 0em
\parskip 0.2cm

\newcommand{\kw}[1]{\mbox{\tt #1}}
\newcommand{\code}[1]{\mbox{\tt #1}}
\newcommand{\file}[1]{\mbox{\tt #1}}

\title{Matchbox: A simple filing system for motes}
\author{David Gay \\
Version 1.0}

\begin{document}
  
\maketitle

\section{Overview}

\emph{Matchbox}, the mote filing system is designed to provide a simple
filing system for mote-based applications. The design has the following
goals:
\begin{itemize}
\item Reliability, specifically:
  \begin{itemize}
  \item data corruption should be detected
  \item catastrophic failure (e.g., power off) should never lead to meta-data
    corruption, data loss should be limited to files being written at the
    time of failure. This implies that file renaming and deletion are 
    atomic operations.
  \end{itemize}
\item Low resource consumption (especially RAM)
\item Consider typical restrictions on flash memories (there are limitations
on the number of times a block can be erased, on the minimum erasure size,
and on multiple writes within a block - these vary depending on flash
technology)
\end{itemize}

Most of the design is based on what is not necessary. We do not need:
\begin{itemize}
\item Security in any form
\item A hierarchical filing system
\item Random file access
\item Multiple readers or writers of the same file
\item Many files open simultaneously, however we do wish to support (at
  least) having two different files open, one for reading and one for
  writing (this supports file copy/edit operations)
\end{itemize}

Expected clients of this filing system are TinyDB (and the Generic Sensor
Kit), for logging data sets to permanent storage and for storing metadata,
and the virtual machine for storing programs.

Matchbox is currently implemented for mica family motes (these motes use
the Atmel AT45DB041B flash chip).

\section{Matchbox}

Matchbox is a flat filing system that supports only sequential reads and
appending writes. Files are unstructured (simply a sequence of
bytes). Space can optionally be preallocated for a file (ensuring both that
space exists in permanent storage, and potentially reducing write
overhead).

The same file cannot be opened for both reading and writing. Files (that
are not open) may be renamed or deleted. In Unix style, renaming a file A
as an existing file B also deletes B (the rename will fail if B is open).

\section{Matchbox API}

The Matchbox component (\kw{Matchbox}) has the following specification:

\begin{verbatim}
  provides {
    interface StdControl;
    interface FileDir;
    interface FileRename;
    interface FileDelete;
    interface FileRead[uint8_t fd];
    interface FileWrite[uint8_t fd];
  }
  uses {
    event result_t ready();
  }
\end{verbatim}

This interface is covered in three parts: initialisation, directory
operations and file operations. The detailed interface specifications are
in Appendix~\ref{sec:interfaces}.

\subsection{General Operation}

Nearly all file system operations are split-phase, and fail (with result
\kw{FAIL}) at initiation if another operation is already in progress. Also,
the filing system will be extremely unhappy (i.e., fail) if another
component uses the external flash during a file system operation.

The \kw{fileresult\_t} type is used in most split-phase completion events
to report the success or failure of the requested operations. The
following results are defined:
\begin{itemize}
\item \kw{FS\_OK}: operation succeeded.
\item \kw{FS\_NO\_MORE\_FILES}: end of directory reached.
\item \kw{FS\_ERROR\_NOSPACE}: filing system is full (there may be a
little space left, but not enough to guarantee completion of the requested
operation while preserving the filing system from corruption in case of
failure).
\item \kw{FS\_ERROR\_FILE\_OPEN}: the requested file is already open for
reading or writing.
\item \kw{FS\_ERROR\_NOT\_FOUND}: the requested file is not found.
\item \kw{FS\_ERROR\_BAD\_DATA}: corrupt data was found (should only occur
for file operations).
\item \kw{FS\_ERROR\_HW}: a problem occurred when accessing the flash chip.
\end{itemize}

The Matchbox files are in the \file{tos/lib/FS} directory, and depend
on files in \file{tos/lib/Util}. Matchbox programs must therefore be
compiled with \kw{-I\%T/lib/FS -I\%T/lib/Util} options.

\subsection{Initialisation}

Matchbox is initialised via the usual \kw{StdControl} interface, however it
is not immediately ready for use. File system operations can only be 
initiated after the \kw{ready} event is signaled.

\subsection{Directory Operations}

Directory contents are listed by calling the \kw{start} command in
\kw{FileDir} (not split-phase), then calling \kw{readNext} to get each
subsequent file. The files are returned in the \kw{nextFile} event. The
file system is considered busy from the call to \kw{start} until
\kw{nextFile} signals an error (normally the \kw{FS\_NO\_MORE\_FILES}
error).

The number of free bytes in the filing system is returned by the
\kw{freeBytes} command.

Files can be renamed and deleted via the split-phase \kw{FileDelete} and
\kw{FileRename} interfaces. Open files can not be renamed or deleted. Renaming
a file $X$ over an existing file $Y$ will delete $Y$ (but will fail if $Y$
is currently open).

\subsection{File Operations}

The \kw{FileRead} and \kw{FileWrite} interfaces are parameterised. Each
interface instance corresponds to a separate \emph{file descriptor} and
allows a separate file to be opened. File descriptors for reading and
writing are separate; file descriptors for reading are obtained with
\code{unique("FileRead")}, file descriptors for writing are obtained with
\code{unique("FileWrite")}. For instance, an application that
simultaneously needs two files open for reading and one for writing might
have the following configuration:

\begin{verbatim}
   configuration MyApp { }
   implementation {
     components Matchbox, MyCode, ...;

     MyCode.FileRead1 -> Matchbox.FileRead[unique("FileRead")];
     MyCode.FileRead2 -> Matchbox.FileRead[unique("FileRead")];
     MyCode.FileWrite -> Matchbox.FileWrite[unique("FileWrite")];
     ...
   }
\end{verbatim}

The \kw{FileRead} interface defines the operations for opening (\kw{open}
command, \kw{opened} event), closing (\kw{close} command, which is not
split-phase), reading (\kw{read} command, \kw{readDone} event) and finding
the remaining bytes in a file (\kw{getRemaining} command, \kw{remaining}
event).

The \kw{FileWrite} interface defines the operations for opening (\kw{open}
command and \kw{opened} event), closing (\kw{close} command and \kw{closed}
event), appending (\kw{append} command and \kw{appended} event),
synchronising (\kw{sync} command and \kw{synced} event) and reserving space
for a file (\kw{reserve} command and \kw{reserved} event). The \kw{opened}
event reports the current size of the file. Appends are only guaranteed to
be flushed after a \kw{closed} or a \kw{synced} event. Space can be
pre-reserved for a file using the \kw{reserve} command, this guarantees
that subsequent appends up to the requested size will not fail due to lack
of space, and leads to somewhat faster appends.

\section{Remote Matchbox Access}

The \kw{Remote} component provides for remote Matchbox access, assuming a
reliable communications channel. It is thus easiest to use it via the UART
(e.g., when debugging), though it can be used via the radio if some other
component is used to provide reliable, bi-directional packet transmission.

The \kw{Remote} component gives access to all file system operations. The
java \code{net.tinyos.matchbox} package provides some low-level
infrastructure to talk to the \kw{Remote} component, while the
\code{net.tinyos.matchbox.tools} package provides some simple java programs
to list the directory (\code{Dir.java}), delete or rename a file
(\code{Delete.java} and \code{Rename.java}), and copy files in and out of
Matchbox (\code{CopyIn.java} and \code{CopyOut.java}).

\section{Acknowledgments}

Timothy Roscoe came up with the Matchbox name and Wei Hong helped define
the filing system requirements.

\newpage
\appendix

\section{Interfaces}
\label{sec:interfaces}

\small
\verbatiminput{../../tos/lib/FS/FileDir.nc}
\newpage
\verbatiminput{../../tos/lib/FS/FileDelete.nc}
\newpage
\verbatiminput{../../tos/lib/FS/FileRename.nc}
\newpage
\verbatiminput{../../tos/lib/FS/FileRead.nc}
\newpage
\verbatiminput{../../tos/lib/FS/FileWrite.nc}

\end{document}

% LocalWords:  TinyDB metadata Atmel API initialisation fileresult FS NOSPACE
% LocalWords:  HW tos initialised StdControl FileDir readNext nextFile FileRead
% LocalWords:  freeBytes FileDelete FileRename FileWrite parameterised readDone
% LocalWords:  getRemaining synchronising pre CopyIn CopyOut
