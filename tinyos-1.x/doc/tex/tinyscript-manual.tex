\documentclass[12pt]{article}

\usepackage{graphicx}
\usepackage{graphics}
\usepackage{multicol}
\usepackage{epsfig,amsmath,amsfonts}
\usepackage{xspace}
\usepackage{subfigure}

\makeatletter                                   % Make '@' accessible. 
\oddsidemargin=0in                              % Left margin minus 1 inch.
\evensidemargin=0in                             % Same for even-numbered pages.
\marginparsep=0pt                               % Space between margin & text

\renewcommand{\baselinestretch}{1.2}              % Double spacing

\textwidth=6.5in                                % Text width (8.5in - margins).
\textheight=9in                                 % Body height (incl. footnotes)

\topmargin=0in                                  % Top margin minus 1 inch.
\headsep=0.0in                                  % Distance from header to body.
\skip\footins=4ex                               % Space above first footnote.
\hbadness=10000                                 % No "underfull hbox" messages.
\makeatother                                    % Make '@' special again.

\newcommand{\mate}{Mat\'{e}\xspace}
\newcommand{\bomb}{Bombilla\xspace}
\newcommand{\kw}[1]{{\tt #1}}
\newcommand{\grammarshift}{\vspace*{-.7cm}}
\newcommand{\grammarindent}{\hspace*{2cm}\= \\ \kill}
\newcommand{\opt}{$_{OPT}$}

\title{The TinyScript Language}
\author{Philip Levis\\pal@cs.berkeley.edu}
\date{}

\begin{document}

%\fontfamily{cmss}                               % Make text sans serif.
%\fontseries{m}                                  % Medium spacing.
%\fontshape{n}                                   % Normal: not bold, etc.
%\fontsize{10}{10}                               % 10pt font, 10pt line spacing 

\maketitle
\vspace{2in}
\begin{center}
Version 1.0\\
July 12, 2004
\end{center}

\fontfamily{cmr}                                % Make text Roman (serif).
\fontseries{m}                                  % Medium spacing.
\fontshape{n}                                   % Normal: not bold, etc.
\fontsize{10}{10}                               % 10pt font, 10pt line spacing


\thispagestyle{empty}
\newpage
\tableofcontents
\newpage

\section{Introduction}

This document describes TinyScript, a language that the \mate VM
framework supports for programming sensor networks. TinyScript is an
imperative, BASIC-like language with dynamic typing and basic control
structures such as conditionals and loops. 

\section{Script Structure}
\label{sec:tinyscript}

This is an example TinyScript program that increments a counter:

\begin{quotation}
\begin{verbatim}

! Define a shared variable, counter
shared counter;

! Increment it
counter = counter + 1;

\end{verbatim}
\end{quotation}

In TinyScript programs, all variables must be declared before any
program statements. For example, the following program is invalid (and
will throw a compilation error):

\begin{quotation}
\begin{verbatim}

! Define a shared variable, counter
shared counter;

! Increment it
counter = counter + 1;

! Define a shared variable, index: ERROR
shared index;

\end{verbatim}
\end{quotation}

Statements generally end with a semicolon. {\tt !} declares the start
of a comment, which extends to the end of a line. For example,

\begin{quotation}
\begin{verbatim}

shared counter; ! Define a shared variable, counter
counter = counter + 1; ! Increment it

\end{verbatim}
\end{quotation}

is valid TinyScript code.

TinyScript function and variable identifiers are case insensitive:
{\tt var} is the same as {\tt VAR} or {\tt Var}. Identifiers are
composed of alphanumeric characters and the underscore ('\_'), but the
first character of an identifier must be a letter or the
underscore. The following are all valid identifiers:

\begin{verbatim}
temp
a51_b
TEMP
temp7
buffer_Index
_3
\end{verbatim}

The following are invalid identifiers:


\begin{verbatim}
@a
4b
sd?
a 5
\end{verbatim}


To be precise, identifiers must follow the pattern {\tt
[A-Za-z\_][A-Za-z0-9\_]*}.

Certain words are TinyScript keywords, and cannot be used in programs
to name variables or functions. In the above scripts, {\tt shared} is
a keyword, declaring {\tt counter} to be a shared variable. The full
list of keywords is:

\begin{verbatim}
not     and     or      xor     eqv     imp
for     to      next    step    until   while
end     private shared  buffer  if       then
else
\end{verbatim}

TinyScript function and variable identifiers are case insensitive:
{\tt var} is the same as {\tt VAR} or {\tt Var}. Keywords exist as
both their uppercase and lowercase versions: {\tt shared} can also be
written {\tt SHARED}, but cannot be written {\tt sHarEd}.

\section{Variables}
\label{sec:variables}

TinyScript programs have two basic variable types: scalars and
buffers. Scalars represent a single data item, such as an integer or a
sensor reading. Buffers are small collections of values. In the above
programs, the keyword {\tt shared} declared the variables to be
scalars. Handlers can declare two kinds of scalars: {\tt private} and
{\tt shared}. Private variables are local to that handler; the
statement {\tt private a;} in two different handlers refers to two
different variables. In contrast, {\tt shared a;} in two different
handlers refer to the same variable. Using a shared variable, handers
can pass data to one another. Buffers, declared with {\tt buffer}, are
implicitly shared variables.

Both values and buffers are dynamically typed. That is, the variables
themselves have no explicit type in a program; instead, their type
is determined dynamically as a program runs. In this program,

\begin{quotation}
\begin{verbatim}

shared counter; 
shared sensor;
counter = random();
sensor = light();

\end{verbatim}
\end{quotation}

the variable {\tt counter} takes the type integer ({\tt rand()}
returns an integer) while {\tt sensor} takes the type light ({\tt
light()} returns a light value).

Types constrain how values can be modified, and how buffers can be
accessed. There is one basic scalar type, integer (16-bit, signed).
Additionally, every sensor has its own type. Integers can be modified
freely, through arithmetic, assignment, and other
transformation. Sensor readings, however, are immutable. You cannot
add two sensor readings, even if from the same sensor.

The idea is that sensor readings should only be what is actually read
from a sensor. Transformations on these readings should be
distinguishable from actual readings.  To modify sensor readings, they
must be cast to an integer with the {\tt int()} function. For example,
this program computes an exponentially weighted moving average of the
light sensor:

\begin{quotation}
\begin{verbatim}

shared sensor; 
shared aggregate;

sensor = light();
aggregate = (aggregate / 2) + (int(sensor) / 2); ! Cast light reading to integer

\end{verbatim}
\end{quotation}

Buffers also have a type, which defines what values can be placed in
it. A cleared buffer has no type, and takes the type of the first
value placed in it. In the following program, {\tt aggBuffer} is
cleared, which clears its type. An integer ({\tt aggregate}) is added
to the buffer, making the buffer of the type integer.

\begin{quotation}
\begin{verbatim}

shared sensor;
shared aggregate;
buffer aggBuffer;

sensor = light();
aggregate = aggregate + int(sensor); ! Cast light reading to integer
aggregate = aggregate / 2;

bclear(aggBuffer); ! Clear all buffer entries and type
aggBuffer[] = aggregate; ! Append aggregate value to buffer
                         ! Buffer is now of type integer

\end{verbatim}
\end{quotation}

Buffers have a fixed maximum size of ten values. The function {\tt
bfull()} can be used to see if a buffer is full, while {\tt bsize()}
indicates how many entries it currently has. Individual buffer values
can be accessed by indexing into a buffer. The following program
obtains the median value stored in a buffer:

\begin{quotation}
\begin{verbatim}

shared size;
shared median;
buffer aggBuffer;
 
bsorta(aggBuffer);        ! Sort buffer entries in ascending order
size = bsize(aggBuffer);  ! Number of entries in buffer
median = aggBuffer[size / 2];  ! Return median value

\end{verbatim}
\end{quotation}

An empty index value implies the tail (last value) of a buffer on
access, or after the tail on assignment (append). For example:

\begin{quotation}
\begin{verbatim}

buffer aggBuffer;
shared val;

val = aggBuffer[];             ! Val is the last value in the buffer
aggBuffer[] = light();    ! Append a new light value to the buffer

\end{verbatim}
\end{quotation}

\section{Functions}

The above code examples used several functions, such as {\tt light()},
{\tt bsorta()}, and {\tt int()}. Functions take a fixed number (zero
or more) of parameters. For example, {\tt bclear()} takes a single
parameter, a buffer to clear, and {\tt rand()} takes no
parameters. Some functions return values (e.g., {\tt rand()}), while
others do not (e.g., {\tt bsorta()}).

Function parameters may have type requirements. However, as TinyScript
is dynamically typed, these types are not checked at compile time. For
example, {\tt bclear()} takes a single parameter, a buffer. Passing it
an integer will cause a run-time error.

Return values of function calls may be ignored. For example,

\begin{quotation}
\begin{verbatim}

rand();

\end{verbatim}
\end{quotation}

is a valid program.

The return values of functions can be directly used as values or
parameters to functions:

\begin{quotation}
\begin{verbatim}

buffer aggBuf;
shared val;

val = aggBuf[rand() \% bsize(aggBuff)]; ! Hope size isn't zero
val = sqrt(bsize(aggBuff)) + 2;         
\end{verbatim}
\end{quotation}

Note that assigning to a buffer is different than assigning to an
element of a buffer. Assigning a scalar to a buffer appends it. If a
scalar is assigned to an element of a buffer, the buffer is
dynamically sized to include that element if need be. Buffers can be
assigned to one another (which results in a copy), but a buffer cannot
be assigned to be the element of another buffer.

\begin{quotation}
\begin{verbatim}

buffer aggBuf;
buffer aggBuf2
shared val;

aggBuf2 = aggBuf;
aggBuf[] = val;
aggBuf2[val \% 8] = val;
aggBuf[1] = aggBuf2;  ! ERROR
\end{verbatim}
\end{quotation}

Currently, TinyScript does not support scripting functions.

\section{Arithmetic, Logic, and Conditionals}

Figure~\ref{fig:arithmetic} shows the set of arithmetic operations
TinyScript provides, as well as their syntax.

\begin{figure}
\centering
\subfigure[Arithmetic Operations]{\scriptsize
\begin{tabular}{|l|c|l|} \hline
Name           & Operator & Example \\ \hline
Addition       & + & val = val + 2;\\ \hline
Subtraction    & - & val = a - b; \\ \hline
Division       & / & val = bsize() / 2; \\ \hline
Multiplication & * & val = 2 * b; \\ \hline
Exponentiation & $^{\wedge}$ & val = val $^{\wedge}$ 2 \\ \hline
\end{tabular}
\label{fig:arithmetic}
}
\subfigure[Comparison Operations]{\scriptsize
\begin{tabular}{|l|c|l|} \hline
Name                  & Operator & Example \\ \hline
Less than             & $<$  & val = val $<$ 2;\\ \hline
Greater than          & $>$  & val = a $>$ b; \\ \hline
Less than or equal    & $<=$ & val = bsize() $<=$ 8; \\ \hline
Greater than or equal & $>=$ & val = b $>=$ 2; \\ \hline
Not equal             & $<>$ & cond = val $<>$ b \\ \hline
\end{tabular}
\label{fig:comparison}
}
\caption{TinyScript Computational Primitives}
\label{fig:tscriptcomp}
\end{figure}



TinyScript also supports logical operations, which are show in
Figure~\ref{fig:logic}. All of these operations only accept integers
as operands. For the boolean operators (e.g., and, not), a value of
zero is considered false; all other values are considered true. All
operators use 0 as false and 1 as true. So, {\tt 1 and 2} resolves to
1, while {\tt 0 and 34} resolves to 0. Figure~\ref{fig:tables}
contains the truth tables for the boolean operators.

\begin{figure}
\centering
\scriptsize
\subfigure[Logical Operations] {
\begin{tabular}{|l|c|l|} \hline
Name           & Operator & Example \\ \hline
And          & AND, and        & ready = full and idle;\\ \hline
Or           & OR, or          & ready = full OR idle;\\ \hline
Not          & NOT, not        & ready = not idle; \\ \hline
Exclusive or & XOR, xor        & diff = a XOR b; \\ \hline
Equivalent   & EQV, eqv        & rval = a eqv b; \\ \hline
Implies      & IMP, imp        & ready = a imp b; \\ \hline
Logical And  & \&              & bits = packetbits \& mask; \\ \hline
Logical Or   & $\mid$          & bits = firstbit $\mid$ secondbi t; \\ \hline
Logical Not  & $^{\sim}$       & mask = $^{\sim}$bits; \\ \hline
\end{tabular}
\label{fig:logic}
}
\subfigure[Truth Tables] {
\begin{tabular}{|c||c|c||c|c||c|c||}\hline
&\multicolumn{2}{|c||}{and}&\multicolumn{2}{c||}{or}&\multicolumn{2}{c||}{not} \\ \hline
  & F & T & F & T & \multicolumn{2}{|c||}{} \\ \hline
F & {\bf F} & {\bf F} & {\bf F} & {\bf T} & \multicolumn{2}{|c||}{{\bf T}} \\ \hline
T & {\bf F} & {\bf T} & {\bf T} & {\bf T} & \multicolumn{2}{|c||}{{\bf F}} \\ \hline \hline
&\multicolumn{2}{|c|}{xor}&\multicolumn{2}{|c|}{eqv}&\multicolumn{2}{|c|}{imp} \\ \hline
  & F & T & F & T & F & T \\ \hline
F & {\bf F} & {\bf T} & {\bf T} & {\bf F} & {\bf T} & {\bf T} \\ \hline
T & {\bf T} & {\bf F} & {\bf F} & {\bf T} & {\bf F} & {\bf T} \\ \hline
\end{tabular}
}
\caption{TinyScript Logical Primitives}
\label{fig:tables}
\end{figure}

The logical operations manipulate integer bit fields. Instead of
manipulating the integer as a singe value, they operate on each bit,
in a manner similar to C operators. For example, {\tt 1 and 2}
resolves to 1, while {\tt 1 \& 2} resolves to zero (1 and 2 share no
common bits), and {\tt 1 $\mid$ 2} resolves to 3.

Finally, TinyScript has standard comparison operators, as shown in
Figure~\ref{fig:comparison}. They resolve to one if true, zero if
false.

\subsection{Control Structures}

TinyScript supports standard language control structures such as
conditionals and loops.

The first set of control structures, conditionals, take this form:

\begin{quotation}
\begin{verbatim}

if <expression> then                if <expression> then
 <block 1>                           <block 1>
end if                              else 
                                     <block 2>
                                    end if
\end{verbatim}
\end{quotation}


If expression resolves to true, then block 1 executes. If the
statement has an else clause and expression resolves to false, then
block 2 executes. There can be nested if-then statements:

\begin{quotation}
\begin{verbatim}

shared idle;
buffer buf;

if bfull(buf) then
  idle = 0;
  if rand() & 1 then
    send(buf);
    bclear(buf);
  end if
  idle = 1;
end if
\end{verbatim}
\end{quotation}

TinyScript provides loops through the {\tt for} construct. There are
two basic forms, unconditional and conditional. Unconditional (for-to)
loops run a specific number of times; their termination condition when
the loop variable takes a specific value. Conditional (for-until)
loops run until an arbitrary condition becomes true. {\tt next}
defines the end of the loop block, and increments the loop
variable. By default, the variable increments by one. However, the
increment step can be set with the {\tt step} keyword. In summary:

\begin{quotation}
\begin{verbatim}

for <x> = <expression> to <to-constant>
  ...
next <x>

for <x> = <expression> to <to-constant> step <step-constant>
  ...
next <x>

for <x> = <expression> until <until-exp>
  ...
next <x>

for <x> = <expression> step <step-constant> until <until-exp>
  ...
next <x>

\end{verbatim}
\end{quotation}

For example, this loop will run one hundred times, blinking the leds,

\begin{quotation}
\begin{verbatim}

private i;

for i = 1 to 100
  leds(i & 7)
next i
\end{verbatim}
\end{quotation}

while this loop will put the values 1,3,5...21 in the buffer (when it has
ten values, it will be full),

\begin{quotation}
\begin{verbatim}

private i;
buffer buf;

bclear(buf);
for i = 1 step 1 until i > 10
  buf[] = i * 2;
next i
\end{verbatim}
\end{quotation}

Standard while loops can be implemented by setting a step of
zero. This loop, for example, will put random values into a buffer
until it is full:

\begin{quotation}
\begin{verbatim}

private i;
buffer buf;

for i = 0 step 0 until bfull(buf)
  buf[] = rand();
next i
\end{verbatim}
\end{quotation}

Parenthesis pairs can be added to define precedence, or for readability.

\begin{quotation}
\begin{verbatim}

private i;
buffer buf;

bclear(buf);
for i = 1 step 1 until i > 10
  buf[] = ((i * 2) + 1);
next i
i = (5 + 2 * 2);    ! i = 9
i = (5 + 2) * 2;    ! i = 14
i = ((((5))));      ! i = 5

\end{verbatim}
\end{quotation}


\mate is under active development. Bugs, contexts, and functions can be sent
to Phil Levis ({\tt pal@cs.berkeley.edu}).

\section{Appendix A: Grammar}
\em \begin{tabbing}
\grammarindent
TinyScript-file: \\
\>  variable-list\opt statement-list\opt\\
\\
variable-list:\\
\> variable \\
\> variable-list variable\\
\\
variable:\\
\> \kw{shared} identifier\\
\> \kw{private} identifier\\
\> \kw{buffer} identifier\\
\\
statement-list:\\
\> statement\\
\> statement-list statement\\
\\
statement:\\
\> assignment\\
\> control-statement\\
\> function-statement \\
\\
control:\\
\> if-statement\\
\> for-unconditional\\
\> for-conditional\\
\\
if-statement:\\
\> \kw{if} expression \kw{then} then-clause \kw{end if}\\
\\
then-clause:\\
\> statement-list\\
\> statement-list \kw{else} statement-list \\
\\
for-unconditional:\\
\> \kw{for} variable-ref \kw{=} expression \kw{to} constant-expression step statement-list \kw{next} variable-ref\\
\\
for-conditional:\\
\> \kw{for} scalar-ref \kw{=} constant-expression step for-condition statement-list \kw{next} scalar-ref\\
\\
for-condition:\\
\> \kw{until} expression\\
\> \kw{while} expression\\
\\
constant-expression:\\
\> constant\\
\\
function-statement:\\
\> function-call \kw{;}\\
\\
function-call:\\
\> name \kw{(} parameter-list\opt \kw{)}\\
\\
parameter-list:\\
\> parameter\\
\> parameter-list \kw{,} parameter\\
\\
parameter:\\
\> expression\\
\\
assignment:\\
\> l-value \kw{=} expression \kw{;} \\
\\
l-value:\\
\> var-ref\\
\> buffer-ref\\
\\
scalar-ref:\\
\> identifer\\
\\
var-ref:\\
\> identifier\\
\\
buffer-ref:\\
\> identifier \kw{[]}\\
\> identifier \kw{[} expression \kw{]}\\
\\
expression:\\
\> function-call\\
\> constant-expression\\
\> variable-expression\\
\> paren-expression\\
\> unary-expression\\
\> binary-expression\\
\> conditional-expression\\
\\
variable-expression:\\
\> buffer-access\\
\> variable-access\\
\\
buffer-access:\\
\> identifier \kw{[]}\\
\> identifier \kw{[} expression \kw{]}\\
\\
variable-access:\\
\> identifier\\
\\
paren-expression:\\
\> \kw{(} expression \kw{)}\\
\\
unary-expression:\\
\> \kw{-} expression\\
\> $\approx$ expression\\
\> \kw{NOT} expression\\
\\
binary-expression:\\
\> expression \kw{+} expression\\
\> expression \kw{-} expression\\
\> expression \kw{*} expression\\
\> expression \kw{/} expression\\
\> expression \kw{\%} expression\\
\> expression \kw{\&} expression\\
\> expression \kw{|} expression\\
\> expression \kw{\#} expression\\
\> expression \kw{AND} expression\\
\> expression \kw{XOR} expression\\
\> expression \kw{OR} expression\\
\> expression \kw{EQV} expression\\
\> expression \kw{IMP} expression\\
\\
conditional-expression:\\
\> expression $=$ expression\\
\> expression $!=$ expression\\
\> expression $>=$ expression\\
\> expression $>$ expression\\
\> expression $<=$ expression\\
\> expression $<$ expression\\
\\
constant:\\
\> \kw{[1-9][0-9]}*\\
\\
identifier:\\
\> \kw{[A-Za-z\_][A-Za-z\_0-9]*}\\

\end{tabbing}

\end{document}
